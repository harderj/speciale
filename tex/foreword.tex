
I came upon the idea to write about this subject because 
I was fascinated by the performance of Q-learning algorithms
implemented by [Mnih et al.].

Coming from a background of mixed pure mathematics and 
probability theory the
main purpose of this master thesis was initially
to investigate what has been proven
about the convergence of Q-learning algorithms.
In particular Q-learning algorithms using (deep) ANNs.

In the course of this I discovered that the frameworks
and settings in which various Q-leaning algorithms are analysed
varies greatly across litterature.
Also questions as to in which degree optimal strategies
exist in these various frameworks turns out to be
non-trivial when the state and action spaces are uncountable.

Therefore this paper is partially about building a framework
for analysing Q-learning algorithms in a variety of settings.
And partially to present the results that occur in each setting
and discuss their importance and generality.
