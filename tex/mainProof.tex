
\begin{proof}[Proof of main theorem (\ref{thm:main})] %todo write intro
  Using \cref{thm:errorprop} we get
  \begin{equation}
    \norm{Q^* - Q_{\pi_K}}_{1,\mu} \leq
    \frac{2 \phi_{\mu, \nu} \gamma}{(1-\gamma)^2}\varepsilon_{\max} +
    \frac{4 \gamma^{K+1}}{(1-\gamma)^2} R_{\max}
    \label{eq:mp1}
  \end{equation}
  where $\varepsilon_{\max} =
  \max_{k \in [K]} \norm{T \wt{Q}_{k-1} - \wt{Q}_k}_{2, \nu}$.
  %todo: specify constants for \Cal{F}_0 !
  Using \cref{thm:oneStep} with $Q = \wt{Q}_{k-1}$ and
  $\Cal{F} = \Cal{F}_0$, we get
  \begin{align}
    \varepsilon_{\max} \leq & 
    C_5 \log(N_{(1/n)})/n
    + 2 \omega(\cl{F})
    + C_3 \sqrt{\log(N_{(1/n)})/n} + C_6 n^{-1}
    \label{eq:mp2}
  \end{align}
  where $N_0 = N_{(1/n)} = \abs{\Cal{N}(1/n, \Cal{F}_0, \norm{\cdot}_\infty)}$.
  It remains only to bound $\omega(\Cal{F}_0)$ and $N_0$.
  We start with $\omega(\Cal{F}_0)$.

  \textbf{Step 1}.
  We want to employ the lemma by \ncite{SH17}
  (\cref{lem:holderapprox})
  to each Hölder smooth part of $g = Tf$ where $f$ is an arbitrary network
  in $\cl{F}_0$
  and then piece it together somehow,
  using that ReLU networks are easily stitched together into bigger
  ReLU networks (see \cref{rem:annGraph}).
  Therefore the first step is to refit our
  Hölder Smooth compositions in $\Cal{G}_0$ to be defined on the unit cube in
  the respective dimensions instead.
  This is a relatively simple procedure:

  Let $f \in \Cal{G}_0$ then $f(\cdot, a) \in
  \Cal{G}(\{p_j, t_j, \beta_j, H_j\})$ for all $a \in \Cal{A}$.
  Therefore $f(\cdot, a) = g_q \circ \dots \circ g_1$ where
  the (sub-)components $(g_{jk})_{k=1}^{p_{j+1}} = g_j$ satisfy
  \begin{equation}
    g_{jk} \in C_{t_j}([a_j, b_j]^{t_j}, \beta_j, H_j)
    , \qquad j \in [q], k \in [p_{j+1}]
  \end{equation}
  Here $a_1 = 0, b_1=1$ and,
  $a_j < b_j \in \R$ are some real numbers for $2 \leq j \leq q$.
  Notice that the Hölder smooth condition implies that
  $g_{jk}([a_j, b_j]^{t_j}) \subseteq [-H_j, H_j]$.
  Define
  \begin{align}
    h_1 = & g_1/(2H_1) + 1/2 \notag
    \\ h_j(u) = & g_j(2H_{j-1} u - H_{j-1})/(2H_j) + 1/2,
    & j \in \{2, \dots, q-1\} \notag
    \\ h_q(u) = & g_q(2H_{q-1}u - H_{q-1})
  \end{align}
  Then $g_q \circ \dots \circ g_1 = h_q \circ \dots \circ h_1$ and
  \begin{align}
    h_{1k} \in &\; C_{t_1}([0,1]^{t_1}, \beta_1, 1) \notag
    \\ h_{jk} \in &\; C_{t_j}([0,1]^{t_j}, \beta_j, (2H_{j-1})^{\beta_j}),
    & j \in \{2, \dots, q-1\} \notag
    \\ h_q \in &\; C_{t_q}([0,1]^{t_q}, \beta_q, H_q(2H_{q-1})^{\beta_q})
  \end{align}
  This concludes our construction of the refit of the components of $g$
  to unit intervals.

  \textbf{Step $\frac{3}{2}$}

  Define 
  $N \defeq \max_{j \in [q]} n^{t_j/(2 \beta^*_j + t_j)}$
  $\eta \defeq \log \left((2W + 1) 6^{t_j} N
  / (W 3^{\beta_j} N^{-\beta_j/t_j}) \right)$,
  and $m \defeq \eta \ceil{\log_2 n}$,
  and assume $n$ is sufficiently large such that
  $N \geq \max \left\{ (\beta_j + 1)^{t_j},
  (H_j+1)e^{t_j} \mid j \in [q] \right\}$.
  \begin{equation}
    W \defeq \max \left( \left\{ (2 H_{j-1})^{\beta_j}
      \mid 1\leq j\leq q-1 \right\}
    \cup \left\{ H_q(2H_{q-1})^{\beta_q}, 1\right\} \right)
  \end{equation}
  By \cref{lem:holderapprox} there exists a ReLU network
  \begin{equation}
    \wh{h}_{jk} \in \Cal{SRN} \left( (\wt{s}_j + 4) \cdot p_{j+1}, V_{\max}, \;
    ( t_j, \wt{d}_j p_{j+1}, \dots,
  \wt{d}_j p_{j+1}, p_{j+1}), L_j + 2 \right)
  \end{equation}
  where $\wt{d}_j = 6(t_j + \ceil{\beta_j})N$ and
  $\wt{s}_j \leq 141 (t_j + \beta_j + 1)^{3 + t_j} N (m + 6)$
  such that
  \begin{equation}
    \norm{\wh{h}_{jk} - h_{jk}}_\infty \leq (2 W + 1) 6^{t_j} N 2^{-m}
    + W 3^{\beta_j} N^{-\beta_j/t_j} \leq 2 W 3^{\beta_j} N^{-\beta_j/t_j}
    \label{eq:hhatbound}
  \end{equation} 
  Since $h_{j+1}$ is defined on $[0, 1]^{t_{j+1}}$ but $\wt{h}_j$ takes values
  in $\R$ we need to restrict $\wt{h}_j$ somehow to stitch
  the two together (by function composition). This is easily done by
  \begin{lem}
    Restriction to $[0, 1]$ is expressible as a two-layer ReLU network
    with 4 non-zero weights.
  \end{lem}
  \begin{proof}
    This is the simple network
    $\min(0,\max(1, u)) = \sigma_r(1 - \sigma_r(1 - u))$
    where $\sigma_r(x) = \max(0,x)$ is the ReLU activation function.
    The weights are $w_1 = w_2 = -1$ and $v_1 = v_2 - 1$.
  \end{proof}
  Now define
  $\wt{h}_{jk} = \tau \circ \wh{h}_{jk}$
  (and $\wt{h}_{j} = (\wt{h}_{jk})_{k \in [p_{j + 1}]}$).
  Then
  \begin{equation}
    \wt{h}_{jk} \in \Cal{SRN}\left( (\wt{s}_j + 4) p_{j+1},
    V_{\max}, (t_j, \wt{d}_j, \dots, \wt{d}_j, 1), L_j + 2 \right)
  \end{equation}
  and since $h_{jk}([0, 1]^{t_j}) \in [0, 1]$ by \cref{eq:hhatbound} we have
  \begin{align}
    \norm{\wt{h}_{jk} - h_{jk}}_\infty
    = \; \norm{\tau \circ \wh{h}_{jk} - \tau \circ h_{jk}}_\infty
     \leq \; \norm{\wh{h}_{jk} - h_{jk}}_\infty
     \leq \; 2 W 3^{-\beta_j} N^{-\beta_j/t_j}
  \end{align}
  Having employed \cref{lem:holderapprox}
  we now need to stitch it
  back together:

  \textbf{Step 2}.
  Now define $\wt{f}:\Cal{S} \to \R$
  as $\wt{f} = \wt{h}_1 \circ \dots \circ \wh{h}_1$.
  If we set
  $\wt{L} \defeq \sum_{j=1}^q (L_j + 2)$,
  $\wt{d} \defeq \max_{j \in [q]} \wt{d}_j p_{j+1}$
  and $\wt{s} \defeq \sum_{j=1}^q (\wt{s}_j + 4) p_{j+1}$.
  Then $\wt{f} \in \Cal{SRN} \left( \wt{s}, V_{\max},
  (w, \wt{d}, \dots, \wt{d}, 1), {\wt{L}} \right)$.
  We now take a moment to verify the size 
  of the constants involved in the network.
  Starting with $\wt{L}$.
  \begin{align*}
    \wt{L} \leq & \; \sum_{j=1}^q (L_j + 2)
    \\ = & \; \sum_{j=1}^q (8 + (\eta \ceil{\log_2 n} + 5)
    (1 + \ceil{\log_2(\beta_j + t_j)}))
    \\ \leq & \; \sum_{j=1}^q (8 + (\eta \log_2 n + \eta + 5)
    (2 + \log_2(\beta_j + t_j)))
    \\ \leq & \; 8q + (2 \eta + 5) \log_2(n)
    \sum_{j=1}^q (2 + \log_2(\beta_j + t_j)) 
    \\ \leq & \; 8q + (2 \eta + 5) \log_2(n)
    (2q + \log(n)^\xi) 
    \\ \leq & \; (10q + 1) (2 \eta + 5) \log_2(e) \log(n)^{1 + \xi}
    \\ \leq & \; C_{\wt{L}} \log(n)^{1 + 2\xi}
  \end{align*}
  where $C_{\wt{L}} = (10q + 1)(2\eta + 5)\log_2(e)$.
  For $\wt{d}$ we have
  \begin{align*}
    \wt{d} = & \; \max_{j \in [q]} \wt{d}_j p_{j+1}
    \\ = & \; \max_{j \in [q]} 6(t_j + \beta_j + 1) N p_{j+1}
    \\ \leq & \; 6 N (\max_{j \in [q]} p_j)
    (\max_{j \in [q]} (t_j + \beta_j + 1))
    \\ \leq & \; 6 N (\log n)^{2 \xi}
    \\ \leq & \; 6 n^{\alpha^*} (\log n)^{\xi^*}
  \end{align*}
  and for $\wt{s}$
  \begin{align*}
    \wt{s} = & \; \sum_{j=1}^q (\wt{s}_j + 4) p_{j+1}
    \\ \leq & \; \sum_{j=1}^q (141 N (m + 6)
    (t_j + \beta_j + 1)^{3 + t_j} + 4) p_{j+1}
    \\ \leq & \; 142 N (\log n)^\xi (2 \eta + 6) \log_2(n)
    \sum_{j=1}^q (t_j + \beta_j + 1)^{3 + t_j}
    \\ \leq & \; 142 N (\log n)^\xi (2 \eta + 6) \log_2(e) \log(n)
    (\log n)^\xi
    \\ = & \; 142 N (2 \eta + 6) \log_2(e) (\log n)^{1+ 2\xi}
    \\ = & \; C_{\wt{s}} n^{a^*} (\log n)^{\xi^*}
  \end{align*}
  where $C_{\wt{s}} = 142(2 \eta + 6) \log_2(e)$.
  Now we bound $\norm{\wt{f} - f(\cdot, a)}_\infty$.
  Define $G_j = h_j \circ \dots \circ h_1$
  , $\wt{G}_j = \wt{h}_j \circ \dots \circ \wt{h}_1$
  for $j \in [q]$,
  $\lambda_j = \prod_{\ell=j+1}^q (\beta_\ell \land 1)$
  for all $j \in [q-1]$ and $\lambda_q = 1$. We have
  \begin{align*}
    \norm{G_j - \wt{G}_j}_\infty
    = & \; \norm{h_j \circ G_{j-1} - h_j \circ \wt{G}_{j-1}
    + h_j \circ \wt{G}_{j-1} - \wt{h}_j \circ \wt{G}_{j-1}}
    \\ \leq & \; \norm{h_j \circ \wt{G}_{j-1} - h_j \circ G_{j-1}}_\infty
    + \norm{h_j \circ \wt{G}_{j-1} - h_j \circ G_{j-1}}_\infty
    \\ \leq & \; W \norm{G_{j-1} - \wt{G}_{j-1}}_\infty^{\beta_j \land 1}
    + \norm{\wt{h}_j - h_j}_\infty^{\lambda_j}
  \end{align*}
  so by induction and \cref{eq:hhatbound}
  \begin{align*}
    \norm{f(\cdot, a) - \wt{f}}_\infty = & \; \norm{G_q - \wt{G}_q}_\infty
    \\ \leq & \; W^q \sum_{j-1}^q \norm{\wt{h}_j - h_j}_\infty^{\lambda_j}
    \\ \leq & \; W^q \sum_{j-1}^q
    \left( 2 W 3^{\beta_j} N^{-\beta_j/t_j} \right)^{\lambda_j}
    \\ \leq & \; 2 q 3^{\max_{j \in [q]} \beta^*_j} W^{q+1} 
    \max_{j \in [q]} N^{-\beta^*_j/t_j}
    \\ \leq & \; c_N^{1/2} \max_{j \in [q]} n^{-\alpha^* \beta^*_j / t_j}
    \\ \leq & \; c_N^{1/2} n^{-\alpha^* \min_{j \in [q]} \beta^*_j / t_j}
  \end{align*}
  and therefore
  \begin{align*}
    \omega(\Cal{F}_0)
     & \leq \; C_N n^{-2 \alpha^* \min_{j \in [q]} \beta^*_j / t_j}
     \leq \; C_N n^{-2 \alpha^* \kappa^*}
    \numberthis \label{eq:boundOmega}
  \end{align*}
  where we define $\kappa^* = \min_{j \in [q]} \beta^*_j/t_j$.
  
  \textbf{Step 3}.
  Finally what is left is to bound the covering number of $\Cal{F}_0$.
  Denote by $\Cal{N}_\delta$ the $\delta$-covering of
  $\cl{SRN}\left( \wt{s}, V_{\max},
  (\wt{d}_j)_{j=1}^{\wt{L} + 1}, \wt{L} \right)$ by
  \[ \Cal{N}_\delta \defeq \Cal{N} \left( \delta,
      \cl{SRN}\left( \wt{s}, V_{\max},
  (\wt{d}_j)_{j=1}^{\wt{L} + 1}, \wt{L} \right),
  \norm{\cdot}_\infty \right) \]
  Since $\Cal{N}_\delta$ is a covering, for any $f \in \Cal{F}_0$
  and $a \in \Cal{A}$ you can find a $g_a \in \Cal{N}_\delta$ such that
  $\norm{f(\cdot, a) - g_a}_\infty < \delta$. Now let
  $g:\Cal{S} \times \Cal{A} \to \R = (s, a) \mapsto g_a(s)$.
  Then $\norm{f - g}_\infty < \delta$, so we can bound the covering number
  of $\Cal{F}_0$ by
  \[ \abs{\Cal{N}(\delta, \Cal{F}_0,  \norm{\cdot}_\infty)}
  \leq \abs{\Cal{N}_\delta}^{\abs{\Cal{A}}} \]
  We now utilize a lemma found in \mcite{AB02}
  \begin{lem}[Covering number of ReLU networks]
    Consider the family of ReLU networks
    \[ \Cal{SRN}\left( s, V_{\max}, (d_j)_{j=0}^{L+1}, L \right) \]
    where $\Cal{SRN}$ is defined in \cref{defn:sparseReLU}.
  Let $D \defeq \prod_{\ell=1}^{L+1} (d_\ell + 1))$. Then for any $\delta > 0$
  \[ \Cal{N} \left(\delta,
      \Cal{SRN}\left( s, V_{\max}, (d_j)_{j=0}^{L+1}, L \right),
  \norm{\cdot}_\infty \right) \leq (2 (L+1) D^2 / \delta)^{s + 1} \]
  \label{lem:ABcovering}
  \end{lem}
  \begin{proof}
    We refer to theorem $14.5$ in \ncite{AB02}.
  \end{proof}
  With \cref{lem:ABcovering} and $n$ sufficiently large we can bound
  \begin{align*}
    \log N_0 = & \; \log \abs{\Cal{N}(1/n, \Cal{F}_0, \norm{\cdot}_\infty)}
    \\ \leq & \; \abs{\Cal{A}} \cdot \log \abs{\Cal{N}_{1/n}}
    \\ \leq & \; \abs{\Cal{A}} (\wt{s} + 1) \log (2 (\wt{L} + 1) \wt{D}^2 n)
    \\ \leq & \; \abs{\Cal{A}} (c_{\wt{s}} n^{\alpha^*} \log(n)^{\xi^*} + 1)
    2 \log \left(2 (c_{\wt{L}} \log(n)^{\xi^*} + 1) \prod_{\ell=1}^{\wt{L} + 1}
    (\wt{d} + 1) \right)
    \\ \leq & \; 2 \abs{\Cal{A}} (c_{\wt{s}} n^{\alpha^*} \log(n)^{\xi^*} + 1)
    \log \left(2 (c_{\wt{L}} \log(n)^{\xi^*} + 1)
    (6 n^{\alpha^*} \log(n)^{\xi^*} + 1)^{\wt{L} + 1} \right)
    \\ \leq & \; 4 \abs{\Cal{A}} c_{\wt{s}} n^{\alpha^*} \log(n)^{\xi^*}
    (\wt{L} + 1) \log \left(24 c_{\wt{L}} \log(n)^{\xi^*}
    n^{\alpha^*} \log(n)^{\xi^*} \right)
    \\ \leq & \; 8 \abs{\Cal{A}} c_{\wt{s}} n^{\alpha^*} \log(n)^{\xi^*}
    c_{\wt{L}} \log(n)^{\xi^*} (\alpha^* + 2) \log(n)
    \\ = & \; 8 c_{\wt{s}} c_{\wt{L}} (\alpha^* + 2)
    n^{\alpha^*} \log(n)^{1 + 2\xi^*}
    \\ = & \; c_{N_0} (\alpha^* + 2) 
    n^{\alpha^*} \log(n)^{1 + 2\xi^*}
    \numberthis \label{eq:boundN0}
  \end{align*}
  Where we define $c_{N_0} = 8 c_{\wt{s}} c_{\wt{L}}$.
  Using \cref{eq:boundN0}, \cref{eq:boundOmega} and
  \cref{eq:mp2} we can bound 
  \begin{align}
    \ve_{\max} \leq C_5 C_{N_0} n^{\alpha^* - 1}/2 \log(n)^{1 + 2\xi^*}
    + \sqrt{C_5 C_{N_0} n^{\alpha^* - 1} \log(n)^{1 + 2\xi^*}}
    + 2 c_N n^{-2\alpha^* \kappa^*} + C_6 n^{-1}
  \end{align}
  Since we are interested in convergence (to 0) we may assume that
  $n$ is sufficiently large such that only the largest exponent in the
  above expression is significant. This leads to the simplification
  \begin{align}
    \ve_{\max} & \leq (2 C_5 C_{N_0} + C_3 + 2 c_N + C_6)
    n^{\max\{ (\alpha^* - 1)/2, -2\alpha^* \kappa^*\} } \log(n)^{1 + 2\xi^*}
    \\ & = C_7 n^{\max\{ (\alpha^* - 1)/2, -2\alpha^* \kappa^*\}}
    \log(n)^{1 + 2\xi^*} 
    \label{eq:boundemaxnew}
  \end{align}
  where $C_7 \defeq (2 C_5 C_{N_0} + C_3 + 2 c_N + C_6)$.
  Now using \cref{eq:mp1} and \cref{eq:boundemaxnew}
  \begin{align*}
    \norm{Q^* - Q_{\pi_K}}_{1, \mu} \leq & \;
    C_7 \frac{\phi_{\mu, \nu} \gamma}{(1-\gamma)^2} V_{\max}^2
    n^{\max\{-2\alpha^*\kappa^*, (\alpha^*  - 1)/2 \}} \log(n)^{1+2\xi^*}
    + \frac{4 \gamma}{(1-\gamma)^2} R_{\max} \gamma^K
  \end{align*}
  where $C_7$ only depends on the constants in \cref{asm:A2}
  finishing the proof.
\end{proof}
