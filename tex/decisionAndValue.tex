
In this section we will develop general theory about
decision processes and value function (including Q-functions)
that is used across all sources considered in this paper,
including the question of optimal policy existence.

\subsection{History dependent decision process}
We define in this section a quite general framework.
We do this partly in the quest to have a united framework
to talk about results from a variety of sources,
and relate them to each other in generality.
And partly to avoid defining various concepts such as value functions
everytime a new context is considered.
A source which uses a setup which is almost as general can be found in
[ref. to Schal].
In this section recall that $\ul{\R} = \R \cup \{-\infty\}$,
$\ol{\R} = \R \cup \{\infty\}$ and
$\ol{\ul{\R}} = \R \cup \{\pm \infty\}$.

\begin{defn}[History dependent decision process]
  A \defemph{history dependent decision process} (HDP) is determined by
  \begin{enumerate}
    \item $(\Cal{S}_n, \Sigma_{\Cal{S}_n})_{n \in \N}$ a 
      measurable space of \defemph{states} for each timestep.
    \item $(\Cal{A}_n, \Sigma_{\Cal{A}_n})_{n \in \N}$ a 
      measurable space of \defemph{actions} for each timestep.
  \end{enumerate}
  for each $n \in \N$ we define the so called \defemph{history} spaces
  \[ \Cal{H}_1 = \Cal{S}_1, \quad
    \Cal{H}_2 = \Cal{S}_1\times \Cal{A}_1\times \Cal{S}_2,
    \quad \Cal{H}_3 = \Cal{S}_1 \times \Cal{A}_1 \times \Cal{S}_2 \times
  \Rext \times \Cal{A}_2 \times \Cal{S}_3 \]
  \[ \Cal{H}_n = \Cal{S}_1 \times \Cal{A}_1
    \times \Cal{S}_2 \times \ol{\ul{\R}} \times \Cal{A}_2
  \times \Cal{S}_3 \times \ol{\ul{\R}}\times \dots \times \Cal{S}_n \]
  \[
    \Cal{H}_\infty = \Cal{S}_1 \times \Cal{A}_1 \times \Cal{S}_2 \times
    \ol{\ul{\R}} \times \dots
  \]
  with associated product $\sigma$-algebras
  \begin{enumerate} \setcounter{enumi}{2}
    \item $(P_n)_{n \in \N}$ a sequence of
      $\Cal{H}_n \times \Cal{A}_n \leadsto \Cal{S}_{n+1}$ kernels
      called the \defemph{transition} kernels.
    \item $(R_n)_{n \in \N}$ a sequence of
      $\Cal{H}_{n+1} \leadsto \ol{\ul{\R}}$ kernels
      called the \defemph{reward} kernels.
  \end{enumerate}
  \label{sett:HDP}
\end{defn}
The name \emph{decision process} is used for many different processes
across litterature but many of them generalize to the above.
Some authors also use the name \emph{dynamic progamming model}.

Notice the slight irregularity in the beginning of the history spaces:
We are missing a reward state after $\Cal{S}_1$. This could have been avoided
by introducing some start reward but we will do without.

\begin{asm}(Reward independence)
  $P_n, R_n$ and policies are only allowed to depend on the past
  states and actions, and not the rewards.
  \label{asm:rewardIndep}
\end{asm}

In all sources known to this writer \cref{asm:rewardIndep} is assumed.
This is a bit of a puzzle since it is obvious that one could
want to define algorithms (policies) that take into account which rewards
they received in the past.
We will also do this but stick to the standard and 
never attempt to evaluate ideal value functions of
policies that depend on rewards.
Thus we will let \cref{asm:rewardIndep} hold
from now on and throughout this paper.

The majority of sources considered in this paper also specialize
with the following:
\begin{asm}[One state and action space]
  $\Cal{S}_1 = \Cal{S}_2 = \dots \defeq \Cal{S}$,
  $\Cal{A}_1 = \Cal{A}_2 = \dots \defeq \Cal{A}$
  \label{asm:oneStateActionSpace}
\end{asm}
We will do without this for the rest of this section in order to
present some results in the generality they deserve.
Later we will look at settings which do not specialize this way.
One could ask if it is possible to embed the general decision process into one
with \cref{asm:oneStateActionSpace} by setting
$\Cal{S} \defeq \bigcup_{i\in\N} \Cal{S}_i$ and
$\Cal{A} \defeq \bigcup_{i\in\N} \Cal{A}_i$ or similar.
One attempt at this can be found in \ncite{BS07} chapter 10,
but this will not be covered here. %consider covering it

Other ways to specialize include reducing one or both of the
transition and reward kernels to functions defined on
$\Cal{S} \times \Cal{A}$. These processes are often called
\emph{deterministic}, but the exact definitions vary across sources, and we will
instead specify each setting individually.

For a decision process we can define
\begin{defn}[Policy]
  A (randomized) \defemph{policy} $\pi = (\pi_n)_{n \in \N}$
  is a sequence of $\Cal{H}_n \leadsto \Cal{A}_n$ kernels.
  The set of all policies we denote $R\Pi$.
  The policy $\pi$ is called \defemph{semi Markov} if each $\pi_i$ only depends
  on the first and last state in the history
  and is called \defemph{Markov} if only the last.
  The sets are denoted $sM\Pi$ and $M\Pi$.
  Furthermore $\pi$ is called \defemph{deterministic} if all $\pi_i$
  are degenerate, i.e. for all $i$ we have
  $\pi_i(\{a_i\} \mid h_i) = 1$ for some $a_i \in \Cal{A}_{i}$.
  Under \cref{asm:oneStateActionSpace}
  it makes sense to make a (Markov) policy $(\pi, \pi, \dots)$,
  where $\pi$ only depends on the last state.
  Such a policy is called \defemph{stationary},
  and the set of them denoted $S\Pi$.
  We denote the deterministic version of the policy classes
  by the letter $D$.
\end{defn}
We have the following inclusions
\[ \begin{matrix}
  S\Pi &\subseteq M\Pi &\subseteq sM\Pi &\subseteq R\Pi
  \\ \rleft{\subseteq} & \rleft{\subseteq} & \rleft{\subseteq} & \rleft{\subseteq} 
  \\ DS\Pi &\subseteq DM\Pi &\subseteq DsM\Pi &\subseteq D\Pi
\end{matrix} \] 

\begin{prop}
A dynamic progamming model together with a policy $\pi$ defines a
probability kernel $\kappa_\pi : \Cal{S}_1 \to \Cal{H}_\infty$.
\end{prop}
\begin{proof}
  This is the Ionescu-Tulcea kernel generated by
  $\dots R_2 P_2 \pi_2 R_1 P_1 \pi_1$.
\end{proof}
This kernel yields a probability measure $\kappa_\pi \mu$ on $\Cal{H}_\infty$
for every $\mu \in \Cal{S}_1$. In particular for any $s \in \Cal{S}_1$
$\kappa_\pi \delta_s$ yields the measure $\kappa_\pi(\cdot \mid s)$
and we shall occasionally write this $\kappa_\pi s$ and
integration with respect to it $\E^\pi_s$.

Across litterature generally %todo: backup this claim
any function mapping a state space $\Cal{S}$ to $\ol{\ul{\R}}$
can be called a (state)
\defemph{value} function. Similarly any $\ul{\ol{\R}}$ valued function
on pairs of states and actions can be called (state)
\defemph{action value} or \defemph{Q}- function.
The idea behind such functions are commonly to estimate the
cumulative rewards associated with a state or state-action pair
and the trajectory of states it can lead to.
In order to define some standard value functions
we will need one of
the following conditions:

\begin{cond}{$F^-$}[Reward finity from above]
  $\int_{[0,\infty]} x \difd R_i(x \mid h) < \infty$ for all
  $h \in \Cal{H}_{i+1}$ and $i \in \N$
  \label{cond:F-}
\end{cond}
\begin{cond}{$F^+$}[Reward finity from below]
  $\int_{[-\infty,0]} x \difd R_i(x \mid h) > -\infty$ for all
  $h \in \Cal{H}_{i+1}$ and $i \in \N$
  \label{cond:F+}
\end{cond}
The letter \emph{F} comes from \ncite{BS07}.
When assuming either of (\cref{cond:F+}) or (\cref{cond:F-})
we ensure that the summation of finitely many rewards
has a well defined mean in $\Rext$,
and then the following definition makes sense 
\begin{defn}[Finite horizon value function]
  Let $\ul{R}_i : \Cal{H}_\infty \to \ol{\ul{\R}}$ be the projection onto the
  $i$th reward. Define
  \[ V_{n,\pi}(s) = \E_s^\pi \sum_{i=1}^n \ul{R}_i \]
  called the $k$th finite horizon value function.
  When $n=0$ we say $V_{0,\pi} = V_0 \defeq 0$ for any $\pi$.
\end{defn}
The finite horizon value function
measures the expected total reward of starting in state $s$
and then follow the policy $\pi$ for $n$ steps.
This way it measures the \emph{value} of that particular state
given a policy and \emph{horizon} (number of steps).
We would like to extend this to an infinite horizon value function,
i.e. letting $n$ tend to $\infty$. To ensure that the integral is well-defined
we need one of the following conditions
\begin{cond}{P}[Reward non-negativity] $R_i([0,\infty] \mid h) = 1$,
  $\forall h \in \Cal{H}_{i+1}, i \in \N$
  \label{cond:P}
\end{cond}
\begin{cond}{N}[Reward non-positivity] $R_i([-\infty, 0] \mid h) = 1$
  $\forall h \in \Cal{H}_{i+1}, i \in \N$
  \label{cond:N} 
\end{cond}
\begin{cond}{D}[Discounting] There exist a bound $R_{\max} > 0$ and a
  $\gamma \in [0,1)$ called the \defemph{discount} factor such that
  $R_i([-R_{\max} \gamma^i, R_{\max} \gamma^i]) = 1$
  $\forall h \in \Cal{H}_{i+1}, i \in \N$
  \label{cond:D}
\end{cond}
Again the letters P, N and D are adopted from \ncite{BS07}.
\begin{defn}
  We define the infinite horizon value function by
  \[ V_\pi(s) = \E_s^\pi \lim_{n \to \infty} \sum_{i=1}^n \ul{R}_i \]
\end{defn}
The infinite horizon value function $V_\pi$ measures the expected total
reward after following the policy $\pi$ an infinite number of steps.

\begin{rem}
  Whenever we are working with the finite horizon value function
  we will always assume that either ($F^+$) or ($F^-$) holds without
  stating this explicitly.
  If a result only holds under e.g. ($F^+$) we will of course be explicit
  about this be marking it accordingly with a ($F^+$).

  Similarly whenever we work with the infinite horizon value function we will
  always assume that at least one of (P), (N) or (D) holds.
  We will mark propositions and theorems
  by e.g. (\cref{cond:D}) (\cref{cond:P}) when
  the result only holds for if discounting \emph{or} reward non-negativity
  is assumed.
  Note that obviously (P) implies $F^+$ and (N) implies $F^-$.
\end{rem}

\begin{rem}
Since we are under \cref{asm:rewardIndep}, when talking about the finite
or infinite value functions,
we can actually reduce the reward kernels to functions
$r_i : \Cal{H}_{i+1} \to \Rext = h \mapsto \int r \difd R_i(r \mid h)$
(note that $r_i$ is measurable due to \cref{prop:intKerMeas}).
Another way of stating this is that the value functions are indifferent
to whether we use deterministic or stochastic rewards.
This however does not mean that we can dispose completely of stochastic
rewards, as they still make a difference to model-free algorithms that
do not know the reward kernel, and therefore cannot simply integrate it.
\end{rem}

For use later we mention some properties of these value functions.
\begin{prop} 
  When well-defined the value functions $V_{n,\pi}, V_\pi$ are measurable
  into $(\ul{\ol{\R}}, \ul{\ol{\bb{B}}})$.
\end{prop}
\begin{proof}  
  Use \cref{prop:intKerMeas}.
\end{proof}

\begin{prop}
  $\lim_{n\to\infty} V_{n, \pi} = V_\pi $
  for all $\pi \in R\Pi$.
  \label{prop:VnLimV}
\end{prop}
\begin{proof}
  By monotone or dominated convergence.
\end{proof}

\begin{prop} Under (D) for any $\pi \in R\Pi$ we have

  $\abs{V_{n,\pi}}, \abs{V_\pi} \leq R_{\max} (1 - \gamma) < \infty$.
  \label{prop:Vbounded}
\end{prop}
\begin{proof}
  For any $\pi \in R\Pi$
  \[ \abs{V_\pi(s)} \leq \E_s^\pi \sum_{i \in \N} \abs{\ul{R}_i}
    \leq \sum_{i \in \N} \gamma^{i-1} R_{\max}
  = R_{\max} / (1-\gamma) \]
  This also covers $V_{n, \pi}$.
\end{proof}
As this bound will occur again and again we denote it
\[ V_{\max} \defeq R_{\max}(1-\gamma) \]

\subsubsection{Optimal policies}

Let $(\Cal{S}_n, \Cal{A}_n, P_n, R_n)_{n \in \N}$ be a decision process.

\begin{defn}[Optimal value functions] 
  \begin{align*}
    V_n^*(s) \defeq & \; \sup_{\pi \in R\Pi} V_{n,\pi}(s) &
    V^*(s) \defeq & \; \sup_{\pi \in R\Pi} V_\pi(s)
  \end{align*}
  This is called the \defemph{optimal value function} (and the $n$th
  optimal value function).
  A policy $\pi^* \in R\Pi$ for which $V_{\pi^*} = V^*$ is called an
  \defemph{optimal policy}.
  If $V_{n, \pi^*} = V^*_n$ it is called $n$-optimal.
  \label{defn:optimalValue}
\end{defn}

\begin{prop} (D)

  $\abs{V^*_k},\; \abs{V^*} \leq V_{\max}$.
\end{prop}
\begin{proof}
  By \cref{prop:Vbounded} all terms in the suprema are within this bound.
\end{proof}

\begin{rem}
It is known that the optimal value function
might not be Borel measurable (see ex. 2 p. 233 \ncite{BS07}).
Perhaps this is not suprising since we are taking a supremum over
sets of policies which have cardinality of at least the continuum.
However it is often possible to show that they are.
We will take these discussions as they occur in various settings.
\end{rem}

At this point some central questions can be asked.
\begin{enumerate}
  \item To which extend does an optimal policy $\pi^*$ exist?
  \item Does $V_n^*$ converge to $V^*$?
  \item When can optimal policies be chosen to be Markov, deterministic, etc.?
  \item Can an algorithm be designed to efficiently find $V^*$ and
    $\pi^*$?
\end{enumerate}
These questions has been answered in a variety of settings.
We will try to address them in order by strength of assumptions
they require.

\subsubsection{Schäls theorem}
In a quite general setting, questions 1 and 2
is investigated in \mcite{S75}.
Here some additional structure on our process is imposed:
\begin{sett}[Schäl]
  \begin{enumerate}
    \item $V_\pi < \infty$ for all policies $\pi \in R\Pi$.
    \item $(\Cal{S}_n, \Sigma_{\Cal{S}_n})$ is assumed to be standard Borel.
      I.e. $\Cal{S}_n$ is a non-empty Borel subset of a Polish space
      and $\Sigma_{\Cal{S}_n}$ is the Borel subsets of $S_n$.
    \item $(\Cal{A}_n, \Sigma_{\Cal{A}_n})$ is similarly assumed to be
      standard Borel.
    \item $\Cal{A}_n$ is compact.
    \item $\forall s \in \Cal{S}_1 :
      Z_n = \sup_{N \geq n} \sup_{\pi \in R\Pi} \sum_{t=n+1}^N
      \E_s^\pi r_n \to 0$ as $n \to \infty$.
  \end{enumerate}
  \label{sett:Schal}
\end{sett}

In this setting Schäl introduced two set of criteria for the existence
of an optimal policy:

\begin{cond}{S}
  \begin{enumerate}
    \item The function \[
	(a_1, a_2, \dots, a_n) \mapsto
	P_n(\cdot \mid s_1, a_1, s_2, a_2, \dots, s_n, a_n)
      \]
      is set-wise continuous (hence the name \defemph{S})
      for all $s_1, \dots, s_n \in \Cal{S}^{\ul{n}}$.
    \item $r_n$ is upper semi-continuous.
  \end{enumerate}
  \label{cond:S}
\end{cond}

\begin{cond}{W}
  \begin{enumerate}
    \item The function
      \[(h_n, a_n) \mapsto P_n(\cdot \mid h_n, a_n)\]
	is weakly continuous (hence the name \defemph{W}).
    \item $r_n$ is continuous.
  \end{enumerate}
  \label{cond:W}
\end{cond}

\begin{thm}[Schäl]
  When either (\cref{cond:S}) or (\cref{cond:W}) hold then
  \begin{enumerate}
    \item There exist an optimal policy $\pi^* \in R\Pi$.
    \item $V^*_n \to V^*$ as $n \to \infty$.
  \end{enumerate}
  \label{thm:SchalExi}
\end{thm}
\begin{proof}
  We refer to \ncite{S75}. %todo: or do we?
\end{proof}

Schäls theorem tells us that optimal policies exist in a wide class
of decision processes. However in many cases we are looking at processes
in which the next state in independent of the history.
In such cases it makes sense to ask if optimal policies can be chosen
within the system of policy subclasses.
Such questions will be addressed in the next section.

\subsection{The Markov decision process and its operators}

\begin{defn}[Markov decision process]
  A \defemph{Markov decision process} (MDP) consists of
  \begin{enumerate}
    \item $(\Cal{S}, \Sigma_{\Cal{S}})$ a 
      measurable space of states.
    \item $(\Cal{A}, \Sigma_{\Cal{A}})$ a 
      measurable space of actions.
    \item $P : \Cal{S} \times \Cal{A} \leadsto \Cal{S}$
      a transition kernel.
    \item $R : \Cal{S} \times \Cal{A} \leadsto \ol{\ul{\R}}$
      a reward kernel.
    \item An optional disount factor $\gamma \in [0,1]$
      (when not discounting put $\gamma = 1$).
  \end{enumerate}
  \label{sett:MDP}
\end{defn}
This is a special case of the
history dependent decision process (\cref{sett:HDP}) with
\begin{itemize}
  \item \Cref{asm:oneStateActionSpace} is satisfied i.e. 
    $\Cal{S}_1 = \Cal{S}_2 = \dots = \Cal{S},
    \quad \Cal{A}_1 = \Cal{A}_2 = \dots = \Cal{A}$.
  \item $P_n$ depends only on $s_n$ and $a_n$ and does not
    differ with $n$, i.e. 
    $P_n(\cdot \mid s_1, \dots, s_n, a_n) = P(\cdot \mid s_n, a_n)$
    for all $n \in \N$.
  \item $R_n$ depends only on $s_n$ and $a_n$ and does not differ
    with $n$ except for a potential discount.
    I.e. $R = R_n/\gamma^{n-1}$ for all $n \in \N$
\end{itemize}
We will write $P$ instead of $P_n$ understanding
kernel compositions as if using $P_n$.

%todo write here how value functions in MDPs look under (D)

At this point it makes sense to define

\begin{defn}[The $T$-operators]
  For a stationary policy $\pi$ and measurable $V:\Cal{S} \to \ol{\ul{\R}}$
  with $V \geq 0$, $V \leq 0$ or $\abs{V} < \infty$
  we define the operators 
  \[ P_\pi V \defeq s \mapsto \int V(s') \difd P\pi(s' \mid s) \]
  \[ T_\pi V \defeq s \mapsto \int r(s, a)
  + \gamma V(s') \difd (P \pi)(a, s'\mid s) \]
  \[ T V \defeq s \mapsto \sup_{a \in \Cal{A}} T_a V(s) \]
  where $T_a = T_{\delta_a}$.
\end{defn}

\begin{prop}[Properties of the $T$-operators]
  Let $\pi = (\pi_1, \pi_2, \dots)$ be a Markov policy.
  \begin{enumerate}
    \item The operators $P_\pi, T_\pi$ and $T$ commutes with limits.
    \item $V_{k, \pi} = T_{\pi_1} V_{k-1, (\pi_2, \dots)}
      = T_{\pi_1} \dots T_{\pi_k} V_0$.
    \item $V_\pi = \lim_{k \to \infty} T_{\pi_1} \dots T_{\pi_k} V_0$
    \item If $\pi$ is stationary $T_\pi V_\pi = V_\pi$.
    \item (D) $T$ and $T_\pi$ are $\gamma$-contractive
      on $\Cal{L}_\infty(\Cal{S})$.
    \item (D) $V_\pi$ is the unique bounded fixed point of $T_\pi$
      in $\Cal{L}_\infty(\Cal{S})$
  \end{enumerate} 
  \label{prop:propTV}
\end{prop}
\begin{proof}
  \leavevmode
  \begin{enumerate}
    \item By monotone or dominated convergence theorems.
      \label{commLimits}
    \item 
      \begin{align*}
	&T_{\pi_1} V_{k,(\pi_2, \dots)}(s_1)
	\\ &= \int r(s_1, a_1) + \gamma
	\int \sum_{i=2}^{k+1} \gamma^{i-2} r(s_i, a_i)
	\difd \kappa_{\pi_2, \dots} (a_2, s_3, a_3, \dots \mid s_2)
	\difd P \pi_1(a_1, s_2 \mid s_1)
	\\ &= \int \sum_{i=1}^{k+1} \gamma^{i-1} r(s_i, a_i)
	\difd \dots P \pi_2 P \pi_1 (a_1, s_2, \dots \mid s_1)
	\\ &= \int \sum_{i=1}^{k+1} \gamma^{i-1} r(s_i, a_i)
	\difd \kappa_\pi (a_1, s_2, \dots \mid s_1)
	\\ &= V_{k+1, \pi}(s_1)
      \end{align*}
      Now use this inductively.
    \item This is by 2. and \cref{prop:VnLimV}.
    \item By 3. $T_\pi V_\pi = T_\pi \lim_{k \to\infty} T_{\pi}^k V_0
      = \lim_{k \to\infty} T_\pi^{k+1} V_0 = V_\pi$.
    \item Let $V, V' \in \Cal{L}_\infty(\Cal{S})$
      and let $K = \norm{V - V'}_\infty$.
      Then since the rewards are bounded
      \[ \abs{T^\pi V - T^\pi V'}
	= \gamma \abs{\int V(s') - V'(s') \difd P\pi(s' \mid s)}
      \leq \gamma K \]
      For $T$ use the same argument and the fact that
      $\abs{\sup_x f(x) - \sup_{y} g(y)} \leq
      \abs{\sup_x f(x) - g(x)}$ for any $f,g : X \to \ul{\R}$.
    \item By 4., 5. and Banach fixed point theorem.
  \end{enumerate}
\end{proof}

\subsection{Q-functions}
\begin{defn}
  Let $\pi \in R\Pi$.
  Define
  \[ Q_{k, \pi}(s, a) = r(s, a) + \gamma \E_{P(\cdot \mid s, a)} V_{k, \pi}
  ,\qquad Q_\pi = r(s, a) + \gamma \E_{P(\cdot \mid s, a)} V_\pi \]
  \[ Q^*_k = \sup_{\pi \in R\Pi} Q_{k, \pi}
  , \qquad Q^* = \sup_{\pi \in R\Pi} Q_\pi \]
  Define $Q_0 = r$ then we make the convention that
  $Q^*_0 = Q_{0,\pi} = Q_0 = r$.
\end{defn}
The idea of Q-functions (and the letter Q) originates to
\mcite{W89}. Upon the definition he notes
\begin{displayquote}
  ``This is much simpler to calculate than [$V_\pi$]
  for to calculate [$Q_\pi$] it is only necessary to look one
  step ahead [\ldots]''
\end{displayquote}
A clear advantage of working with Q-function
$Q:\Cal{S}\times\Cal{A} \to \Rext$ rather than a value function
$V:\Cal{S}\to \Rext$,
is that finding the optimal action in state $s$
requires only a maximization over the Q-function itself:
$a = \argmax_{a \in \Cal{A}} Q(s,a)$.
This should be compared to finding a best action according to a value
function $V$:
$a = \argmax_{a \in \Cal{A}} r(s,a) + \gamma \E_{P(\cdot \mid s,a)} V$.
Besides being less simple,
this requires taking an expectation with respect to 
both the reward and transition kernel.
Later we will study settings where we are not allowed to know
the process kernels when attempting to find the optimal strategy.
In these situations the advantage of Q-functions is clear.
For now however the transition kernel will remain known and we
will in this section see how the results of state-value functions
translate to Q-functions.
The results in this section are original in the generality here presented,
as I was unable to find them elsewhere.

\begin{prop} (D)

  $\lim_{k \to \infty} Q_{k, \pi} = Q_\pi$. Furthermore it holds that
  $\abs{Q_{k, \pi}}, \abs{Q_\pi}, \abs{Q^*_k}, \abs{Q^*} \leq V_{\max}$.
\end{prop}
\begin{proof}
  By dominated convergence or monotone convergence and \cref{prop:Vbounded}.
\end{proof}

In parallel to the operators for state-value functions we define
\begin{defn}[$T$ operators for Q-functions]
  For any stationary policy $\pi \in S\Pi$
  and measurable $Q:\Cal{S} \times \Cal{A} \to \ol{\ul{\R}}$ with
  $Q \geq 0, Q \leq 0$ or $\abs{Q} < \infty$ we define
  \[ P_\pi Q(s, a) = \int Q(s', a') \difd \pi P(s', a' \mid s, a) \]
  \[ T_\pi Q = r + \gamma P_\pi Q \]
  \[ T Q(s, a) = r(s, a) + \gamma
  \int \sup_{a' \in \Cal{A}} Q(s', a') \difd P(\cdot \mid s, a) \]
  where $T_a = T_{\delta_a}$.
\end{defn}
 
\begin{prop}[Properties of T-operators for Q-functions]
  Let $\pi = (\pi_1, \pi_2, \dots) \in M\Pi$ be a Markov policy
  and $\tau \in S\Pi$ stationary.
  \leavevmode
  \begin{enumerate}
    \item $T_\tau Q_{k, \pi}
      = r + \gamma \E T_\tau V_{k, \pi}$
    \item $Q_{k, \pi} = T_{\pi_1} \dots T_{\pi_k} Q_0$. 
    \item $T_\tau Q_\tau = Q_\tau$.
    \item (D) $T_\tau$ is $\gamma$-contractive on
      $\Cal{L}_\infty(\Cal{S}\times\Cal{A})$
      and $Q_\tau$ is the unique fixed point of $T_\tau$ in
      $\Cal{L}_\infty(\Cal{S}\times\Cal{A})$.
  \end{enumerate}
  \label{prop:TQ}
\end{prop}
\begin{proof}
  \leavevmode
  \begin{enumerate}
    \item This is essentially due to properties of the kernels. The idea is
      sketched here
      \begin{align*}
	T_\mu Q_{k, \pi} = r + \gamma \int r + \gamma V_{k, \pi}
	\difd P \difd \mu P
	= r + \gamma \int r + \gamma V_{k, \pi} \difd P \mu \difd P
	= r + \gamma \int T_\mu V_{k, \pi} \difd P
      \end{align*}
    \item Use 1. iteratively starting with
      $\mu = \pi_1, \pi = (\pi_2, \pi_3, \dots)$.
    \item By 2. $T_\pi Q_\pi = T_\pi (r + \gamma \E \lim_{k\to\infty} T_\pi^k V_0)
      = \lim_{k\to\infty} T_\pi (r + \gamma \E T_\pi^k V_0)
      = \lim_{k\to\infty} (r + \gamma \E T_\pi^{k+1} V_0)
      = r + \gamma \E \lim_{k\to\infty} T^{k+1}_\pi V_0
      = r + \gamma \E V_\pi = Q_\pi$.
    \item The contrativeness of $T_\pi$ follows from the same argument as for
      value functions. 2. and Banach fixed point theorem does the rest.
  \end{enumerate}
\end{proof}

\begin{defn}
  Let $\pi : \Cal{S} \leadsto \Cal{A}$ be a stationary policy. Define
  $A_s = \argmax_{a \in \Cal{A}} Q(s, a)$.
  If there exist a measurable subset $B_s \subseteq A_s$
  for every $s \in \Cal{S}$ such that
  \[ \pi \left( B_s \Mid s \right) = 1 \]
  then $\pi$ is said to be \defemph{greedy} with respect to $Q$ and is
  denoted $\pi_Q$.
\end{defn}

\begin{prop}
  For any integrable $Q : \Cal{S} \times \Cal{A} \to \ol{\ul{\R}}$
  if $\pi_Q$ is greedy with respect to $Q$ then $T_{\pi_Q} Q = TQ$.
\end{prop}
\begin{proof}
  \begin{align*}
    T_{\pi_Q} Q &= r + \gamma \int Q(s, a) \difd \pi P(s, a \mid \cdot)
    \\ &= r + \gamma \int \int Q(s, a)
    \difd \pi_Q(a \mid s) \difd P(s \mid \cdot)
    \\ &= r + \gamma \int \max_{a \in \Cal{A}} Q(s, a)
    \difd P(s \mid \cdot)
    \\ &= T Q
  \end{align*}
\end{proof}

%todo greedy policy

\subsection{Bertsekas-Shreve framework}
The theory described here is largely based on the text book
\emph{Stochastic Optimal Control: Discrete-time Case} by
\mcite{BS07}.
Their framework is cost-based as opposed to the this paper reward-based outset.
This means that (P) and (N), upper and lower semicontinuity,
suprema and infima, ect. are opposite to the source.
\begin{sett}[BS]
  \leavevmode
  \begin{enumerate}
    \item We consider an MDP $(\Cal{S}, \Cal{A}, P, R, \gamma)$
      (see \cref{sett:MDP}).
    \item $\Cal{S}$ and $\Cal{A}$ are Borel spaces.
    \item $\Cal{A}$ is compact.
    \item $P(S \mid \cdot)$ is continuous for any $S \in \Sigma_{\Cal{S}}$.
    \item $r(s,a) = \gamma^{1-i} \int x \difd R(x \mid s, a)$ 
      is upper semicontinuous and uniformly bounded from above
      (least upper bound denoted $0 < R_{\max} < \infty$).
    \item The policies must consist of
      universally measurable probability kernels.
  \end{enumerate}
  \label{sett:BS}
\end{sett}
The original setup in \ncite{BS07}
is slightly different than the setup here presented.
Besides having a state and action space, it also features a 
non-empty Borel space called the
\emph{disturbance space} $W$, a \emph{disturbance kernel}
$p: \Cal{S} \times \Cal{A} \to W$,
instead of a transition kernel which on the other hand is a deterministic
\emph{system function} $f : \Cal{S} \times \Cal{A} \times W \to \Cal{S}$
which should be Borel measurable.
Moreover it allows for constrains on the action space for each state.
This is made precise by a function $U:\Cal{S} \to \Sigma_{\Cal{A}}$
and a restriction on $R\Pi$ that all policies $\pi$ should satisfy
$\pi(U(s) \mid s) = 1$.
Lastly the rewards are interpreted as negative costs, and thus
$g$ is required to be semi \emph{lower}continuous.

By setting $P(\cdot \mid s, a) = f(s, a, p(\cdot \mid s, a))$
and maximizing rewards of upper semicontinuous instead of
minimizing lower semicontinuous ones, we fully capture
all aspects of the original process and its results,
except the for the action constrains. %todo make a more precise argument

Notice that \cref{sett:BS} implies (\cref{cond:F+}).
Throughout this section 
are always assumed. 

\begin{prop}
  Let $\Cal{X}, \Cal{Y}$ be separable and metrizable,
  $\kappa : \Cal{X} \to \Cal{Y}$ be a continuous probability kernel
  and $f:\Cal{X} \times \Cal{Y} \to \ul{\ol{\R}}$ be Borel-measurable
  satisfying one of
  $f \leq 0, f \geq 0, \abs{f} < \infty$.
  If $f$ is bounded from above (below) and upper (lower) semicontinuous
  then
  \[ x \mapsto \int f \difd \kappa(\cdot \mid x) \]
  is bounded from above (below) and upper (lower) semicontinuous. 
  \label{prop:BS7_31}
\end{prop}
\begin{proof}
We refer to \ncite{BS07} prop. 7.31. %todo do it yourself
\end{proof}

%proposition 8.6 Stoch. Opt. Control
\begin{prop}[Prop. 8.6 in BS]
  $V^*_k = T^k V_0$ and is upper semicontinuous.
  Furthermore there exists a sequence of deterministic, stationary,
  Borel-measurable policies
  $\tau^*_1, \tau^*_2, \dots \in DS\Pi$
  such that $\pi^*_k = (\tau^*_k, \dots, \tau^*_1)$ is $k$-optimal.
  \label{prop:BSprop8_6}
\end{prop}

%cor. 9.17.2
\begin{thm}[Cor. 9.17.2 in BS]
  Under (\cref{cond:N}) or (\cref{cond:D})
  $V^* = \lim_{k\to\infty} V_k^*$ and is upper semicontinuous.
  Furthermore there exist a deterministic
  stationary, Borel-measurable policy $\pi^*$.
  \label{thm:BScor9.17.2}
\end{thm}

\subsubsection{Analytic setting}
For comparison, we include here an similar result in an alternative setting,
also considered by \ncite{BS07}.

\begin{sett}[BS Analytic]
  The same as \cref{sett:BS} except:
  $P$ is not necessarily continuous.
  $r$ is upper semianalytic.
  $\Cal{A}$ is not necessarily compact, but
  there exists a $k \in \N$ such that
  $\forall \lambda \in \R, n \geq k, s \in \Cal{S}$
  \[ A^\lambda_n(s) = \left\{ a \in \Cal{A} \Mid r(s, a)
  + \gamma \int V^*_n P(\cdot \mid s, a) \geq \lambda \right\} \]
  is a compact subset of $\Cal{A}$.
  \label{sett:BSA}
\end{sett}
This \cref{sett:BSA}, was actually more widely discussed in \ncite{BS07}.
We have put more emphasis on the semicontinuous setting, as it
appears restrictive to assume the semianalytical property.

\begin{thm}[Prop. 9.17 BS]
  Under \cref{sett:BSA} we have
  $V^* = \lim_{n \to \infty} V^*_n$ for all $s \in \Cal{S}$
  and there exists a optimal policy $\pi^*$ which is stationary
  and deterministic.
\end{thm}
\begin{proof}
We refer to \mcite{BS07} prop. 9.17.
\end{proof}

\subsubsection{Implications for value-functions}
Let \cref{sett:BS} hold.

\begin{prop}
  $V^* = V_{\pi^*} = T_{\pi^*} V^* = T V^*$
  
  (D) $V^*$ is the unique fixed point of $T$ in $\Cal{L}_\infty(\Cal{S})$.
  \label{prop:VoptEqVpiOpt}
\end{prop}
\begin{proof}
  Since $\pi^*$ is optimal $V^* = V_{\pi^*}$ which by \cref{prop:propTV}
  equals $T_{\pi^*} V_{\pi^*}$.
  By \cref{thm:BScor9.17.2} and \cref{prop:BSprop8_6}
  $T V^* = T \lim_{k\to\infty} T^k V_0 =
  \lim_{k\to\infty} T^{k+1} V_0 = V^*$.
  If (D) holds $V^* \in \Cal{L}_\infty(\Cal{S})$ so by \cref{prop:propTV} 5.
  and 6. we are done.
\end{proof}

\begin{prop}
  \leavevmode
  \begin{enumerate}
    \item $Q^*_k = r + \gamma \E V^*_k$ and is upper semicontinuous.
    \item (N) (D) $Q^* = r + \gamma \E V^*$ and is upper semicontinuous.
    \item (N) (D) $\sup_{a \in \Cal{A}} Q^*(s, a) = V^*(s)$.
    \item (N) (D) $Q^* = \lim_{k\to\infty} Q_k^*$.
    \item (N) (D) $Q^* = Q_{\pi^*}$.
  \end{enumerate}
\end{prop}
\begin{proof}
  \leavevmode
  \begin{enumerate}
    \item Since $V_k^*$ is measurable due to \cref{prop:BSprop8_6}
      we see that
      $Q_k^* = \sup_{\pi \in R\Pi} (r + \gamma \E V_{k,\pi})
      \leq r + \gamma \E V_k^* = r + \gamma \E V_{\pi_k^*}
      \leq Q_k^*$.
      \Cref{prop:BS7_31} gives upper semicontinuity.
    \item Since $V^*$ is measurable due to \cref{thm:BScor9.17.2}.
      Now follow the argument for 1.
    \item Let $s \in \Cal{S}$ then $\sup_{a \in \Cal{A}} Q^*(s, a) = 
      \sup_{a \in \Cal{A}} (r(s, a) + \gamma \E_{P(\cdot \mid s, a)} V^*)
      = T V^*(s) = V^*(s)$.
    \item By monotone or dominated convergence and \cref{thm:BScor9.17.2}.
    \item By \cref{prop:VoptEqVpiOpt} and 2.
      $Q^* = r + \gamma \E V^* = r + \gamma \E V_{\pi^*} = Q_{\pi^*}$.
      \end{enumerate}
\end{proof}

\begin{prop}
  \leavevmode
  \begin{enumerate}
    \item $TQ^*_k = r + \gamma \E T V^*_k$ and if
      $\pi^* = (\pi^*_1, \pi^*_2 \dots)$
      is $k$-optimal then
      $Q^*_k = T_{\pi^*_1} \dots T_{\pi^*_k} r = T^k r$.
    \item $TQ^* = r + \gamma \E T V^*$ and $TQ^* = Q^*$.
    \item (D) $T$ is $\gamma$-contractive on
      $\Cal{L}_\infty(\Cal{S}\times\Cal{A})$
      and $Q^*$ is the unique fixed point of $T$ in 
      $\Cal{L}_\infty(\Cal{S}\times\Cal{A})$.
  \end{enumerate}
  \label{prop:TQfp}
\end{prop}

\begin{proof}
  \leavevmode
  \begin{enumerate}
    \item \begin{align*}
	TQ^*_k(s,a) &= T(r + \gamma \E V^*_k)(s,a)
	\\ &= r(s,a) + \gamma
	\int \sup_{a' \in \Cal{A}} (r(s',a')
	+ \gamma \E_{P(\cdot \mid s', a')} V^*_k)
	\difd P(s' \mid s,a)
	\\ &=r(s,a) + \gamma
	\int \sup_{a' \in \Cal{A}} \left(r(s',a') + \gamma
	\int V_k^*(s'') \difd P(s'' \mid s', a') \right)
	\difd P(s' \mid s,a)
	\\ &= r(s, a) + \gamma
	\int T V^*_k(s') \difd P(s' \mid s, a)
      \end{align*}
      To get $Q^*_k = T^k r$ use this inductively
      $Q^*_k = r + \gamma \E V^*_k = r+ \gamma TV^*_{k-1}
      = T Q^*_{k-1} = \dots$.
      The statement $Q^*_k = T_{\pi^*_1} \dots T_{\pi^*_k} r$
      is from \cref{prop:TQ}.
    \item The argument from 1. also implies this first statement in
      2. Now $TQ^* = r + \gamma \E TV^* = r + \gamma \E V^* = Q^*$
      by \cref{prop:VoptEqVpiOpt}.
    \item The argument is similar to \cref{prop:propTV} pt. 5.
  \end{enumerate}
\end{proof}

\begin{cor} (D)

  For any $Q \in \Cal{L}_\infty(\Cal{S} \times \Cal{A})$
  $T^k Q$ converges to $Q^*$ with rate $\gamma^k$.
  That is
  \[ \norm{T^k Q - Q^*}_\infty \leq \gamma^k \norm{Q - Q^*}_\infty \]
  \label{cor:QrateSimple}
\end{cor}
\begin{proof}
  This is directly from \cref{prop:TQfp} pt. 3.
\end{proof}

\begin{prop}
  \leavevmode
  \begin{enumerate}
    \item Let $\pi_i$ be greedy w.r.t. $Q_{i-1}^*$ then
      $(\pi_i, \pi_{i-1}, \dots, \pi_1)$ is $i$-optimal for any $i \in \N$.
    \item (N) (D) Any greedy strategy for $Q^*$ is optimal and such exist.
  \end{enumerate}
\end{prop}
\begin{proof}
  \begin{enumerate}
    \item Such greedy policies exist because $Q_{k, \pi}$ is upper
      semicontinuous by \cref{prop:BSprop8_6}.
      For induction base observe that
      $ Q_{1, \pi_1} = T_{\pi_1} Q_0 = T Q_0 = Q_1^*$.
      Now assume $Q_{i-1, {\pi_{i-1}, \dots, \pi_1}} = Q^*_{i-1}$.
      Then
      $Q_{i, (\pi_i, \dots, \pi_1)}
      = T_{\pi_i} Q_{i-1, (\pi_{i-1}, \dots, \pi_1)}
      = T_{\pi_i} Q^*_{i-1} = T Q_{i-1}^* = Q_i^*$.
    \item Since $Q$ is upper semicontinuous in the second entry
      the set $A_s = \argmax_{a \in \Cal{A}} Q(s, a)$ is non-empty
      and measurable for all $s$.
      Pick (by axiom of choice) an $a_s \in A_s$ for every $s \in \Cal{S}$.
      Then $\pi(\cdot \mid s) = \delta_{a_s}$ is greedy with respect to $Q$.
      %todo is pi measurable?
  \end{enumerate}
\end{proof}

\begin{rem}
  Most of the results of this section hold
  also under \cref{sett:BSA} with the addition
  that 'semicontinuous' is replaced by 'semianalytic'.
\end{rem}

\subsection{Theoretical Q-iteration}

Based on the results established so far we can as a non-practical
example design the following algorithm:

\begin{figure}[H]
\begin{algorithm}[H] %\label{algocf:fq} % this labels line, could not fix
\caption{Simple theoretical Q-iteration}
\KwIn{MDP $(\Cal{S}, \Cal{A}, P, R, \gamma)$, number of iterations $K$}
$\forall (s, a) \in \Cal{S} \times \Cal{A} :
r(s, a) \leftarrow \int x \difd R(x \mid s, a)$.

$\wt{Q}_0 \leftarrow r$

\For{$k = 0,1,2,\dots,K-1$}{
  $ \forall (s, a) \in \Cal{S} \times \Cal{A} :
  \wt{Q}_{k+1}(s, a) \leftarrow r(s, a)
  + \gamma \int \sup_{a' \in \Cal{A}} \wt{Q}_k(s', a') \difd P(s' \mid s, a)$
}
Define $\pi_K$ as the greedy policy w.r.t. $\wt{Q}_K$ \\
\KwOut{An estimator $\widetilde{Q}_K$ of $Q^*$ and policy $\pi_K$}
\label{alg:theoSimpleQ}
\end{algorithm}
\end{figure}

\begin{prop}(D)

  The output $\wt{Q}_K$ of \cref{alg:theoSimpleQ} converges to the optimal
  Q-function $Q^*$ with rate $\gamma^K$ concretely
  $\norm{\wt{Q}_K - Q^*}_\infty \leq \gamma^K \norm{Q^*}_\infty$.
  \label{prop:theoSimpleQConv}
\end{prop}
\begin{proof}
  This is by \cref{cor:QrateSimple}.
\end{proof}

\subsubsection{Finite Q-iteration}
We have shown how if one knows the dynamics
of a stationary decision process satisfying rather broad criteria, 
such as continuity and compactness,
the optimal policy and state-value function can be found
simply by iteration over the $T$-operator and picking a greedy strategy
(see \cref{prop:theoSimpleQConv}).
Of course this is practical computationally, only if
the resulting $Q$ functions can be represented and computed in finite
space and time.
An obvious situation in which such a representation and computation is possible,
is the finite case.
\begin{asm}
  $\Cal{S}\times\Cal{A}$ is finite.
  \label{asm:finite}
\end{asm}
Say $\abs{\Cal{S}} = k$ and $\abs{\Cal{A}} = \ell$.
In this case the transition operator $P$ can be represented as a
matrix of \emph{transition probabilities}
\[ P \defeq \begin{pmatrix}
    P(s_1 \mid s_1, a_1) & \dots & P(s_k \mid s_1, a_1)
    \\ \vdots & \vdots & \vdots
    \\ P(s_1 \mid s_k, a_\ell) & \dots & P(s_k \mid s_k, a_\ell)
\end{pmatrix} \]
then the algorithm becomes

\begin{algorithm}[H] %\label{algocf:fq} % this labels line, could not fix
\caption{Simple finite Q-iteration}
\KwIn{MPD $(\Cal{S}, \Cal{A}, P, R, \gamma)$, number of iterations $K$}
Set $ r \leftarrow \left(\int r \difd R(r \mid s_1, a_1),
\dots, \int r \difd R(r \mid s_k, a_\ell) \right)^T $

and $ \wt{Q}_0 \leftarrow r$.

\For{$k = 0,1,2,\dots,K-1$}{
  Set $m(\wt{Q}_k) \leftarrow (\max_{a \in \Cal{A}} Q(s_1, a), \dots,
  \max_{a \in \Cal{A}}Q(s_k, a))^T$

  Update action-value function:
  \[ \wt{Q}_{k+1} \leftarrow
    r + \gamma P m(\wt{Q}_k)
  \]
}
Define $\pi_K$ as the greedy policy w.r.t. $\wt{Q}_K$ \\
\KwOut{An estimator $\widetilde{Q}_K$ of $Q^*$ and policy $\pi_K$}
\label{alg:finiteSimpleQ}
\end{algorithm}

\begin{prop}
  The output $\wt{Q}_K$ from \cref{alg:finiteSimpleQ} is
  $K$-optimal and
  $\norm{\wt{Q}_K - Q^*}_\infty \leq \gamma^K \norm{Q^*}_\infty$.
\end{prop}
\begin{proof}
  See \cref{prop:theoSimpleQConv}.
\end{proof}


%todo check measurability (universal) issues in the above sections

\subsection{Approximation}
In this section we will look at what happens if we
instead use approximations of the Q-functions and $T$ operator.
This means that we are in a setting where we can somehow
calculate $r$ and $TQ$ for any $(s,a) \in \cl{S} \times \cl{A}$,
but it is hard or infeasible to represent them (or at least one of them)
directly.
This setting is not very well-studied in the case of a
continuous state space (at least in the sources known to this writer).
This is perhaps because it is considered solved
by the results of theoretical Q-learning presented in the previous section.
However as we have argued, this only have practical relevance 
when it is feasible to represent $TQ$.
Therefore we find it relevant to consider this setting in more detail.
What \emph{is} very well-studied is a further generalized setting
where $T$ and $r$ are assumed to be unknown,
that is, one has only access to their distributions via sampling from them.
We will deal with this setting in the next section.
In following we present some rather simple bounding techniques
which is inspired by arguments found in e.g. \ncite{F20},
together with some standard results from approximation theory
on artificial neural networks and Bernstein polynomials.
Throughout this section we assume (D)
i.e. that we are discounting with some $\gamma \in [0,1)$.

Let us consider any norm $\norm{\cdot}$ on
$(\cl{F}, \norm{\cdot})$ where $\cl{F} \subseteq \cl{Q}$ is
a subset of the space of bounded
Q-functions $\cl{Q} = \cl{L}_\infty(\cl{S}\times\cl{A})$.
Let $\wt{Q}_0$ be any Q-function which is bounded in $\norm{\cdot}$.
Suppose we approximate $T\wt{Q}_0$ by a Q-function $\wt{Q}_1$
to $\ve_1 > 0$ precision and then approximate $T\wt{Q}_1$ by $\wt{Q}_2$
and so on. This way we get a sequence of Q-functions satisfying
\[ \norm{T\wt{Q}_{k-1} - \wt{Q}_k} \leq \ve_k, \forall k \in \N \]

First observe that
\begin{align*}
  \norm{T^k \wt{Q}_0 - \wt{Q}_k}
  &\leq \norm{T^k \wt{Q}_0 - T \wt{Q}_{k-1}} + \norm{T\wt{Q}_{k-1} - \wt{Q}_k}
  \\ &\leq \gamma \norm{T^{k-1} \wt{Q}_0 - \wt{Q}_{k-1}}
  + \norm{T\wt{Q}_{k-1} - \wt{Q}_k}
\end{align*}

Using this iteratively we get
\[ \norm{T^k \wt{Q}_0 - \wt{Q}_k} \leq \sum_{i=1}^k \gamma^{k-i} \ve_i
\defeq \ve_{\mathrm{approx}}(k) \]

Then we can bound
\begin{align*}
  \norm{Q^* - \wt{Q}_k}
  &\leq \norm{Q^* - T^k \wt{Q}_0} + \norm{T^k \wt{Q}_0 - \wt{Q}_k}
  \\ &\leq \gamma^k \norm{Q^* - \wt{Q}_0}
  + \ve_{\mathrm{approx}}(k)
\end{align*}

These terms are called respectively the \emph{algorithmic} error
and the \emph{approximation} error.

The algorithmic error converges exponentially, so one is often happy with this
part not spending time trying to bound this tighter.
The approximation error depends on our step-wise approximations. For example
if $\ve_i(k) = \ve$ for some $\ve > 0$ we easily get the bound
\begin{equation}
  \ve_{\mathrm{approx}}(k) = \ve \frac{1-\gamma^k}{1-\gamma} \leq \frac{\ve}{1-\gamma}
  \label{eq:approxEpsBound}
\end{equation}
If $\ve_i \leq c\gamma^i$ we get $\ve_{\mathrm{approx}}(k) \leq ck \gamma^k \to 0$ as
$k \to \infty$.
Generally if one can show that $\ve_i \to 0$ we have
\begin{prop} $ \sum_{i-1}^k \gamma^{k-i} \ve_i \to 0 $
  whenever $\ve_k \to 0$ as $k \to \infty$.
\end{prop}
\begin{proof}
  Let $\ve > 0$. Find $N$ such that $\ve_n \leq \ve (1-\gamma)/2$ 
  for all $n>N$ and find $M>N$ such that
  $\gamma^M \leq
  \ve \gamma^N \left( \sum_{i=1}^N \gamma^{N-i} \ve_i \right)^{-1}$.
  Then for all $m>M$
  \begin{align*}
    \sum_{i=1}^m \gamma^{m-i} \ve_i
    &\leq \gamma^{m-N} \sum_{i=1}^N \gamma^{N-i} \ve_i
    + \sum_{i=N+1}^m \gamma^{m-i} \ve (1-\gamma)/2
    \leq \ve/2 + \ve/2 \leq \ve
  \end{align*}
\end{proof}

\subsubsection{Using artifical neural networks}

\begin{sett}
  An MDP $(\cl{S}, \cl{A}, P, R, \gamma)$ with
  $\cl{S} = [0,1]^w$ and $\cl{A}$ finite.
  Assume that $r$ is continuous and
  $P$ is setwise-continuous.
  \label{sett:annApprox}
\end{sett}

\begin{defn}\label{def_ANN}
  An \textbf{ANN} (Artificial Neural Network) with structure
  $(d_i)_{i=0}^{L+1} \subseteq \N$,
  activation functions $\sigma_i = (\sigma_{ij})_{j=1}^{d_i}$, where
  $\sigma_{ij} : \R \to \R$ are real-valued functions on $\R$,
  and weights $W_i \in M^{d_i \times d_{i-1}}, \; v_i \in \R^{d_i}, \;
  i \in [L+1]$
  is the function $F:\R^{d_0} \to \R^{d_{L+1}}$ 
  \[ F = w_{L+1} \circ \sigma_L \circ w_L
  \circ \sigma_{L-1} \circ \dots \circ w_1 \]
  where $w_i$ is the affine function $x \mapsto W_i x + v_i$ for all $i$.
\end{defn}

To clarify we have $\sigma_i(x_1, \dots, x_{d_i})
= (\sigma_{i1}(x_1), \dots, \sigma_{id_{i}}(x_{d_{i}}))$.
$L \in \N_0$ is interpreted as the number of \emph{hidden layers} and
$d_i$ is the number of neurons or nodes in layer $i$.

We denote the class of these networks (or functions)
\[ \cl{DN} \left(\sigma_{ij}, ( d_i )_{i=0}^{L+1} \right) \]

An ANN is called \emph{deep} if there are two or more hidden layers.

%Todo note that ANNs imbed nicely in each other

\begin{thm}[Universal Approximation Theorem for ANNs]
  Let $\sigma: \R \to \R$ be non-constant, bounded and continuous
  activation function.
  Let $\ve > 0$ and $f \in C([0,1]^w)$.
  Then there exists an $N \in \N$ and a network
  $F \in \cl{DN}(\sigma, (w, N, 1))$
  with one hidden layer
  and activation function $\sigma$ such that
  \[ \norm{F - f}_\infty < \ve \]
  In other words $\bigcup_{N \in \N} \cl{DN}(\sigma, (w, N, 1))$ is
  dense in $C([0,1]^w)$.
  \label{thm:uniApprox}
\end{thm}
\begin{proof}[Discussion of proofs]
  The original proof in \mcite{C89} is very short and elegant,
  but non-constructive,
  using the Riesz Representation and Hahn-Banach theorems to
  obtain a contractiction to the statement that
  $\bigcup_{N \in \N} \cl{DN}(\sigma, (w,N,1))$
  is dense in $C([0,1]^w)$.
  Furthermore it considered only \emph{sigmoidal} activations
  functions, meaning that $\sigma$ should satisfy
  \[ \sigma(x) \to \begin{cases} 0 & x \to -\infty
  \\ 1 & x \to \infty \end{cases} \]
  
  This was extended in \mcite{CCR90} to the statement as presented above
  and their proof is constructive. 
\end{proof}

\begin{prop}
  Consider \cref{sett:annApprox} let
  and $\sigma : \R \to \R$ be a non-constant, bounded, continuous
  activation function. Let $\varepsilon > 0$.
  Then for every $k \in \N$ there exists a $N\in \N$ and a sequence of
  Q-networks $(\wt{Q}_i)_{i=1}^k \subseteq \cl{DN}(\sigma,
  \{w \abs{\cl{A}}, N, 1\})$ such that
  \[ \norm{T\wt{Q}_{i-1} - \wt{Q}_i}_\infty < \ve \]
  for all $i \in [k]$.
  In particular
  \[ \norm{Q^* - \wt{Q}_k}_\infty < \varepsilon/(1 - \gamma) \]
\end{prop}

This gives us the first method of how to approximate
$Q^*$ arbitrarily closely on continuous state spaces, in the case
where it is infeasible to represent $TQ$ directly.

\subsubsection{Using Bernstein polynomials}

We here discuss another approach using multivariate Bernstein polynomials
for approximation instead of neural networks.
In this case the need a slightly stronger form of continuity, namely
Lipschitz continuity, to establish the bounds.

\begin{sett}
  An MDP $(\cl{S}, \cl{A}, P, R, \gamma)$ with
  $\cl{S} = [0,1]^w$ and $\cl{A}$ finite.
  Assume that there exists a probability measure $\mu \in \cl{S}$, such that
  $P(\cdot \mid s, a)$ has density
  $p(\cdot \mid s, a) : \cl{S} \to \R$ with respect to
  $\mu$ for all $(s, a) \in \cl{S}\times\cl{A}$. 
  Furthermore assume that $r(\cdot, a), \; p(s \mid \cdot, a)$ are
  Lipschitz
  with constants $L_r,\; L_p$ respectively for all
  $(s, a) \in \cl{S} \times \cl{A}$.
  \label{sett:polyApprox}
\end{sett}

\begin{defn}[Bernstein polynomial]
  The multivariate Bernstein polynomial $B_{f, n}$ with exponents
  $n=(n_1, \dots, n_w) \in \N^w$ approximating the function $f:[0,1]^w \to \R$
  is defined by
  \begin{equation*}
    B_{f, n}(x_1, \dots, x_w) =
    \sum_{j = 1}^w \sum_{k_j = 0}^{n_j}
    f\left( \frac{k_1}{n_1}, \dots, \frac{k_w}{n_w} \right)
    \prod_{\ell = 1}^w \left(
    \binom{n_\ell}{k_\ell} x_\ell^{k_\ell}(1-x_\ell)^{n_\ell - k_\ell} \right)
  \end{equation*}
  \label{defn:Bfn}
\end{defn}
Notice that this a polynomial of (multivariate) degree $n_1 + \dots + n_w$.

\begin{thm}
  Let $f : [0,1]^w \to \R$ be Lipschitz (see \cref{defn:Lipschitz})
  w.r.t. the standard euclidean 2-norm induced metrics on $[0,1]^w$ and $\R$
  with constant $L$. 
  Then for any $n = (n_1, \dots, n_w) \in \N^w$ there exists a polynomial
  $B_{f,n} : [0,1]^w \to \R$ of degree $\leq \norm{n}_1$ such that
  \begin{enumerate}
    \item $\norm{f - B_{f,n}}_2
      \leq \frac{L}{2} \sqrt{\sum_{j=1}^w \frac{1}{n_j}}$
    \item $\norm{B_{f,n}}_\infty \leq \norm{f}_\infty$
  \end{enumerate}
\end{thm}

\begin{lem}
  $TQ(\cdot, a)$ is Lipschitz in $\norm{\cdot}_2$ with constant
  $ L_T = (L_r + \gamma V_{\max} L_p) $
  for all $a \in \cl{A}$ and $Q : \cl{S} \times \cl{A} \to [-V_{\max},V_{\max}]$.
\end{lem}

Now we can bound

\begin{prop}
  \[ \ve_{\mathrm{approx}} \leq \frac{L_r + \gamma V_{\max} L_p}{2(1-\gamma)}
  \sqrt{\sum_{j=1}^w \frac{1}{n_j}} \]
\end{prop}

For example if we put $n_j = m$ for all $j$ we get

\begin{prop}
  \[ \norm{Q^* - \wt{Q}_k} \leq \norm{Q^* - \wt{Q}_0}
    + \frac{L_r + \gamma V_{\max} L_p}{2(1-\gamma)} \sqrt{w}
  m^{-1/2} \]
  In particular $\norm{Q^* - \wt{Q}_k}_\infty
  = \cl{O}(\gamma^{-k} + \frac{1}{\sqrt{m}})$
  when using $k$ iterations and approximating
  with multivariate polynomials of maximum degree $w \cdot m$.
\end{prop}

This gives a very concrete way of constructing an arbitrarily good
approximation to $Q^*$ using polynomials.


