
\documentclass{article}

% preamble speciale Jacob Harder
% 31. jan. 2020

%packages

\usepackage[utf8]{inputenc} %utf8 is probably good
\usepackage{amsmath}
\usepackage{amssymb}
\usepackage{amsthm}
\usepackage{graphicx} %for including images
\usepackage{float} %for exact placement of figure (and more?)
\usepackage{mathtools} %for \mathclap (stacking under sums)
\usepackage{dsfont} %for boldface numbers
\usepackage{bm} %vectors in bold
\usepackage[ruled,vlined]{algorithm2e}
\usepackage{cleveref}
\usepackage{ifthen}
\usepackage{commath}
\usepackage[a4paper,width=150mm,top=25mm,bottom=25mm]{geometry}
\usepackage{upgreek}
%\usepackage[inline]{enumitem}
\usepackage{multicol}
%\usepackage{fancyhdr} %maybe later..
%\pagestyle{fancy}
\usepackage{mathabx}
\usepackage{csquotes}
\usepackage{tikz}
\usepackage{outline} %for subitems in lists

%front page
\usepackage{wallpaper}
\usepackage{titling}

%bibliography
\usepackage[numbers]{natbib}
\bibliographystyle{plainnat}
\newcommand{\mcite}[1]{[\citenum{#1}, \citeauthor{#1} (\citeyear{#1})]}
\newcommand{\ncite}[1]{\cite{#1}}

%linespread and geometry
\linespread{1.3}

%commands
\newcommand{\Cal}{\mathcal}
\newcommand{\cl}{\mathcal}
\newcommand{\fk}{\mathfrak}
\newcommand{\bb}{\mathbb}
\newcommand{\Q}{\bb{Q}}
\newcommand{\Z}{\bb{Z}}
\newcommand{\N}{\bb{N}}
\newcommand{\R}{\bb{R}}
\newcommand{\C}{\bb{C}}
\newcommand{\E}{\bb{E}}
\newcommand{\Rext}{\ol{\ul{\R}}}
\newcommand{\Var}{\mathrm{Var}}
\newcommand{\Prob}{\mathds{P}} %fundamental probability measure
\newcommand{\idc}{\mathds{1}}
\newcommand{\ve}{\varepsilon} %abbreviation for epsilon
\newcommand{\Yp}{\Upupsilon} %abbreviation for Ypsilon
\newcommand{\difd}{\; \mathrm{d}} %differential d
\newcommand{\wt}{\widetilde}
\newcommand{\wh}{\widehat}
\newcommand{\ol}{\overline}
\newcommand{\ul}{\underline}
\newcommand{\Mid}{\;\middle\vert\;}
\newcommand{\id}{\text{id}}
\newcommand{\supp}{\text{supp}}
\newcommand{\defemph}[1]{\textbf{#1}} %first-mentions of names
\DeclarePairedDelimiter\ceil{\lceil}{\rceil}
\DeclarePairedDelimiter\floor{\lfloor}{\rfloor}
\DeclareMathOperator*{\argmax}{argmax}
\DeclareMathOperator*{\argmin}{argmin}
\newcommand{\defeq}{\vcentcolon=} %definition equality symbol
%add single eq. tag in align*
\newcommand\numberthis{\addtocounter{equation}{1}\tag{\theequation}}
\newcommand{\rleft}[1]{\rotatebox[origin=c]{90}{\ensuremath{#1}}}
\newcommand{\vrel}[3]{ % for vertical subseteq e.g.
\vcenter{\halign{\hfill##\hfill\cr
\ensuremath{#1}\cr
\rotatebox[origin=c]{270}{\ensuremath{#2}}\cr
\ensuremath{#3}\cr
}}}
\newcommand{\lar}{\leftrightarrow}
\newcommand{\Span}{\mathrm{span}}
\newcommand{\Gr}{\mathrm{Gr}}

%theorems
\theoremstyle{definition}
\newtheorem{thm}{Theorem}[chapter]
\newtheorem{lem}[thm]{Lemma}
\newtheorem{defn}[thm]{Definition}
\newtheorem{cor}[thm]{Corollary}
\newtheorem{rem}[thm]{Remark}
\newtheorem{prop}[thm]{Proposition}
\newtheorem{asm}{Assumption}
\newtheorem{example}[thm]{Example}
%\newtheorem{cond}{Condition}
\newtheorem{sett}{Setting}
\newtheorem{innercond}{Condition}
\newenvironment{cond}[1]
  {\renewcommand\theinnercond{#1}\innercond}
  {\endinnercond}

%cref
\crefname{algocf}{alg.}{algs.}
\Crefname{algocf}{Algorithm}{Algorithms}
\crefname{innercond}{}{}
\Crefname{innercond}{}{}



\begin{document}


\subsection{Measure Theory}

We work with a background probability space $(\Omega, \Sigma_\Omega, \Prob)$.
For a measurable space $(\Cal{X}, \Sigma_{\Cal{X}})$ we denote
the set of probability measures on this space $\Cal{P}(\Sigma_\Cal{X})$ or
simply $\Cal{P}(\Cal{X})$ when the $\sigma$-algebra is unambiguous.
When taking cartesian products $\Cal{X} \times \Cal{Y}$ of measurable spaces
$(\Cal{X}, \Sigma_\Cal{X}), (\Cal{Y}, \Sigma_\Cal{Y})$ we always endow such
with the product $\sigma$-algebra $\Sigma_\Cal{X} \otimes \Sigma_\Cal{Y}$,
unless otherwise specified.
A map $f: \Cal{X} \to \Cal{Y}$ is called $\Sigma_{\Cal{X}}$-$\Sigma_{\Cal{Y}}$
measurable provided $f^{-1}(\Sigma_{\Cal{Y}}) \subseteq \Sigma_{\Cal{X}}$
and we denote the set of such functions $\Cal{M}(\Sigma_{\Cal{X}},
\Sigma_{\Cal{Y}})$.
By a random variable $X$ on $(\Cal{X}, \Sigma_{\Cal{X}})$ mean a
$\Sigma_\Omega$-$\Sigma_{\Cal{X}}$ measurable map.

\subsubsection{Kernels}

\begin{defn}[Probability kernel]
  Let $(\Cal{X}, \Sigma_\Cal{X}), (Y, \Sigma_\Cal{Y})$ be measurable spaces.
  A function
  \[ \kappa(\cdot \mid \cdot) : \Sigma_\Cal{Y} \times \Cal{X} \to [0,1] \]
  is a $(\Cal{X}, \Sigma_\Cal{X})$-\defemph{probability kernel}
  on $(\Cal{Y}, \Sigma_\Cal{Y})$ provided
  \begin{enumerate}
    \item $B \mapsto \kappa(B \mid x) \in \Cal{P}(\Sigma_\Cal{Y})$
      that is $\kappa(\cdot \mid x)$ is a probability measure
      for any $x \in \Cal{X}$.
    \item
      $x \mapsto \kappa(B \mid x) \in \Cal{M}(\Sigma_\Cal{X}, \Sigma_\Cal{Y})$
      that is $\kappa(B \mid \cdot)$ is ($\Sigma_\Cal{X}$-$\Sigma_\Cal{Y}$)
      measurable for any $B \in \Sigma_\Cal{Y}$.
  \end{enumerate}
  When the $\sigma$-algebras are unambiguous we shall simply say an
  $\Cal{X} \leadsto \Cal{Y}$ kernel.
  For any $x \in \Cal{X}$ and $f \in \Cal{L}_1(\kappa(\cdot \mid x))$
  we write the integral of $f$ over $\kappa(\cdot \mid x)$ as
  $\int f(y) \difd \kappa(y \mid x)$.
  \label{defn:probKer}
\end{defn}

We now state some fundamental results on probability kernels
\begin{thm}[Integration of a kernel]
  Let $\mu \in \Cal{P}(\Cal{X})$ and $\kappa : \Cal{X} \leadsto \Cal{Y}$.
  Then there exists a uniquely determined probability measure
  $\lambda \in \Cal{P}(\Sigma_\Cal{X} \otimes \Sigma_\Cal{Y})$
  such that
  \[ \lambda(A \times B) = \int_A \kappa(B, x) \difd \mu(x) \]
  \label{thm:intKer}
  We denote this measure $\lambda = \kappa * \mu$.
\end{thm}
\begin{proof}
  We refer to [ref to EH markov, thm. 1.2.1].
\end{proof}

Notice that by \cref{thm:intKer}
besides getting a probability measure on $\Cal{X} \times \Cal{Y}$
we get an induced probability measure
on $\Cal{Y}$ defined by $B \mapsto (\kappa * \mu)(\Cal{X} \times B)$.
We will denote this measure by $\kappa(\cdot \mid \mu)$.
This way $\kappa(\cdot \mid \cdot)$ can also be seen as a mapping from
$\Cal{P}(\Cal{X}) \to \Cal{P}(\Cal{Y})$ (in the second entry).

For an idea how to actually compute integrals over kernel derived measures
we here include
\begin{thm}[Extended Tonelli and Fubini]
  Let $\mu \in \Cal{P}(\Cal{X})$,
  $f \in \Cal{M}(\Sigma_\Cal{X} \otimes \Sigma_\Cal{Y}, \bb{B})$
  be a measurable function and
  $\kappa : \Cal{X} \leadsto \Cal{Y}$ be a probability kernel.
  Then
  \[ \int \abs{f} \difd \kappa(\cdot \mid \mu)
  = \int \int \abs{f} \difd \kappa(\cdot \mid x) \difd \mu(x) \]
  Furthermore if this is finite, i.e. $f \in \Cal{L}_1(\kappa(\cdot, \mu))$
  then $A_0 \defeq \left\{ x \in \Cal{X} \Mid
    \int f \difd \kappa(\cdot \mid x) < \infty \right\}
  \in \Sigma_\Cal{X}$
  with $\mu(A_0) = 1$, 
  \[ x \mapsto \begin{cases}
      \int f \difd \kappa(\cdot \mid x) & x \in A_0
      \\ 0 & x \not\in A_0
  \end{cases} \]
  is $\Sigma_\Cal{X}$-$\bb{B}$ measurable and
  \[ \int f \difd \kappa(\cdot \mid \mu)
  = \int_{A_0} \int f \difd \kappa(\cdot \mid x) \difd \mu(x) \]  
  \label{thm:extTonFub}
\end{thm}
\begin{proof}
  We refer to [ref to EH markov, thm. 1.3.2 + 1.3.3]
\end{proof}

\begin{prop}[Composition of kernels]
  Let $\kappa : \Cal{X} \leadsto \Cal{Y}, \psi : \Cal{Y} \leadsto \Cal{Z}$
  be probability kernels. Then
  \[ (\psi \circ \kappa)(A \mid x) \defeq
    \int \psi(A \mid y) \difd \kappa(y \mid x)
  ,\qquad \forall A \in \Sigma_{\Cal{Z}}, x \in \Cal{X} \]
  is a $\Cal{X} \leadsto \Cal{Z}$ probability kernel called the
  composition of $\kappa$ and $\psi$. The composition operator
  $\circ$ is associative, i.e. if $\phi : \Cal{Z} \leadsto \Cal{W}$ is
  a third probability kernel then $(\phi \circ \psi) \circ \kappa = 
  \phi \circ (\psi \circ \kappa)$.
  The associativity also extends to measures, i.e.
  $\forall \mu \in \Cal{X}
  : (\psi \circ \kappa)(\cdot \mid \mu) = \psi(\cdot \mid \kappa(\cdot \mid \mu)) $
  and this is uniquely determined by $\psi, \kappa$ and $\mu$.
  \label{prop:compKer}
\end{prop}
\begin{proof}
  The first assertion is a trivial verification of the two conditions
  in \cref{defn:probKer} and left as an exercise.
  For the associativity we refer to [todo ref to EH markov, lem. 4.5.4].
\end{proof}

\Cref{prop:compKer} actually makes the class of measurable spaces
into a category [todo ref: see Lawvere, The Category of Probabilistic
Mappings], with identity $\id_{\Cal{X}}(\cdot \mid x) = \delta_x$.
Notice that the mapping $(A, x) \mapsto \delta_x(A) \kappa(A \mid x)$
defines a probability kernel $\Cal{X} \leadsto \Cal{X} \times \Cal{Y}$
which we could denote $\id_{\Cal{X}} \times \kappa$.
Now if $\psi : \Cal{X} \times \Cal{Y} \leadsto \Cal{Z}$ is a kernel
then by \cref{prop:compKer} the composition
$(\id_{\Cal{X} \times \Cal{Y}} \times \psi)
\circ (\id_{\Cal{X}} \times \kappa)$
is a kernel $\Cal{X} \to \Cal{X} \times \Cal{Y} \times \Cal{Z}$
which we will denote $\psi * \kappa$.
It inherits associativity from $\circ$ and again this associativity
extends to application on measures: if $\mu$ is a measure on $\Cal{X}$
then $\psi * (\kappa * \mu) = (\psi * \kappa) * \mu$.

\subsubsection{Kernel derived processes}

Let $(\Cal{X}_n, \Sigma_{\Cal{X}_n})_{n \in \N}$ be a sequence
of measurable spaces. For each $n \in \N$ define
$\Cal{X}^{\ul{n}} \defeq \Cal{X}_1 \times \dots \times \Cal{X}_n$,
$\Sigma_{\Cal{X}^{\ul{n}}} \defeq \Sigma_{\Cal{X}_1} \otimes
\dots \otimes \Sigma_{\Cal{X}_n}$
and let
$\kappa_n : \Cal{X}^{\ul{n}} \leadsto \Cal{X}_{n+1}$ be a probability kernel.

\begin{prop}[Existence and uniqueness of finite kernel processes]
  For any probability measure $\rho_1 \in \Cal{P}(\Cal{X}_1)$
  and every $n \in \N$ there exists a unique probability measure 
  $\rho_n$ on $\Cal{X}^{\ul{n}}$ defined by
  \[ \rho_n \defeq \kappa_{n-1} * \dots * \kappa_1 * \rho_1 \]
  (with the convention that an empty $*$-product is $\id_{\Cal{X}_1}$
  by context)
  \label{prop:finSeqKer}
\end{prop}
\begin{proof}
  This follows simply by induction using \cref{prop:compKer}.
\end{proof}

Let $\Cal{X}^{\ul{\infty}} \defeq \prod_{n \in \N} \Cal{X}_n$
and $\Sigma_{\Cal{X}^{\ul{\infty}}} \defeq \bigotimes_{n \in \N}
\Sigma_{\Cal{X}_n}$.
\Cref{prop:finSeqKer} is not enough to establish existence of a
kernel generated measure on
$(\Cal{X}^{\ul{\infty}}, \Sigma_{\Cal{X}^{\ul{\infty}}})$ 
which we will need later.
This problem was solved by Cassius Ionescu-Tulcea in 1949:

\begin{thm}[Ionescu-Tulcea extension theorem]
  For every $\mu \in \Cal{P}(\Cal{X}_1)$ 
  there exists a unique probability measure
  $\rho \in \Cal{P}(\Cal{X}^{\ul{\infty}})$ such that
  \[ \rho_n(A) = \rho \left( A \times \prod_{k=n+1}^\infty \Cal{X}_k \right)
  , \qquad \forall A \in \Sigma_{\Cal{X}^{\ul{n}}} \]
  for all $n \in \N$.
  We denote this measure
  $\rho_\mu^{(\kappa_1, \kappa_2, \dots)}$.
  \label{thm:ionescuTulcea}
\end{thm}
\begin{proof}
  Todo: what about this.
\end{proof}

We here include lemma about the behavior of the Ionescu-Tulcea measure
for use later.
\begin{lem}
  The Ionescu-Tulcea measure satisfies
  $\rho_\mu^{(\kappa_1, \dots)}
  = \rho_{\kappa_1 * \mu}^{(\kappa_2, \dots)}$.
  \label{lem:ionescu}
\end{lem}
\begin{proof}
  Notice that by associativity
  $\kappa_n * \dots * \kappa_1 * \mu
  = (\kappa_n * \dots * \kappa_2) * (\kappa_1 * \mu)$.
  This implies that
  \[ \rho_{\mu}^{(\kappa_1, \dots)}
    \left( A \times \prod_{k=n+1}^\infty \Cal{X}_k \right)
    = \rho_{\kappa_1 * \mu}^{(\kappa_2, \dots)}
  \left( A \times \prod_{k=n+1}^\infty \Cal{X}_k \right) \]
  for all $n \in \N$ and $A \in \Sigma_{\Cal{X}^{\ul{n}}}$.
  By the uniqueness in \cref{thm:ionescuTulcea} we are done.
\end{proof}

\subsection{Dynamic programming}

\begin{defn}[Dynamic programming model]
  A general dynamic programming model is determined by
  \begin{enumerate}
    \item $(\Cal{S}_n, \Sigma_{\Cal{S}_n})_{n \in \N}$ a 
      measurable space of \defemph{states} for each timestep.
    \item $(\Cal{A}_n, \Sigma_{\Cal{A}_n})_{n \in \N}$ a 
      measurable space of \defemph{actions} for each timestep.
  \end{enumerate}
  for each $n \in \N$ write $\Cal{H}_n = \Cal{S}_1 \times \Cal{A}_1
  \times \dots \times \Cal{S}_n$, $\Cal{H}_\infty = \Cal{S}_1
  \times \Cal{A}_1 \times \dots$,
  with associated $\sigma$-algebras $ \Sigma_{\Cal{H}_n} \defeq \left(
    \bigotimes_{n=1}^{n-1} (\Sigma_{\Cal{S}_n} \otimes \Sigma_{\Cal{A}_n})
  \right) \otimes \Sigma_{\Cal{S}_n}$ and
  $\Sigma_{\Cal{H}_\infty} \defeq 
  \bigotimes_{n=1}^{\infty} (\Sigma_{\Cal{S}_n} \otimes \Sigma_{\Cal{A}_n})$.

  \begin{enumerate} \setcounter{enumi}{2}
    \item $(P_n)_{n \in \N}$ a sequence of
      $\Cal{H}_n \times \Cal{A}_n \leadsto \Cal{S}_{n+1}$ kernels
      called the \defemph{transition} kernels.
    \item $(R_n)_{n \in \N}$ a sequence of
      $\Cal{H}_{n+1} \leadsto \R$ kernels
      called the \defemph{reward} kernels.
  \end{enumerate}
\end{defn}

For such a model we can define
\begin{defn}[Policy]
  A (randomized) \defemph{policy} $\pi = (\pi_n)_{n \in \N}$
  is a sequence of $\Cal{H}_n \leadsto \Cal{A}_n$ kernels.
  The set of all policies we denote $R\Pi$.
\end{defn}

\begin{prop}[Existence and uniqueness of policy generated processes]
  Let $(\pi_n)_{n \in \N}$ be a policy and 
  $\mu \in \Cal{P}(\Cal{S}_1)$ be a probability measure.
  Then for every $n \in \N$ there exists a unique probability measure
  $\rho_n \in \Cal{P}(\Cal{H}_n)$
  such that $\rho_1 = \mu$ and
  $\rho_n = (P_n * \pi_n)(\cdot \mid \rho_{n-1})$.
  Futhermore there exists a unique probability measure
  $\rho \in \Cal{P}(\Cal{H}_\infty)$ satisfying
  \[ \rho_n(H_n) = \rho \left( H_n \times
  \prod_{k=n+1}^\infty (\Cal{A}_k \times \Cal{S}_k) \right) \]
  We will call this the \defemph{process} measure for $\pi$ and $\mu$
  and denote it $\rho_{\mu}^\pi$ with a slight abuse of notation
  ($\rho_\mu^{(P_n * \pi_n)_{n \in \N}}$ would be less abusive).
  Expectations it we denote $\E_\mu^\pi$.
  In particular when $\mu = \delta_s$, i.e. the one-point measure of
  $s \in \Cal{S}_1$ we write $\rho_s^\pi$ and $\E_s^\pi$ for short.
  \label{prop:polProc}
\end{prop}
\begin{proof}
  This is directly from \cref{prop:finSeqKer} and 
  \cref{thm:ionescuTulcea} with
  $\kappa_1 = P_1 * \pi_1, \kappa_2 = P_2 * \pi_2 \dots$.
\end{proof}


\subsubsection{Optimal policies}

Let $(\Cal{S}_n, \Cal{A}_n, P_n, R_n)_{n \in \N}$ be a dynamic programming model.
%From here on we assume that the reward kernel $R_n(\cdot \mid \cdot)$ is
%bounded from above across $n \in \N$ with upper bound
%$R_{\max}$.
%For all $n \in \N$ let $c_n : \Cal{H}_\infty \to (\Cal{S}_n, \Cal{A}_n)$
%be the projection of the history on the $n$th state-action pair.
Define the $n$th \defemph{expected reward function}
$r_n : \Cal{H}_{n+1} \to \R$ by
$r_n(h) = \int r \difd R_n(r \mid h)$ for any $h \in
\Cal{H}_{n+1}$.

%We now introduce a so called \defemph{discount factor}
%$\gamma \in (0,1)$.

In litterature the terminology varies and generally %todo: backup this claim
any function mapping a state space $\Cal{S}$ to $\R$
can be called a (state)
\defemph{value} function. Similarly any function mapping some the
product of a state space and an
action space $\Cal{A}$ to $\R$ can be called (state)
\defemph{action value} or \defemph{Q}- function.
The idea behind such functions are (usually) to estimate the
cumulative rewards associated with a state or state-action pair
and the trajectory of states it can lead to.

\begin{asm}[General assumption]
  We assume that $\sum_{i \in \N} \E_\pi r^+_i < \infty$
  for all policies $\pi \in R\Pi$.
  \label{asm:generalE}
\end{asm}

Then following definition makes sense
\begin{defn}[Ideal and optimal value functions] 
  Let $\pi$ be a policy.
  We define
  \begin{align*}
    V_n^\pi(s) \defeq & \; \E_s^\pi \sum_{i \in [n]} r_i &
    V^\pi(s) \defeq & \; \E_s^\pi \sum_{i \in \N} r_i
  \end{align*}
  called the \defemph{ideal} value functions for the policy $\pi$
  and
  \begin{align*}
    V_n^*(s) \defeq & \; \sup_{\pi \in R\Pi} V_n^\pi(s) &
    V^*(s) \defeq & \; \sup_{\pi \in R\Pi} V^\pi(s)
  \end{align*}
  called the \defemph{optimal} value functions.
  A policy for which $V^{\pi^*} = V^*$ is called an
  \defemph{optimal} policy.
\end{defn}

%The discount factor can be interpreted as measuring how much
%we consider future rewards, as our ideal value value is the
%(infinite) sum of discounted rewards of the future states.

At this point many interesting questions can be asked.
\begin{enumerate}
  \item Does an optimal policy $\pi^*$ exist?
  \item Does $V_n^*$ converge to $V^*$?
  \item Can $\pi^*$ be chosen to be deterministic?
\end{enumerate}

These questions has been answered in a variety of settings.
In a quite general setting, questions 1 and 2
was investigated by M. Schäl in 1974
[todo ref. to On Dynamic Programming:
Compactness of the space of policies, 1974].
Here some additional structure on our model is imposed:
\begin{sett}[Schäl]
  \begin{enumerate}
    \item $(\Cal{S}_n, \Sigma_{\Cal{S}_n})$ is assumed to be standard Borel.
      I.e. $\Cal{S}_n$ is a non-empty Borel subset of a Polish space
      and $\Sigma_{\Cal{S}_n}$ is the Borel subsets of $S_n$.
    \item $(\Cal{A}_n, \Sigma_{\Cal{A}_n})$ is similarly assumed to be
      standard Borel.
    \item $\Cal{A}_n$ is compact.
    \item $\forall s \in \Cal{S}_1 :
      Z_n = \sup_{N \geq n} \sup_{\pi \in R\Pi} \sum_{t=n+1}^N
      \E_s^\pi r_n \to 0$ as $n \to \infty$.
  \end{enumerate}
  \label{sett:Schal}
\end{sett}
In this setting Schäl introduced two set of criteria for the existence
of an optimal policy:

\begin{cond}{S}
  \begin{enumerate}
    \item The function \[
	(a_1, a_2, \dots, a_n) \mapsto
	P_n(\cdot \mid s_1, a_1, s_2, a_2, \dots, s_n, a_n)
      \]
      is set-wise continuous (hence the name \defemph{S})
      for all $s_1, \dots, s_n \in \Cal{S}^{\ul{n}}$.
    \item $r_n$ is upper semi-continuous.
  \end{enumerate}
  \label{cond:S}
\end{cond}

\begin{cond}{W}
  \begin{enumerate}
    \item The function
      \[(h_n, a_n) \mapsto P_n(\cdot \mid h_n, a_n)\]
	is weakly continuous (hence the name \defemph{W}).
    \item $r_n$ is continuous.
  \end{enumerate}
  \label{cond:W}
\end{cond}

\begin{thm}[Existence and convergence of optimal policies in DP]
  When either \cref{cond:S} or \cref{cond:W} hold then
  \begin{enumerate}
    \item There exist an optimal oplicy $\pi^* \in R\Pi$.
    \item $V^*_n \to V^*$ as $n \to \infty$.
  \end{enumerate}
\end{thm}
\begin{proof}
  We refer to [todo ref: On Dynamic Programming: Compactness of the space of
  policies, M. Schäl 1974]. %todo: or do we?
\end{proof}

\subsection{Stationary policies}

We will now specialize the dynamic programming model to:

\begin{sett}[Finite action decision model]
  \begin{enumerate}
    \item $\Cal{S}_1 = \Cal{S}_2 = \dots \defeq \Cal{S}$ and $\Cal{S}$ is
      standard Borel.
    \item $\Cal{A}_1 = \Cal{A}_2 = \dots \defeq \Cal{A}$ and $\Cal{A}$ is
      finite.
    \item There exist a kernel $P : \Cal{S} \times \Cal{A} \leadsto
      \Cal{S}$ such that
      \[ P_n(\cdot \mid s_1, a_1, \dots, s_n, a_n) = P(\cdot \mid s_n, a_n),
      \qquad \forall n \in \N \]
    \item There exist a kernel $R : \Cal{S} \times \Cal{A} \leadsto
      (-\infty, R_{\max}]$
      such that
      \[ R_n(\cdot \mid s_1, a_1, \dots, s_n, a_n, s_{n+1})
	= \gamma^{n-1} R(\cdot \mid s_n, a_n),
      \qquad \forall n \in \N \]
      where $\gamma \in [0,1)$ is called the \defemph{discount factor}.
      We donte $r(s, a) \defeq \int r' \difd R(r' \mid s, a)$.
    \item $r$ is semi upper continuous.
  \end{enumerate}
  \label{sett:stationary}
\end{sett}

\begin{prop}
  \Cref{sett:stationary} implies \cref{sett:Schal} and \cref{cond:S}.
\end{prop}
\begin{proof}
  Pt. 1 is by definition.
  We naturally endow $\Cal{A}$ with the discrete topology
  and the powerset $\sigma$-algebra, making it
  standard Borel and compact.
  The last point in \cref{sett:Schal} is implied by
  \cref{asm:generalE} and the discounting in \cref{sett:stationary} pt. 4.
  Pt. 1 in \cref{cond:S} is trivial since all functions are
  continuous from a discrete space and pt. 2 is by definition.
\end{proof}

\begin{defn}[Markov and stationary policies]
  A policy $(\pi_n)_{n \in \N}$ called \defemph{Markov}
  if $\pi_n$ only depends on the last item in the history,
  that is $\pi_n(\cdot \mid s_1, a_1, \dots a_{n-1}, s_n)
  = \pi'_n(\cdot \mid s_n)$ for some kernel $\pi'_n : \Cal{S}
  \leadsto \Cal{A}$ for all $n \in \N$. The set of Markov policies
  we denote $M\Pi$.
 
  A Markov policy is called \defemph{stationary}
  if there exist a kernel $\pi : \Cal{S} \leadsto \Cal{A}$ such that
  $\pi'_n = \pi$ for all $n \in \N$.
  The space of such policies we denote $\Pi$.
\end{defn}
We remark that
\[ \Pi \subseteq M\Pi \subseteq R\Pi \]
\begin{prop}
  There is an optimal policy which is Markov.
\end{prop}
\begin{proof}
  We first show that there exist optimal finite-horizon Markov policy
  for each $n \in \N$.
  Let $\tau_1: \Cal{S} \leadsto \Cal{A}$ be a policy such that
  $\tau_1(\argmax_{a \in \Cal{A}} r(s, a) \mid s) = 1$.
  Then clearly
  $V^{\tau_1}_1(s) = \max_{a \in \Cal{A}} r(s, a) = V^*_1(s)$
  and $\tau_1$ is Markov as any one-step policy.
  For $n \in \N$ assume
  $(\tau_{n}, \dots, \tau_1)$ is an optimal finite-horizon
  Markov policy.
  Let
  \[ \tau_{n+1}\left(\argmax_{a \in \Cal{A}} r(s, a) + 
      \gamma \int V_n^{(\tau_n, \dots, \tau_1)}(s') \difd P(s' \mid s, a)
  \Mid s \right) \]
  Then $(\tau_{n+1}, \dots, \tau_1)$ is an $n+1$-optimal Markov policy.
  \dots
\end{proof}

\begin{defn}[The $T$-operators]
  For a stationary policy $\pi$ we define the operator
  $T^\pi$ on $\Cal{L}_\infty(\Cal{S})$ by
  \[ T^\pi(V) \defeq s \mapsto \int r(s, a)
  + \gamma V(s') \difd (P * \pi)(s, a, s'\mid s) \]
  When $\pi = \delta_a$ for some $a \in \Cal{A}$ we simply write $T^a$.
  %$V$ if $\pi_V( A_s \mid s ) = 1$ for all
  %$s \in \Cal{S}$. We define
  %\[ T \defeq V \mapsto T^{\pi_V} V \in \Cal{L}_\infty(\Cal{S}) \]
\end{defn}

\begin{prop}
  $T^\pi$ is $\gamma$-contractive on $\Cal{L}_\infty(\Cal{S})$.
\end{prop}
\begin{proof}
  Let $V, V' \in \Cal{L}_\infty(\Cal{S})$
  and let $K = \norm{V - V'}_\infty$.
  Then
  \[ \norm{T^\pi V - T^\pi V'}_\infty
    = \sup_{s \in \Cal{S}} \abs{ \gamma
    \int V - V' \difd (P \circ \pi)(\cdot \mid s)}
  \leq \gamma K \]
\end{proof}

Let $\pi = (\pi_n)_{n \in \N}$ be a
Markov policy.

\begin{prop}
  $V^\pi = T^{\pi_1} V^{(\pi_2, \dots)}$.
  In particular if $\pi$ is stationary then $V^\pi$ is the unique bounded
  fixed point for $T^\pi$.
\end{prop}
\begin{proof}
  Fix $s \in \Cal{S}$ and let $\rho_s^\pi$ be the process measure
  (see \cref{prop:polProc}) of $\pi$.
  \begin{align*}
    T^{\pi_1} V^{(\pi_2, \dots)}
    = & \; \int V^{(\pi_2, \dots)}(s_2) \difd (P * \pi_1)(s_1, a_1, s_2 \mid \mu)
    \\ = & \; \int \int \sum_{n=1}^\infty \gamma^n r(s'_{n+1}, a'_{n+1})
    \difd \rho_{s_2}^{(\pi_2, \dots)}(s'_1, a'_1, \dots)
    \difd (P * \pi_1)(s_1, a_1, s_2 | \mu)
    \\ = & \; \int \int \sum_{n=1}^\infty \gamma^n r(s'_{n+1}, a'_{n+1})
    \difd \rho_{(P \circ \pi_1)(\cdot \mid \mu)}^
    {(\pi_2, \dots)}(s'_1, a'_1, \dots)
  \end{align*}
\end{proof}

\begin{prop}
  There exists a stationary policy $\tau$ such that
  $V^\tau \geq V^\pi$.
\end{prop}
\begin{proof}

\end{proof}

%% scratch

%We include here a corollary to this theory by
%[todo ref. to Total Expected Discounted Reward MDPs: Existence of
%Optimal Policies, thm 14]. %todo ask Stefan: is this an OK reference?

%Even measure theoretic questions arise because
%it turns out that $V^*$ may not be Borel measurable.
%However it will belong to a broader class of functions,
%the so called \defemph{universally measurable},
%which means that we can integrate in the same way
%that we are used anyway.



\end{document}
