

\subsubsection{Differences in notation}
% Notational differences between this paper
% and [YangXieWang]

Because $\sigma$ is used ambigously in \cref{thm:main}
we denote the probability distribution $\sigma$
from \ncite{F20} p. 20 by $\nu$ instead.
I avoid the shorthand defined in
\ncite{F20} p. 26 bottom:
$\norm{f}_n^2 = 1/n \cdot \sum_{i=1}^n f(X_i)^2$.
and use $p$-norms instead.
The conversion to the notation used here becomes
$\norm{f}_n \leadsto \norm{f}/n$.
The letter $r$ is used in \ncite{F20} to denote the euclidean dimension of
the state space, while here we use $w$.

\subsubsection{The decision model}

\begin{sett}[Fan et al.]
  \leavevmode
  \begin{enumerate}
    \item We're considereing an MDP (\cref{sett:MDP}).
      That is a state and action space
      $(\Cal{S}, \Cal{A})$ and a transition and reward kernel $P, R$
      which only depends on the previous state-action pair.
    \item $S \subseteq \R^w$ is a compact subset of a euclidean space.
    \item $\Cal{A}$ is finite.
    \item Discounted factor satisfy $0 < \gamma < 1$.
  \end{enumerate}
  \label{sett:Fan}
\end{sett}

%todo establish connection to Bertsekas model, e.g. is this u.s.c?


\subsubsection{ReLU Networks}
\begin{defn}[Sparse ReLU Networks]
  For $s,V \in \R$ a (s,V)-\defemph{Sparse ReLU Network} is an ANN $f$
  with all activation functions being \emph{ReLU}
  i.e. $\sigma_{ij} = \max(\cdot, 0)$
  and with weights $(W_\ell, v_\ell)$ satisfying
  \begin{multicols}{3}
    \begin{itemize}
      \item $\max_{\ell \in [L+1]} \norm{\widetilde{W}_\ell}_\infty \leq 1$
      \item $\sum_{\ell = 1}^{L+1} \norm{\widetilde{W}_\ell}_0 \leq s$
      \item $\max_{j \in [d_{L+1}]} \norm{f_j}_\infty \leq V$
    \end{itemize}
  \end{multicols}
  Here $\widetilde{W}_\ell = (W_\ell, v_\ell)$.
  The set of them we denote $\Cal{F}\left(L, \{d_i\}_{i=0}^{L+1},s,V \right)$.
  \label{def:sparseReLU}
\end{defn}
The idea to work with this particular subclass of neural networks come from
\ncite{SH17} (p. 22), which establishes the following lemma

\begin{lem}[Approximation of Hölder Smooth Functions by ReLU networks]
  Let $m,M \in \Z_+$ with $N \geq \max\{(\beta + 1)^r, (H + 1) e^r\}$,
  $L = 8 + (m + 5) (1 + \ceil{\log_2(r + \beta)})$, 
  $d_0 = r, d_j = 6(r + \ceil{\beta}) N, d_{L+1} = 1$.
  Then for any $g \in \Cal{C}_r \left( [0,1]^r, \beta, H \right)$
  there exists a ReLU network
  $f \in \Cal{F}\left(L, \{d_j\}_{j=0}^{L+1}, s, \infty \right)$
  with $s \leq 141 (r + \beta + 1)^{3 + r} N (m+6)$
  such that
  \begin{equation*}
    \norm{f - g}_\infty \leq (2 H + 1) 6^r N (1 + r^2 + \beta^2) 2^{-m}
    + H 3^{\beta} N^{-\beta/r}
  \end{equation*}
  \label{lem:holderapprox} 
    %todo: decide whether to do this proof
\end{lem} 
\vspace*{-\baselineskip}

In the course of establishing the results in \ncite{F20} we will not go
very much into this result or other properties of ReLU networks in particular,
instead putting emphasis on how to use this result to obtain the main
theorem, which we will present shortly.

\subsubsection{Fitted Q-Iteration}
The algorithm analysed by [Fan et al] is
\begin{figure}[H]
\begin{algorithm}[H] %\label{algocf:fq} % this labels line, could not fix
  \caption{Fitted Q-Iteration Algorithm}
  \KwIn{MDP $(\Cal{S}, \Cal{A}, P, R, \gamma)$, function class $\Cal{F}$,
    sampling distribution $\nu$, number of iterations $K$,
  number of samples $n$, initial estimator $\widetilde{Q}_0$}
  \For{$k = 0,1,2,\dots,K-1$}{
    Sample i.i.d. observations $\{(S_i, A_i), i \in [n]\}$ from $\nu$
    obtain $R_i \sim R(S_i, A_i)$ and $S'_i \sim P(S_i, A_i)$ \\
    Let $Y_i = R_i + \gamma \cdot \max_{a \in \Cal{A}} \widetilde{Q}_k(S'_i, a)$ \\
    Update action-value function:
    \[ \widetilde{Q}_{k+1} \leftarrow
      \argmin_{f \in \Cal{F}} \frac{1}{n}
    \sum_{i=1}^n (Y_i - f(S_i, A_i))^2 \]
  }
  Define $\pi_K$ as the greedy policy w.r.t. $\widetilde{Q}_K$ \\
  \KwOut{An estimator $\widetilde{Q}_K$ of $Q^*$ and policy $\pi_K$}
  \label{alg:fqi}
\end{algorithm}
\end{figure}


