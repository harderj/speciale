% this should explain measure theoretic issues in the project

We are mostly concerned with
a random process
\begin{equation}
(Z_i)_{i=1}^K
= (S_i, A_i, R_i)_{i=1}^K \in (\Cal{S}\times\Cal{A}\times(0,R_{\max}))^K
\end{equation}
where $\Cal{S} \subseteq \R^d$ is compact and $\Cal{A}$ is finite,
so we can model this as a discrete (and finite) time random process in a compact
subset of $\R^{d+1}$ having the Markov property, namely that
\begin{equation}
  \Prob(Z_j \in A \mid Z_{j-1}, \dots, Z_1) = \Prob(Z_j \in A \mid Z_{j-1})
  \label{eq:MarkovProperty}
\end{equation}
These random variables live on some background probability space, denote this
$(\Omega, \Cal{H}, \Prob)$.
