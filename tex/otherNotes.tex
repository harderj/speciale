
\begin{defn}[Interior]
  For a subset $A \subseteq \cl{X}$
  of a topological space $(\cl{X}, \cl{O}_\cl{X})$
  the \defemph{interior} $A^\circ \subseteq A$ of $A$ is the
  union of all open sets $U \in \cl{O}_\cl{X}$ which are contained in $A$.
  That is
  \[A^\circ = \bigcup_{U \in \cl{U}} U, \text{ where } \cl{U}
  = \left\{ U \in \cl{O}_\cl{X} \Mid U \subseteq A \right\}\]
  \label{defn:interior}
\end{defn}

\begin{defn}[Order Topology]
  Given a totally ordered set $(\cl{X}, <)$ the \defemph{order topology}
  is the topology generated by the subbase of sets on the form
  \[ \left\{ x \mid a < x \right\},\; a \in \cl{X}\; \hrm{and} \;
  \left\{ x \mid x < b \right\}, b \in \cl{X} \]
  \label{defn:orderTop}
\end{defn}

\begin{defn}[$\sigma$-algebra]
  A $\sigma$-\defemph{algebra} $\Sigma$ on a set $\cl{X}$ is a pavement (family of
  subsets of $\cl{X}$) $\Sigma \subseteq 2^\cl{X}$ (where $2^\cl{X}$ denotes the
  powerset of $\cl{X}$) satisfying
  \begin{itemize}
    \item $\emptyset,\; \cl{X} \in \Sigma$.
    \item $A \in \Sigma \implies \cl{X} \setminus A \in \Sigma$.
    \item If $A_1, A_2, \dots \in \Sigma$
      are a countable collection of subsets of $\cl{X}$
      in $\Sigma$ then $\bigcup_{i \in \N} A_i \in \Sigma$.
  \end{itemize}
  The pair $(\cl{X}, \Sigma)$ of a set and a $\sigma$-algebra on it is
  called a \defemph{measurable space}.
  \label{defn:sigmaAlg}
\end{defn}

\begin{thm}
  For any pavement $\Gamma \subseteq 2^\cl{X}$ of a set $\cl{X}$ there exists
  a \emph{smallest} $\sigma$-algebra $\Sigma \subseteq 2^\cl{X}$ on $\cl{X}$
  satisfying
  \begin{enumerate}
    \item $\Gamma \subseteq \Sigma$.
    \item For any $\sigma$-algebra $\Sigma'$ for which $\Gamma \subseteq \Sigma'$
      it holds that $\Sigma \subseteq \Sigma'$.
  \end{enumerate}
  This smallest $\sigma$-algebra is denoted $\sigma(\Gamma)$.
\end{thm}

\begin{defn}[Borel $\sigma$-algebra]
  For a topological space the \defemph{Borel} $\sigma$-algebra is the smallest
  $\sigma$-algebra containing all open sets.
  \label{defn:BorelAlg}
\end{defn}

\begin{defn}[Product $\sigma$-algebra]
  Let $(\cl{X}_i, \cl{A}_i)_{i \in I}$ be a collection of measurable spaces.
  the product $\sigma$-algebra
  \[ \bigotimes_{i \in I} \cl{A}_i \]
  is the smallest $\sigma$-algebra making all coordinate projections
  $\rho_i : \prod_{j \in I} \cl{X}_j \to \cl{X}_i$
  measurable.
  In particular if $\abs{I} = 2$
  \[ \cl{A}_1 \otimes \cl{A}_2 = \sigma \left(
      \left\{ A_1 \times \cl{X}_2 \mid A_1 \in \cl{A}_1 \right\} \cup
  \left\{ \cl{X}_1 \times A_2 \mid A_2 \in \cl{A}_2 \right\} \right) \]
  \label{defn:prodSigmaAlg}
\end{defn}

\begin{defn}[Dynkin class]
  Let $D$ be a pavement of $X$,
  that is a collection of subsets of $X$.
  $D$ is called a \defemph{Dynkin class} if
  \begin{enumerate}
    \item $X \in D$,
    \item If $A, B \in D$ and $A \subseteq B$ then $B \setminus A \in D$,
    \item If $A_1, A_2, \dots \in D$ with $A_n \subseteq A_{n+1}$ for
      all $n \in \N$ then $\bigcup_{n=1}^\infty A_n \in D$.
  \end{enumerate}
  \label{defn:DynkinClass}
\end{defn}

\begin{thm}[Dynkins $\pi$-$\lambda$ theorem]
  Let $P$ be a pavement of $X$ which is stable under finite intersections
  (such are called $\pi$-systems) and $D$ a Dynkin class
  (see \cref{defn:DynkinClass}).
  If $P \subseteq D$ then $\sigma(P) \subseteq D$
  where $\sigma(P)$ is the smallest $\sigma$-algebra containing $P$.
  \label{thm:DynkinPiLambda}
\end{thm}

\begin{defn}[Measure]
  Given a measurable space $(\cl{X}, \Sigma)$ a \defemph{measure}
  is a function $\mu : \Sigma \to [0, \infty]$ satisfying
  \begin{enumerate}
    \item $\mu(\emptyset) = 0$
    \item $\mu\left( \bigcup_{i \in \N} A_i \right) =
      \sum_{i \in \N} \mu(A_i)$
      for any countable collection of mutually disjoint sets
      $A_1, A_2, \dots \in \Sigma$.
  \end{enumerate}
  If there exists a sequence of subsets
  $A_1 \subseteq A_2 \subseteq \dots \subseteq
  \cl{X}$ with $\bigcup_{i \in \N} A_i = \cl{X}$ and $\mu(A_i) < \infty$ for all
  $i \in \N$ then $\mu$ is called $\sigma$\defemph{-finite}.
  If $\mu(\cl{X}) < \infty$ then $\mu$ is called \defemph{finite},
  and if furthermore $\mu(\cl{X}) = 1$ then $\mu$ is called a
  \defemph{probability measure}.
\end{defn}

\begin{thm}[Carathéodory's extension theorem]
  Let $\cl{X}$ be a set and $\cl{S} \subset 2^\cl{X}$ be a pavement of $\cl{X}$
  satisfying
  \begin{enumerate}
    \item $\emptyset \in \cl{X}$
    \item $S,\; T \in \cl{S} \implies S \cap T \in \cl{X}$
    \item For $S,\; T \in \cl{S}$ there exists finitely many disjoint
      subsets $S_1, S_2, \dots, S_n \in \cl{S}$ so that
      $ S \setminus T = \bigcup_{i=1}^n S_i$.
  \end{enumerate}
  ($\cl{S}$ is then called a \emph{semi-ring}).
  Let $\mu : \cl{S} \to [0,\infty]$ be a function satisfying
  \begin{enumerate}[label=\roman*.]
    \item $\mu(\emptyset) = 0$
    \item For a countable mutually disjoint collection of subsets
      $S_1, S_2, \dots \in \cl{S}$ it holds that
      $\mu\left( \bigcup_{i \in \N} S_i \right)
      = \sum_{i \in \N} \mu(S_i)$.
  \end{enumerate}
  Then $\mu$ has an extension to a measure $\mu$ on $\sigma(\cl{S})$.
  Furthermore if there exists an increasing sequence of subsets
  $S_1 \subseteq S_2 \subseteq \dots \in \cl{S}$ of $\cl{S}$
  satisfying $\bigcup_{i \in \N} S_i = \cl{X}$ and
  $\mu(S_i) < \infty$ for all $i \in \N$ then
  the extension is unique.
  In particular if $\cl{X} \in \cl{S}$ and $\mu(\cl{X})=1$ then
  $\mu$ extends uniquely to a probability measure on $(\cl{X}, \sigma(\cl{S}))$.
  \label{thm:caratheo}
\end{thm}

\begin{defn}[Measurable function]
  A functions $f : \cl{X} \to \cl{Y}$ between two measurable spaces
  are called \defemph{measurable} if
  \[ f^{-1}(\Sigma_\cl{Y}) =
    \left\{ f^{-1}(B) \mid B \in \Sigma_\cl{Y} \right\}
  \subseteq \Sigma_\cl{X} \]
  The set of such functions we denote
  $\cl{M}(\Sigma_\cl{X}, \Sigma_\cl{Y})$ or $\cl{M}(\cl{X}, \cl{Y})$.
  \label{defn:measFunc}
\end{defn}

\begin{defn}[Almost sure uniform convergence of random processes]
  A sequence of random processes $X_n : \Cal{X} \times \Omega \to \R$
  is said to converge \defemph{almost surely
  uniformly} to $X: \Cal{X} \times \Omega \to \R$ if and only if
  \[ \Prob(\sup_{x \in \Cal{X}} \abs{X_n(x) - X(x)} \to 0) = 1 \]
\end{defn}

\begin{defn} [Uniform convergence in probability of random processes]
  A sequence of random processes $X_n: \Cal{X} \times \Omega \to \R$
  is said to converge \defemph{uniformly
  in probability} to $X : \Cal{X} \times \Omega \to \R$ if and only if
  \[ \sup_{x \in \Cal{X}} \abs{X_n(x) - X(x)} \overset{P}{\to} 0 \]
  \label{defn:uniformConvProb}
\end{defn}

\begin{defn}
  A sequence of events $A_1, A_2, \dots \subseteq \Omega$
  is said to be \defemph{asymptotically almost sure}
  if $\Prob(A_k) \to 1$ for $k \to \infty$.
  \label{defn:aas}
\end{defn}

\begin{example}
  For example if $U_1, U_2, \dots \sim \hrm{Unif}(0,1)$ are i.i.d. random
  variables, $X_k = \max_{i \in [k]} U_i$ for $k \in \N$ and $\ve > 0$ then
  the events $(A_k)_{k \in \N} = (X_k > 1 - \ve)_{k \in \N}$
  are asymptotically almost sure
  since $\Prob(A_k) \to 1$ as $k \to \infty$.
  The property
  $X_k > 1 - \ve$ is then said to hold \emph{asympotically almost surely}.
  \label{example:aas}
\end{example}

\begin{prop}
  $\id_{\cl{P}(X)} = \mu \mapsto \kappa \circ \mu$
  where $\kappa(\cdot \mid x) = \delta_x(\cdot)$.
  Thus $\kappa$ can be seen as an identity mapping on $\cl{P}(X)$.
  \label{prop:identityKernel}
\end{prop}
\begin{proof}
  \[ \kappa \mu (A) = \int \delta_x(A) \difd \mu(x) = \mu (A) \]
\end{proof}

\begin{defn}[Lipschitz continuity]
  Let $(\cl{X}, d_\cl{X}), \; (\cl{Y}, d_\cl{Y})$ be metric spaces.
  A function $f: \cl{X} \to \cl{Y}$ is
  said to \defemph{Lipschitz} with constant $L > 0$ if
  \[ d_\cl{Y}(f(x),f(y)) \leq L d_\cl{X}(x,y) \]
  \label{defn:Lipschitz}
\end{defn}

\begin{defn}[Differentiability in one variable]
  A function $f : A \to \R$ where $A \subseteq \R$ is an open subset of the
  real numbers is \defemph{differentiable}
  at $x \in \R$ if the \defemph{derivative}
  \[ f'(x) \defeq \lim_{x_n \to x} \frac{f(x) - f(x_n)}{x - x_n} \]
  exists, is finite and is the same for any sequence
  $(x_n)_{n \in \N} \subseteq A$
  converging to $x$
  with $x_n \neq x$ for all $n \in \N$.
  If $f$ is differentiable at $A$ if it is differentiable for every
  $x \in A$.
  If $f' : A \to \R$ is continuous then we write $f \in C^1(A)$.
  If $f'' = (f')' : A \to \R$ exists and is continuous we write
  $f \in C^2(A)$. Like this for $k \in \N_0$ we say that $C^k$
  is the set of $k$ times continuously
  differentiable functions, 
  and we write $f^{(k)}$ for the $k$th derivative,
  when $k=0$ we have $C^0(A) = C(A)$ the set of continuous functions and
  $f^{(0)} = f$.
  This extends to $C^\infty$, called the set of
  \defemph{smooth} functions, for any element is continuously differentiable
  $n$ times for any $n \in \N_0$.
  \label{defn:diffR}
\end{defn}

\begin{defn}[Partial derivatives]
  Let $f : U \to \R$ where $U \subseteq \R^n$ 
  is open be a function
  satisfying for some $x = (x_1, \dots, x_n) \in U$ that
  $f_{x, i} = x_i \mapsto f(x_1, \dots, x_i, \dots, x_n) \in C^1(\rho_i(U))$
  where $\rho_i : U \to \R$ is projection onto the $i$th coordinate.
  The partial derivative of $f$ with respect to the $i$th variable at $x$ is the
  function $\delta_i f(x) \defeq f_{x, i}'(x_i)$.
  For $k \in \N_0$
  if $f_{x, i} \in C^k$ then write
  $\delta_i^k f(x) \defeq f^{(k)} f_{x, i} (x_i)$ whenever this exists.
  If $\alpha = (\alpha_1, \dots, \alpha_n) \in \N_0^n$ we denote by
  $\delta^\alpha f(x) \defeq \delta_1^{\alpha_1} \dots \delta_n^{\alpha_n} f(x)$.
  \label{defn:partialDer}
\end{defn}
\begin{rem}
  A standard result called \emph{Schwartz's theorem} say that the order
  in which partial derivatives are taken does not matter when
  these such derivates are continuous.
\end{rem}

\begin{defn}[Differentiability in $\R^n$]
  A function $f: U \to \R$ defined on an open set $U \subseteq \R^n$
  is said to be $C^k$ for $k \in \N_0$ if
  the partial derivatives
  $\partial^\alpha f : U \to \R$ exists and is continuous for all
  $\alpha \in \N_0^n$ with $\norm{\alpha}_1 = \alpha_1 + \dots + \alpha_n \leq k$.
  \label{defn:diffRn}
\end{defn}

\begin{defn}[Absolutely continuity of measures]
  Let $\mu, \nu \in \cl{P}(\cl{X})$ be $\sigma$-finite measures
  then $\mu$ is said to be \defemph{absolutely continuous} with respect to 
  $\nu$, written $\mu << \nu$ if for all $A \in \Sigma_\cl{X}$ we have
  $\nu(A)=0 \implies \mu(A)=0$.
  \label{defn:absContMeas}
\end{defn}

\begin{thm}[Radon-Nikodym]
  Let $\mu, \nu \in \cl{P}(\cl{X})$ with $\mu << \nu$.
  Then there exists a positive measurable function
  $f : \cl{X} \to [0, \infty)$
  such that $\mu(A) = \int_A f \difd \nu$.
  This function is denoted $f = \frac{\difd \mu}{\difd \nu}$.
  %By some authors also called the \emph{Rattata-Nidoking} theorem.
  \label{thm:radonNiko}
\end{thm}

\begin{thm}[Banach fixed point theorem]
  Let $(\Cal{X}, d)$ be a complete metric space
  and $T:\Cal{X} \to \Cal{X}$ be a contraction,
  i.e. $d(Tx, Ty)<\gamma d(x, y)$ for some $0 < \gamma < 1$
  and all $x,y \in \Cal{X}$.
  Then $T$ has a unique fixed point $x^*$ and for every $x\in \Cal{X}$
  it holds that $T^k x \to x^*$ as $k \to \infty$, with rate
  $d(T^k x, x^*) < \gamma^k d(x, x^*)$.
  \label{thm:BanachFP}
\end{thm}

