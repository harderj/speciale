
\subsection{Measure Theory}

We work with a background probability space $(\Omega, \Sigma_\Omega, \Prob)$.
For a measurable space $(\Cal{X}, \Sigma_{\Cal{X}})$ we denote
the set of probability measures on this space $\Cal{P}(\Sigma_\Cal{X})$ or
simply $\Cal{P}(\Cal{X})$ when the $\sigma$-algebra is unambiguous.
When taking cartesian products $\Cal{X} \times \Cal{Y}$ of measurable spaces
$(\Cal{X}, \Sigma_\Cal{X}), (\Cal{Y}, \Sigma_\Cal{Y})$ we always endow such
with the product $\sigma$-algebra $\Sigma_\Cal{X} \otimes \Sigma_\Cal{Y}$,
unless otherwise specified.
A map $f: \Cal{X} \to \Cal{Y}$ is called $\Sigma_{\Cal{X}}$-$\Sigma_{\Cal{Y}}$
measurable provided $f^{-1}(\Sigma_{\Cal{Y}}) \subseteq \Sigma_{\Cal{X}}$
and we denote the set of such functions $\Cal{M}(\Sigma_{\Cal{X}},
\Sigma_{\Cal{Y}})$.
By a random variable $X$ on $(\Cal{X}, \Sigma_{\Cal{X}})$ mean a
$\Sigma_\Omega$-$\Sigma_{\Cal{X}}$ measurable map.

\subsubsection{Kernels}

\begin{defn}[Probability kernel]
  Let $(\Cal{X}, \Sigma_\Cal{X}), (Y, \Sigma_\Cal{Y})$ be measurable spaces.
  A function
  \[ \kappa(\cdot \mid \cdot) : \Sigma_\Cal{Y} \times \Cal{X} \to [0,1] \]
  is a $(\Cal{X}, \Sigma_\Cal{X})$-\defemph{probability kernel}
  on $(\Cal{Y}, \Sigma_\Cal{Y})$ provided
  \begin{enumerate}
    \item $B \mapsto \kappa(B \mid x) \in \Cal{P}(\Sigma_\Cal{Y})$
      that is $\kappa(\cdot \mid x)$ is a probability measure
      for any $x \in \Cal{X}$.
    \item
      $x \mapsto \kappa(B \mid x) \in \Cal{M}(\Sigma_\Cal{X}, \Sigma_\Cal{Y})$
      that is $\kappa(B \mid \cdot)$ is ($\Sigma_\Cal{X}$-$\Sigma_\Cal{Y}$)
      measurable for any $B \in \Sigma_\Cal{Y}$.
  \end{enumerate}
  When the $\sigma$-algebras are unambiguous we shall simply say an
  $\Cal{X} \leadsto \Cal{Y}$ kernel.
  For any $x \in \Cal{X}$ and $f \in \Cal{L}_1(\kappa(\cdot \mid x))$
  we write the integral of $f$ over $\kappa(\cdot \mid x)$ as
  $\int f(y) \difd \kappa(y \mid x)$.
  \label{defn:probKer}
\end{defn}

We now state some fundamental results on probability kernels
\begin{thm}[Integration of a kernel]
  Let $\mu \in \Cal{P}(\Cal{X})$ and $\kappa : \Cal{X} \leadsto \Cal{Y}$.
  Then there exists a uniquely determined probability measure
  $\lambda \in \Cal{P}(\Sigma_\Cal{X} \otimes \Sigma_\Cal{Y})$
  such that
  \[ \lambda(A \times B) = \int_A \kappa(B, x) \difd \mu(x) \]
  \label{thm:intKer}
  We denote this measure $\lambda = \kappa \mu$.
\end{thm}
\begin{proof}
  We refer to [ref to EH markov, thm. 1.2.1].
\end{proof}

Notice that by \cref{thm:intKer}
besides getting a probability measure on $\Cal{X} \times \Cal{Y}$
we get an induced probability measure
on $\Cal{Y}$ defined by $B \mapsto (\kappa \mu)(\Cal{X} \times B)$.
We will denote this measure by $\kappa \circ \mu$.
This way $\kappa$ can also be seen as a mapping from
$\Cal{P}(\Cal{X}) \to \Cal{P}(\Cal{Y})$.
Also note that $\kappa \circ \delta_x = \kappa(\cdot \mid x)$.

For an idea how to actually compute integrals over kernel derived measures
we here include
\begin{thm}[Extended Tonelli and Fubini]
  Let $\mu \in \Cal{P}(\Cal{X})$,
  $f \in \Cal{M}(\Sigma_\Cal{X} \otimes \Sigma_\Cal{Y}, \bb{B})$
  be a measurable function and
  $\kappa : \Cal{X} \leadsto \Cal{Y}$ be a probability kernel.
  Then
  \[ \int \abs{f} \difd \kappa \circ \mu
  = \int \int \abs{f} \difd \kappa(\cdot \mid x) \difd \mu(x) \]
  Furthermore if this is finite, i.e. $f \in \Cal{L}_1(\kappa(\cdot, \mu))$
  then $A_0 \defeq \left\{ x \in \Cal{X} \Mid
    \int f \difd \kappa(\cdot \mid x) < \infty \right\}
  \in \Sigma_\Cal{X}$
  with $\mu(A_0) = 1$, 
  \[ x \mapsto \begin{cases}
      \int f \difd \kappa(\cdot \mid x) & x \in A_0
      \\ 0 & x \not\in A_0
  \end{cases} \]
  is $\Sigma_\Cal{X}$-$\bb{B}$ measurable and
  \[ \int f \difd \kappa(\cdot \mid \mu)
  = \int_{A_0} \int f \difd \kappa(\cdot \mid x) \difd \mu(x) \]  
  \label{thm:extTonFub}
\end{thm}
\begin{proof}
  We refer to [ref to EH markov, thm. 1.3.2 + 1.3.3]
\end{proof}

\begin{prop}[Composition of kernels]
  Let $\kappa : \Cal{X} \leadsto \Cal{Y}, \psi : \Cal{Y} \leadsto \Cal{Z}$
  be probability kernels. Then
  \[ (\psi \circ \kappa)(A \mid x) \defeq
    \int \psi(A \mid y) \difd \kappa(y \mid x)
  ,\qquad \forall A \in \Sigma_{\Cal{Z}}, x \in \Cal{X} \]
  is a $\Cal{X} \leadsto \Cal{Z}$ probability kernel called the
  composition of $\kappa$ and $\psi$. The composition operator
  $\circ$ is associative, i.e. if $\phi : \Cal{Z} \leadsto \Cal{W}$ is
  a third probability kernel then $(\phi \circ \psi) \circ \kappa = 
  \phi \circ (\psi \circ \kappa)$.
  The associativity also extends to measures, i.e.
  $\forall \mu \in \Cal{X}
  : (\psi \circ \kappa) \circ \mu = \psi \circ (\kappa \circ \mu) $
  and this is uniquely determined by $\psi, \kappa$ and $\mu$.
  \label{prop:compKer}
\end{prop}
\begin{proof}
  The first assertion is a trivial verification of the two conditions
  in \cref{defn:probKer} and left as an exercise.
  For the associativity we refer to [todo ref to EH markov, lem. 4.5.4].
\end{proof}

\Cref{prop:compKer} actually makes the class of measurable spaces
into a category [todo ref: see Lawvere, The Category of Probabilistic
Mappings], with identity $\id_{\Cal{X}}(\cdot \mid x) = \delta_x$.
Notice that the mapping $(A, x) \mapsto \delta_x(A) \kappa(A \mid x)$
defines a probability kernel $\Cal{X} \leadsto \Cal{X} \times \Cal{Y}$
which we could denote $\id_{\Cal{X}} \times \kappa$.
Now if $\psi : \Cal{X} \times \Cal{Y} \leadsto \Cal{Z}$ is a kernel
then by \cref{prop:compKer} the composition
$(\id_{\Cal{X} \times \Cal{Y}} \times \psi)
\circ (\id_{\Cal{X}} \times \kappa)$
is a kernel $\Cal{X} \to \Cal{X} \times \Cal{Y} \times \Cal{Z}$
which we will denote $\psi \kappa$.
It inherits associativity from $\circ$ and again this associativity
extends to application on measures: if $\mu$ is a measure on $\Cal{X}$
then $\psi (\kappa \mu) = (\psi \kappa) \mu$.

\begin{prop}
  Let $\kappa : \Cal{X} \to \Cal{Y}$ be a probability kernel
  and $f : \Cal{Y} \to \ol{\ul{\R}}$ be integrabel.
  Then $x \mapsto \int f \difd \kappa(\cdot \mid x)$ is measurable
  into $(\ol{\ul{\R}}, \ol{\ul{\bb{B}}})$.
  \label{prop:intKerMeas}
\end{prop}
\begin{proof}
  Simple functions are measurable since $\kappa$ is a kernel.
  Now extend by sums and limits.
\end{proof}

\subsubsection{Kernel derived processes}

Let $(\Cal{X}_n, \Sigma_{\Cal{X}_n})_{n \in \N}$ be a sequence
of measurable spaces. For each $n \in \N$ define
$\Cal{X}^{\ul{n}} \defeq \Cal{X}_1 \times \dots \times \Cal{X}_n$,
$\Sigma_{\Cal{X}^{\ul{n}}} \defeq \Sigma_{\Cal{X}_1} \otimes
\dots \otimes \Sigma_{\Cal{X}_n}$
and let
$\kappa_n : \Cal{X}^{\ul{n}} \leadsto \Cal{X}_{n+1}$ be a probability kernel.
Then $\kappa^{\ul{n}} \defeq \kappa_n \dots \kappa_1$ is a kernel
from $\Cal{X}_1$ to $\Cal{X}^{\ul{n}}$.
So for any probability measure $\rho_1 \in \Cal{P}(\Cal{X}_1)$
there exists a unique probability measure 
$\rho_n$ on $\Cal{X}^{\ul{n}}$ defined by
$\kappa^{\ul{n}} \rho_1$.

Let $\Cal{X}^{\ul{\infty}} \defeq \prod_{n \in \N} \Cal{X}_n$
and $\Sigma_{\Cal{X}^{\ul{\infty}}} \defeq \bigotimes_{n \in \N}
\Sigma_{\Cal{X}_n}$.
We are not equipped to establish existence of a
kernel generated measure on
$(\Cal{X}^{\ul{\infty}}, \Sigma_{\Cal{X}^{\ul{\infty}}})$ 
yet which we will need.
This problem was solved by Cassius Ionescu-Tulcea in 1949:

\begin{thm}[Ionescu-Tulcea extension theorem]
  For every $\mu \in \Cal{P}(\Cal{X}_1)$ 
  there exists a unique probability measure
  $\rho \in \Cal{P}(\Cal{X}^{\ul{\infty}})$ such that
  \[ \rho_n(A) = \rho \left( A \times \prod_{k=n+1}^\infty \Cal{X}_k \right)
  , \qquad \forall A \in \Sigma_{\Cal{X}^{\ul{n}}}, n \in \N \]
  We denote this measure
  $\dots \kappa_2 \kappa_1 \mu = \prod_{i=1}^\infty \kappa_i \mu \defeq \rho$.
  \label{thm:ionescuTulcea}
\end{thm}
\begin{proof}
  Todo: what about this.
\end{proof}

\begin{prop}[Ionescu-Tulcea kernel]
  Let $\mu_x$ denote the Ionescu-Tulcea measure of a
  sequence of probability kernels
  $\kappa_i : \Cal{X}^{\ul{i}} \to \Cal{X}_{i+1}$
  with starting measure $\delta_x$ on $\Cal{X}_1$ for any $x \in \Cal{X}_1$.
  Then $\kappa(A \mid x) = \mu_x(A)$ defines a probability kernel
  $\kappa : \Cal{X}_1 \to \Cal{X}^{\ul{\infty}}$.
\end{prop}
\begin{proof}
  Since we already know that $\mu_x$ is a probability measure for any
  $x \in \Cal{X}_1$,
  we just have to show that $\kappa(A \mid x) = \mu_x(A)$ is measurable for all
  $A \in \bigotimes_i \Sigma_{\Cal{X}_i}$.
  \dots todo
\end{proof}

\begin{lem}
  The Ionescu-Tulcea kernel satisfies
  $\prod_{i=1}^\infty \kappa_i = (\prod_{i=2}^\infty \kappa_i) \kappa_1 $.
  \label{lem:ionescu}
\end{lem}
\begin{proof}
  Let $x \in \Cal{X}_1$.
  Notice that by associativity of the finitely induced measures
  $\kappa_n \dots \kappa_1 \delta_x
  = (\kappa_n \dots \kappa_2) (\kappa_1 \delta_x)$.
  This implies that
  \[ \prod_{i=1}^\infty \kappa_i \delta_x
    \left( A \times \prod_{k=n+1}^\infty \Cal{X}_k \right)
    = \prod_{i=2}^\infty \kappa_i \kappa_1 \delta_x
  \left( A \times \prod_{k=n+1}^\infty \Cal{X}_k \right) \]
  for all $n \in \N$ and $A \in \Sigma_{\Cal{X}^{\ul{n}}}$.
  By the uniqueness in \cref{thm:ionescuTulcea} we are done.
\end{proof}


\begin{prop}
  Let $\Cal{X}, \Cal{Y}$ be separable and metrizable,
  $\kappa : \Cal{X} \to \Cal{Y}$ be a continuous probability kernel
  and $f:\Cal{X} \times \Cal{Y} \to \ul{\ol{\R}}$ be Borel-measurable
  satisfying one of
  $f \leq 0, f \geq 0, \abs{f} < \infty$.
  If $f$ is bounded from above (below) and upper (lower) semicontinuous
  then
  \[ x \mapsto \int f \difd \kappa(\cdot \mid x) \]
  is bounded from above (below) and upper (lower) semicontinuous. 
  \label{prop:BS7_31}
\end{prop}

\begin{proof}
  We refer to [BS SOC, prop. 7.31]. %todo do it yourself
\end{proof}

