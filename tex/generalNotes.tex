
In Reinforcement Learning (RL) in general
we are concerned with finding an optimal policy
for an agent in some environment.
This environment is described by
a sequence of state and action spaces
$\Cal{S}_1, \Cal{A}_1, \Cal{S}_2, \dots$
and rules $P_1, R_1, P_2, \dots$ specifying which states and rewards
and likely to follow after some action is chosen.
One can then specify rules $\pi$, called a \emph{policy},
for how the agent should act in every situation is the environment.
Given an environment and a policy one obtains stochastic process,
that is, a distribution on sequences of states, actions and
rewards.
One can then measure the performance of the policy by looking at
the expected sum of rewards called the \emph{value function}
$V_\pi$ of the policy.
The goal of reinforcement learning is to find an optimal policy $\pi^*$,
maximizing the value function.

$V_\pi$ is viewed as function that evaluates for each \emph{starting state}
$s \in \cl{S}_1$ the expected total return.
There might therefore be different optimal policies for each such starting
state.
Traditionally one defines an optimal value function $V^*(s)$
by taking supremum over all policies $\sup_\pi V_\pi(s)$ for every state
$s \in \cl{S}_1$.
Then an optimal policy $\pi^*$ should satisfy $V_{\pi^*} = V^*$,
i.e. it should be optimal uniformly across all starting states $\cl{S}_1$.
The existence of optimal policies defined in this way is a non-trivial
question and we will devote some time on this.

A particular kind class of environments are called Markov decision processes
(MDPs),
and work with the same state space $\cl{S}$, action space $\cl{A}$ and rules
$P, R$ throughout the process.
In an MDP we can use a value function $V_1$ to obtain a policy $\pi_1$
by choosing actions
leading to states with high values (according to $V_1$).
We can then evaluate value of $\pi_1$ yielding a new value function $V_2$.
This process can be continued indefinitely yielding a sequence of value
functions and policies.
Variations of this idea are called \emph{value iteration} and
\emph{policy iteration}, and have been shown to converge to the optimal
policy in many cases.

A problem with value functions defined on the set of states $\cl{S}$ is that
picking optimal actions require knowledge of the transition dynamics $P$.
Often we want to design algorithms that do not require such knowledge of $P$,
so called \emph{model-free} algorithms.
To meet this requirement \emph{Q-functions} are introduced, which evaluates
the value of a state-action pair, instead of only a state.
Given a Q-function $Q$, picking best actions according to $Q$ now
merely require maximization over $Q$ itself.
This is obviously an advantage if we are in situations where $P$ is actually
unknown. %todo find good example
However it turns out also to be more convenient to work with computationally.
In this thesis we show that value and policy iteration can be done
for Q-functions in a virtually identical manner, when the process dynamics
are known.

When the process dynamics are hidden designing algorithms becomes trickier.
In such settings approaches to the problem
fall in two categories. In the \emph{indirect} approaches
one attempts to estimate the process dynamics first and then afterwards
methods for the known-dynamics are applied.
The \emph{direct} approaches basically covers \emph{the rest}.
In the direct category we find the popular \emph{temporal difference}
algorithms on which \emph{fitted Q-iteration}
and \emph{deep Q-learning} as used in \ncite{M15} is based.

A further categorization of RL-algorithms can be made into
\emph{off-policy} and \emph{on-policy} classes.
This is simply whether the algorithm learns from data
(states, actions and rewards) arising from 
following its own policy (on-policy) or it can learn from more
arbitrary data (off-policy).
This \emph{more arbitrary data} could for example be
the trajectory of another algorithm when interacting with a decision process,
or simply state-action-reward pairs drawn from some distribution.
As an example \emph{fitted Q-iteration} is off-policy while
\emph{deep Q-learning} is on-policy.

