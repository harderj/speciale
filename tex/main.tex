\documentclass{article}

% preamble speciale Jacob Harder
% 31. jan. 2020

%packages

\usepackage[utf8]{inputenc} %utf8 is probably good
\usepackage{amsmath}
\usepackage{amssymb}
\usepackage{amsthm}
\usepackage{graphicx} %for including images
\usepackage{float} %for exact placement of figure (and more?)
\usepackage{mathtools} %for \mathclap (stacking under sums)
\usepackage{dsfont} %for boldface numbers
\usepackage{bm} %vectors in bold
\usepackage[ruled,vlined,linesnumbered]{algorithm2e}
\usepackage{cleveref}
\usepackage{ifthen}
\usepackage{commath}
\usepackage[a4paper,width=150mm,top=25mm,bottom=25mm]{geometry}
\usepackage{upgreek}
%\usepackage[inline]{enumitem}
\usepackage{multicol}
%\usepackage{fancyhdr} %maybe later..
%\pagestyle{fancy}
\usepackage{mathabx}
\usepackage{csquotes}
\usepackage{outline} %for subitems in lists
\usepackage[inline]{enumitem} % for e.g. horizontal enumerate
\usepackage{wrapfig}

\usepackage{tikz}
\usetikzlibrary{calc,trees,positioning,arrows,chains,shapes.geometric,%
    decorations.pathreplacing,decorations.pathmorphing,shapes,%
    matrix,shapes.symbols}

%front page
\usepackage{wallpaper}
\usepackage{titling}

%bibliography
\usepackage[numbers]{natbib}
\bibliographystyle{plainnat}
\newcommand{\mcite}[1]{[\citenum{#1}, \citeauthor{#1} (\citeyear{#1})]}
\newcommand{\ncite}[1]{\cite{#1}}

%linespread and geometry
\linespread{1.3}

%commands
\newcommand{\Cal}{\mathcal} % to be deleted
\newcommand{\cl}{\mathcal}
\newcommand{\fk}{\mathfrak}
\newcommand{\bb}{\mathbb}
\newcommand{\hrm}{\mathrm}
\newcommand{\Q}{\bb{Q}}
\newcommand{\Z}{\bb{Z}}
\newcommand{\N}{\bb{N}}
\newcommand{\R}{\bb{R}}
\newcommand{\C}{\bb{C}}
\newcommand{\E}{\bb{E}}
\newcommand{\Rext}{\ol{\R}}
\newcommand{\Var}{\mathrm{Var}}
\newcommand{\Prob}{\mathds{P}} %fundamental probability measure
\newcommand{\idc}{\mathds{1}}
\newcommand{\ve}{\varepsilon} %abbreviation for epsilon
\newcommand{\Yp}{\Upupsilon} %abbreviation for Ypsilon
\newcommand{\difd}{\; \mathrm{d}} %differential d
\newcommand{\wt}{\widetilde}
\newcommand{\wh}{\widehat}
\newcommand{\ol}{\overline}
\newcommand{\ul}{\underline}
\newcommand{\Mid}{\;\middle\vert\;}
\newcommand{\id}{\text{id}}
\newcommand{\supp}{\text{supp}}
\newcommand{\defemph}[1]{\textbf{#1}} %first-mentions of names
\DeclarePairedDelimiter\ceil{\lceil}{\rceil}
\DeclarePairedDelimiter\floor{\lfloor}{\rfloor}
\DeclareMathOperator*{\argmax}{argmax}
\DeclareMathOperator*{\argmin}{argmin}
\newcommand{\defeq}{\vcentcolon=} %definition equality symbol
%add single eq. tag in align*
\newcommand\numberthis{\addtocounter{equation}{1}\tag{\theequation}}
\newcommand{\rleft}[1]{\rotatebox[origin=c]{90}{\ensuremath{#1}}}
\newcommand{\vrel}[3]{ % for vertical subseteq e.g.
\vcenter{\halign{\hfill##\hfill\cr
\ensuremath{#1}\cr
\rotatebox[origin=c]{270}{\ensuremath{#2}}\cr
\ensuremath{#3}\cr
}}}
\newcommand{\lar}{\leftrightarrow}
\newcommand{\Span}{\mathrm{span}}
\newcommand{\Gr}{\mathrm{Gr}}

%theorems
\theoremstyle{definition}
\newtheorem{thm}{Theorem}[chapter]
\newtheorem{lem}[thm]{Lemma}
\newtheorem{defn}[thm]{Definition}
\newtheorem{cor}[thm]{Corollary}
\newtheorem{rem}[thm]{Remark}
\newtheorem{prop}[thm]{Proposition}
\newtheorem{asm}{Assumption}
\newtheorem{example}[thm]{Example}
%\newtheorem{cond}{Condition}
\newtheorem{sett}{Setting}
\newtheorem{innercond}{Condition}
\newenvironment{cond}[1]
  {\renewcommand\theinnercond{#1}\innercond}
  {\endinnercond}
%cref
\crefname{algocf}{alg.}{algs.}
\Crefname{algocf}{Algorithm}{Algorithms}
\crefname{innercond}{}{}
\Crefname{innercond}{}{}

%allow page breaks in align
\allowdisplaybreaks



\title{A Theorical Analysis of Fitted Q-Iteration}

\author{Jacob Harder \\ University of Copenhagen}

\begin{document}

\maketitle

\subsection*{Abstract}

In this paper we present
\begin{enumerate}
  \item a framework for studying Q-learning for decision processes with
    in the generality of non-Markov dynamics and
    continuous state and action spaces
  \item sufficient criteria for existence of optimal policies in general
    (possibly non-Markov and with continuous state and action spaces)
    decision processes based on \mcite{S75} and \mcite{BSA83}
  \item relations between value-iteration and Q-iteration and their convergence
    properties in the setting of Markov decision processes with
    continuous state and action space
  \item bounds on deviations from optimality of Q-iteration when using function
    approximators focusing in particular on two function classes:
    \begin{enumerate}
      \item Artifical neural networks
      \item Bernstein polynomials
    \end{enumerate}
\end{enumerate}

\section{Introduction}
\subsection{Foreword}

I came upon the idea to write about Q-learning when
I was fascinated by the performance of the algorithms
implemented by \mcite{M15}.

Coming from a mathematical background
my main purpose of this master thesis was initially
to investigate what has been proven
about the convergence of Q-learning algorithms
and what mathematical theory is relevant to establish such proofs.
In particular Q-learning algorithms using
artificial neural networks.

In the course of this I discovered that the frameworks
and settings in which various Q-leaning algorithms are analysed
varies greatly across litterature.
Also questions as to in which degree optimal strategies
exist in these various frameworks turns out to be
non-trivial when the state and action spaces are uncountable.

Therefore this paper is partially about building a framework
for analysing Q-learning algorithms in a variety of settings.
And partially to present the results that occur in each setting
and discuss their importance and generality.


%\subsection{Reinforcement Learning}
In Reinforcement Learning (RL) we are concerned with finding an optimal policy
for an agent in some environment.
Typically (also in the case of Q-learning) this environment is a
Markov decision process

\begin{Definition}
	A Markov decision process (MDP) $(\Cal{S}, \Cal{A}, P, R, \gamma)$
	consists of
\begin{itemize}
\item $\Cal{S}$ a set of states
\item $\Cal{A}$ a set of actions
\item $P : \Cal{S} \times \Cal{A} \to \Cal{P}(\Cal{S})$ its Markov transition kernel
\item $R : \Cal{S} \times \Cal{A} \to \Cal{P}(\R)$ its immediate reward distribution
\item $\gamma \in (0,1)$ the discount factor
\end{itemize}
\end{Definition}

A policy (for an MDP) is a function
\[\pi : \Cal{S} \to \Cal{P}(\Cal{A})\]
With this we can define the state-value function $V^\pi : \Cal{S} \to \R$
\[ V^\pi(s) = \E \left( \sum_{t \geq 0} \gamma^t R_t \mid
R_t \sim R(S_t, A_t), S_t \sim P(S_{t-1}, A_{t-1}), A_t \sim \pi(S_t), S_0 = s
\right) \]
And the state-action-value (Q-) function $Q^\pi : \Cal{S} \times \Cal{A} \to \R$
\[ Q^\pi(s, a) = \E (R(s, a) + \gamma V^\pi(S_0) \mid S_0 \sim P(s, a)) \]
The optimal Q-function is defined as
\[ Q^*(s, a) = \sup_\pi Q^\pi(s, a) \]
One can show that there is a policy $\pi^*$ such that $Q^* = Q^{\pi^*}$.
This is the optimal policy - the goal of RL. 

Note that $V^\pi$, $Q^\pi$ and $Q^*$ are usually infeasible to calculate to
machine precision, unless $\Cal{S} \times \Cal{A}$ is finite and not very big.

\subsection{Q-Learning}

Let $\pi: \Cal{S} \to \Cal{P}(\Cal{A})$ be a policy. We define the operator
\[ (P^\pi Q)(s, a) = \E (Q(S', A') \mid S' \sim P(s, a), A' \sim \pi(S')) \]
Intuitively this operator yields the expected state-action-value function
when looking \emph{one step ahead} following the policy $\pi$ and taking
expectation of $Q$.

We define the operator $T^\pi$ called the Bellman operator by
\[ (T^\pi Q)(s, a) = \E R(s, a) + \gamma (P^\pi Q)(s, a) \]
This operator adjust the $Q$ function to look more like $Q^\pi$ making one
"iteration" of "propagation of rewards" discounting with $\gamma$.
Indeed it is easily seen that $Q^\pi$ is a fixed point for $T^\pi$.

The \emph{greedy} policy $\pi$ with respect to a state-action value function
$Q$ is the one for which $\pi(s,a) = 1$ when $a = \argmax_a Q(s,a)$
and 0 otherwise.
\[ (T Q)(s, a) = T^{\pi_Q} Q \]
called the Bellman optimality operator.

The Bellman optimality equation is says that $Q^* = TQ^*$.

\subsection{Artificial Neural Networks}

\begin{Definition}\label{def_ANN}
	An \textbf{ANN} (Artificial Neural Network) with structure
	$\{ d_i \}_{i=0}^{L+1} \subseteq \N$,
	activation functions $\sigma_i = (\sigma_{ij} : \R \to \R)_{j=1}^{d_i}$
	and weights $\{ W_i \in M^{d_i \times d_{i-1}}, v_i \in \R^{d_i} \}_{i=1}^{L+1}$
	is the function $F:\R^{d_0} \to \R^{d_{L+1}}$ 
	\[ F(x) = w_{L+1} \circ \sigma_L \circ w_L \circ \sigma_{L-1} \circ \dots \circ w_1 x \]
	where $w_i$ is the affine function $x \mapsto W_i x + v_i$ for all $i$.

	Here $\sigma_i(x_1, \dots, x_{d_i})
	= (\sigma_{i1}(x_1), \dots, \sigma_{id_{i}}(x_{d_{i}}))$.

	$L \in \N_0$ is called the number of hidden layers.

	$d_i$ is the number of neurons or nodes in layer $i$.
\end{Definition}

An ANN is called \emph{deep} if there are two or more hidden layers.

\subsection{Fitted Q-Iteration}

We here present the algorithm which everything in this paper revolves around:

\begin{algorithm}[H]
	\caption{Fitted Q-Iteration Algorithm}
	\KwIn{MDP $(\Cal{S}, \Cal{A}, P, R, \gamma)$, function class $\Cal{F}$,
		sampling distribution $\nu$, number of iterations $K$,
		number of samples $n$, initial estimator $\widetilde{Q}_0$}
	\For{$k = 0,1,2,\dots,K-1$}{
		Sample i.i.d. observations $\{(S_i, A_i), i \in [n]\}$ from $\nu$
		obtain $R_i \sim R(S_i, A_i)$ and $S'_i \sim P(S_i, A_i)$ \\
		Let $Y_i = R_i + \gamma \cdot \max_{a \in \Cal{A}} \widetilde{Q}_k(S'_i, a)$ \\
		Update action-value function:
		\[ \widetilde{Q}_{k+1} \leftarrow
			\argmin_{f \in \Cal{F}} \frac{1}{n}
			\sum_{i=1}^n (Y_i - f(S_i, A_i))^2 \]
	}
	Define $\pi_K$ as the greedy policy w.r.t. $\widetilde{Q}_K$ \\
	\KwOut{An estimator $\widetilde{Q}_K$ of $Q^*$ and policy $\pi_K$}
\end{algorithm}

\subsection{Assumption 1: Holder Smoothness} %todo spell Holder properly
\begin{Definition}
	Let $\Cal{D}\subseteq \R^r$ be compact and $\beta,H>0$. A function $f:\Cal{D}\to \R$
	we call Holder smooth if
	\[ \sum_{{\alpha} : |{\alpha}| < \beta}
		\norm{\partial^{\alpha}f}_\infty +
		\sum_{{\alpha} : \norm{{\alpha}}_1 = \floor{\beta}}
		\sup_{x \neq y} \frac{|\partial^\alpha (f(x) - f(y))|}
		{\norm{x-y}_\infty^{\beta-\floor{\beta}}} \leq H \] 
	Where $\alpha = (\alpha_1, \dots, \alpha_r) \in \N^r$.
	We write $f \in C_r(\Cal{D}, \beta, H)$.
\end{Definition}

\begin{Definition}
	We consider families of \emph{Compositions of Holder Functions}
	\[ \Cal{G}(\{p_j, t_j, \beta_j, H_j\}_{j \in [q]}) \]
	where $t_j, p_j \in \N$, $t_j\leq p_j$ and $H_j, \beta_j > 0$,
	defined as containing $f$ when $f = g_q \circ \dots \circ g_1$
	for $g_j : [a_j, b_j]^{p_j} \to [a_{j+1}, b_{j+1}]^{p_{j+1}}$
	functions on some real hypercubes that only depend on $t_j$ of their inputs
	for each of their components $g_{jk}$,
	and satisfies $g_{jk} \in C_{t_j}([a_j, b_j]^t_j, \beta_j, H_j)$.
\end{Definition}

\begin{Assumption}\label{asm:A1}
	Let
	\[ \Cal{G}_0 = \{ f : \Cal{S} \times \Cal{A} \to \R : f(\cdot, a) \in
		\Cal{G}(\{p_j, t_j, \beta_j, H_j \}_{j \in [q]} ), \forall a\in \Cal{A} \} \]
	
	It is assumed that $T f \in \Cal{G}_0$ for any $f \in \Cal{F}_0$.

	I.e. when using the Bellman optimality operator on our sparse ReLU networks,
	we should stay in the class of compositions of Holder smooth functions.
\end{Assumption}

\subsection{Assumption 2: Concentration Coefficients}
\begin{Assumption}\label{asm:A2}
	Let $\nu_1, \nu_2 \in \Cal{P}(\Cal{S}\times \Cal{A})$ be probability measures,
	Lebesgue- absolutely continuous in $\Cal{S}$
	Define
	\[ \kappa(m, \nu_1, \nu_2) = \sup_{\pi_1, \dots, \pi_m}
		\left[ \E_{v_2} \left( \frac{\mathrm{d} (P^{\pi_m} \dots P^{\pi_1} \nu_1)}
		{\mathrm{d} \nu_2} \right)^2 \right]^{1/2} \]
	Let $\nu$ be the sampling distribution from the algorithm, and $mu$ the distribution
	over which we measure the error in the main theorem, then we assume
	\[ (1 - \gamma)^2 \sum_{m\geq 1} \gamma^{m-1} m \kappa(m, \mu, \nu)
		= \phi_{\mu, \nu} < \infty \]
\end{Assumption}

\subsection{The Main Theorem}
\begin{Theorem}[Yang, Xie, Wang]
	For any $K \in \N$ let $Q^{\pi_K}$ be the action-value function
	corresponding to policy $\pi_K$ which is return by Algorithm 1,
	when run with a sparse ReLU network on the form
	\[ \Cal{F}_0 = \{f(\cdot, a) \in \Cal{F}(L^*, \{d_j^*\}_{j=0}^{L^*+1},s^*)
		\mid a \in \Cal{A} \} \]
	where
	\[ L^* \lesssim (\log n)^{\xi'}, d_0 = r, d_j^*, d_{L+1}=1, \lesssim n^{\xi'},
		s^* \asymp n^{\alpha^*} \cdot (\log n)^{\xi'} \]
	Let $\mu$ be any distribution over $\Cal{S} \times \Cal{A}$.
	Under assumptions \cref{asm:A1, asm:A2} 
	\resizebox{\textwidth}{!}{ 
		$ \norm{Q^* - Q^{\pi_K}}_{1,\mu} \leq C \cdot \frac{\phi_{\mu,\nu}
			\cdot \gamma}{(1-\gamma)^2}
			\cdot |\Cal{A}| \cdot (\log n)^{\xi^*} \cdot n^{(\alpha^* - 1)/2} 
			+ \frac{4 \gamma^{K+1}}{(1-\gamma)^2} \cdot R_{\max} $ }
	Here $\alpha^* \in (0,1), C, \xi', \xi^*, \phi_{\mu,\nu} \in \R_{+}$
	are constants depending on the assumptions
	and $R_{\max}$ the maximum possible reward.
\end{Theorem}

 %old intro
In this thesis we give an introduction to Q-learning and discuss
convergence results of \emph{Q-learning} algorithms from its beginning in
1989 \ncite{W89} to a result obtained in the preprint \mcite{F20}.
The introduction includes fundamental theory of the underlying field of
\emph{dynamic programming} and \emph{reinforcement learning} (RL)
and related topics such as \emph{value iteration}.

\section{Motivation}
The topic was inspired by the performance of the algorithms
implemented by \mcite{M15}.
In \ncite{M15} it was shown how a single algorithm was able to achieve super
human performances in a variety of problems, namely playing Atari 2600 video
games, only using raw pixels and a reward (score) as input and a large number of
interactions with the environment.
The algorithm used in \ncite{M15} is based on a Q-learning
algorithm called the \emph{deep Q-network} (DQN) algorithm.

The purpose of this thesis is
to investigate what has been proven
about the convergence of the DQN algorithm
and what mathematical theory is relevant to establish such proofs.
This is however not an easy task, as we will only cover a small fraction
of the topic.

The work began by considering the preprint
\mcite{F20} which claims to establish theoretical justification for
the convergence of DQN to \emph{the optimal Q-function}.
In the course of reading \ncite{F20} it became clear that the background
context of dynamic programming, reinforcement learning and value iteration
was essential to understand the results of \ncite{F20} and similar papers
and also compare such results.
Also questions as to in which settings optimal policies exists turned out
to be a non-trivial question.
Therefore in the end this thesis is partly about presenting the results of
\ncite{F20} and similar papers.
And partly to build the background theory necessary to understand and
compare these results.

This provides an introduction to the field and sheds light 
on the original question of what can be said about convergence of
RL algorithms. Finally we will discuss some of the many questions that
still remain.

\section{What is Reinforcement Learning?}

RL is a broad topic and a main branch of
\emph{machine learning} alongside \emph{supervised} and \emph{unsupervised
learning}. Because of its broadness it overlaps with other disciplines
such as \emph{control theory} and dynamic programming.
To understand RL, we will now briefly describe its roots in dynamic programming.

In Reinforcement Learning, as in dynamic programming,
we are concerned with finding an optimal policy
for an agent in some environment.
This environment is described by
a sequence of state and action spaces
$\Cal{S}_1, \Cal{A}_1, \Cal{S}_2, \dots$
and rules (or dynamics) formalized as probability kernels
$P_1, R_1, P_2, \dots$ specifying which states and rewards
and likely to follow after some action is chosen.
One can then specify rules $\pi$, called a \emph{policy},
for how the agent should choose actions in every situation is the environment.
Given an environment and a policy one obtains stochastic process,
that is, a distribution on sequences of states, actions and
rewards.
One can then measure the performance of the policy by looking at
the expected sum of rewards called the \emph{value function}
$V_\pi$ of the policy.
The goal of reinforcement learning is to find an optimal policy $\pi^*$,
maximizing the value function.

$V_\pi$ is viewed as function that evaluates for each \emph{starting state}
$s \in \cl{S}_1$ the expected total rewards when starting in state $s$
and following policy $\pi$.
There might therefore be different optimal policies for each such starting
state.
Traditionally one defines an optimal value function $V^*(s)$
by taking supremum over all policies $\sup_\pi V_\pi(s)$ for every state
$s \in \cl{S}_1$.
Then an optimal policy $\pi^*$ should satisfy $V_{\pi^*} = V^*$,
i.e. it should be optimal uniformly across all starting states $\cl{S}_1$.
The existence of optimal policies defined in this way is a non-trivial
question and we will devote some time on this.

A particular class of environments which are called Markov decision processes
(MDPs).
In an MDP the same state space $\cl{S}$, action space $\cl{A}$ and rules
$P, R$ are used throughout the process.
They are by far the most well-studied environments.
With an MDP and a value function $V_1$ satisfying certain assumptions 
one can obtain a policy $\pi_1$ by choosing actions
leading to the maximum average values (according to $V_1$).
Such policies are called \emph{greedy policies}.
We can then evaluate value of $\pi_1$ yielding a new value function $V_2$.
This leads to an operator on the space of value functions called the
\emph{Bellman operator} and should satisfy $TV_1 = V_2$.
This process of applying the Bellman operator
can be continued indefinitely yielding a sequence of value
functions and policies.
Variations of this idea are called \emph{value iteration} and
\emph{policy iteration},
and is derived from dynamic programming.
We show that value iteration converges to the optimal value functions
given mild assumptions on the MDP.
Furthermore we show that the optimal value functions is a fixed point
of the Bellman optimality operator: $TV^* = V^*$
This is called the \emph{Bellman optimality equation} and
is central to all problems in dynamic programming.

We have now described RL as dynamic programming.
To go beyond this, RL commonly refers to algorithms that
are not merely value iterations, but instead work without
directly using the transition and reward dynamics,
and instead estimate value functions based only on sampling from the
environment.
Such algorithms are called \emph{model-free} as opposed to
\emph{model-based} algorithms, that employ knowledge about the
environment such as its transition distributions directly
(such as pure dynamic programming).

A further categorization of RL-algorithms can be made into
\emph{off-policy} and \emph{on-policy} classes.
This is simply whether the algorithm learns from data
(states, actions and rewards) arising from 
following its own policy (on-policy) or it can learn from more
arbitrary data (off-policy).
This \emph{more arbitrary data} could for example be
the trajectory of another algorithm when interacting with a decision process,
or simply state-action-reward pairs drawn from some distribution.
In this thesis we will put emphasis on off-policy algorithms,
to which both FQI and DQN belong.

\section{What is Q-learning?}

A problem with value functions defined on the set of states $\cl{S}$ is that
picking optimal actions require knowledge of the transition dynamics $P$.
This is especially a problem for model-free algorithms.
To get around this problem \emph{Q-functions} were introduced, which evaluates
the value of a state-action pair, instead of only a state.

Given a Q-function $Q$, picking best actions according to $Q$ now
merely require maximization over $Q$ itself.
Also it turns out that Q-functions is
more convenient to work with computationally.
In this thesis we show that value and policy iteration can be done
for Q-functions in a virtually identical manner, when the process dynamics
are known.

When the process dynamics are hidden designing algorithms becomes trickier.
In such settings approaches to the problem
fall in two categories. In the \emph{indirect} approaches
one attempts to estimate the process dynamics first and then afterwards
methods for the known-dynamics are applied.
The \emph{direct} approaches basically covers \emph{the rest}.
In the direct category we find the popular \emph{temporal difference}
algorithms on which \emph{fitted Q-iteration} (FQI)
and the \emph{deep Q-network} (DQN) algorithm of \ncite{M15} is based.
Many direct approaching such as FQI and DQN can be seen as
stochastic approximations of the Bellman optimality equation.

\emph{Q-learning} is the category of algorithms that iteratively updates
Q-functions in the attempt to improve the derived policy.
\emph{Deep} Q-learning is then the subcategory of algorithms which
uses deep neural networks as approximators for the Q-functions.

\section{Basic concepts and notation}

The real numbers $\R$ is endowed with
the standard ordering with
giving rise to the
standard order topolog
(\cref{defn:orderTop}).
This in turn give rise to the standard Borel $\sigma$-algebra
(\cref{defn:BorelAlg}) $\bb{B} = \sigma(\cl{O})$
generated by the open sets $\cl{O}$ of the standard topology on $\R$.

When considering a measurable space $\cl{X}$ 
we always denote its $\sigma$-algebra
$\Sigma_\cl{X}$ when not ambiguous.
We always
consider the cartesian product of measurable spaces
with the product $\sigma$-algebra (\cref{defn:prodSigmaAlg})
unless otherwise specified.
We denote the set of measurable functions (\cref{defn:measFunc})
$\cl{X} \to \cl{Y}$ between two measurable spaces by 
$\cl{M}(\Sigma_\cl{X}, \Sigma_\cl{Y})$ or $\cl{M}(\cl{X}, \cl{Y})$
when the $\sigma$-algebras are not ambiguous
or simply $\cl{M}(\cl{X})$ when $\cl{Y} = \R$.

The set of probability measures on $\cl{X}$ is denoted
$\cl{P}(\Sigma_\cl{X})$ or $\cl{P}(\cl{X})$ when $\Sigma_\cl{X}$ is implicit
(not to be confused with the powerset of $\cl{X}$
which we denote $2^{\cl{X}}$).

An $\cl{X}$-valued random variable $X : \Omega \to \cl{X}$ is a
measurable function from \emph{the background probability space}
measure space $(\Omega, \Sigma_\Omega, \Prob)$ into some measurable
space $\cl{X}$.
By abstract change of variable the distribution of the random variable $X$
is the image probability measure $\mu = X(\Prob)$ and we write
$X \sim \mu$.

When talking about functions $f_1, f_2, \dots : \cl{X} \to \R$
limits are always understood pointwise, unless otherwise stated,
meaning that $f_n \to f$ is to be read as
$\forall x \in \cl{X} : f_n(x) \to f(x)$.
The same goes for logical operators, e.g. $f > 0$ is to be understood
as $f(x) > 0$ for all $x \in \cl{X}$.





\subsection{Reinforcement learning in general}

\subsubsection{Reinforcement learning}

In Reinforcement Learning (RL) we are concerned with finding an optimal policy
for an agent in some environment.
This environment is described by a so-called decision process
consisting of a sequence of state and action spaces
$\Cal{S}_1, \Cal{A}_1, \Cal{S}_2, \dots$
and rules $P_1, R_1, P_2, \dots$ specifying which states and rewards
and likely to follow after some action is chosen.
In the general case $\Cal{S}_i, \Cal{A}_i$ are measurable spaces
and $P_i, R_i$ are probability kernels on $S_{i+1}$ and $\Rext$, respectively.
One then attempts to find a \emph{policy} (behavior or strategy) that
maximizes the rewards returned from the environment.
A policy and an distribution over starting states
$\mu \in \Cal{P}(\Cal{S}_1)$
give rise to a countable stochastic process,
$ (X_i)_{i\in \N} = (S_i, A_i, R_i)_{i\in \N}$ 
that is a probability measure $P^\pi_\mu$. on
$\Cal{S}_1 \times \Cal{A}_1 \times \dots$.
Intuitively $S_1$ is drawn from $\mu$,
then for all $i \in \N$
$A_i$ is drawn from $\pi(\cdot \mid S_i)$,
a reward is then drawn from $R(\cdot \mid S_i, A_i)$,
then $S_{i+1}$ is drawn from $P(\cdot \mid S_i, A_i)$ and so on.


In RL there is a categorization of algorithms into
\emph{off-policy} and \emph{on-policy} classes.
This is simply whether the algorithm learns from data
(states, actions and rewards) arising from 
following its own policy (on-policy) or it can learn from more
arbitrary data (off-policy).
This \emph{more arbitrary data} could for example be
the trajectory of another algorithm when interacting with a decision process,
or simply state-action-reward pairs drawn from some distribution.
In this paper we exclusively consider off-policy algorithms.

\subsection{Measure theory}

\documentclass{article}

% preamble speciale Jacob Harder
% 31. jan. 2020

%packages

\usepackage[utf8]{inputenc} %utf8 is probably good
\usepackage{amsmath}
\usepackage{amssymb}
\usepackage{amsthm}
\usepackage{graphicx} %for including images
\usepackage{float} %for exact placement of figure (and more?)
\usepackage{mathtools} %for \mathclap (stacking under sums)
\usepackage{dsfont} %for boldface numbers
\usepackage{bm} %vectors in bold
\usepackage[ruled,vlined,linesnumbered]{algorithm2e}
\usepackage{cleveref}
\usepackage{ifthen}
\usepackage{commath}
\usepackage[a4paper,width=150mm,top=25mm,bottom=25mm]{geometry}
\usepackage{upgreek}
%\usepackage[inline]{enumitem}
\usepackage{multicol}
%\usepackage{fancyhdr} %maybe later..
%\pagestyle{fancy}
\usepackage{mathabx}
\usepackage{csquotes}
\usepackage{outline} %for subitems in lists
\usepackage[inline]{enumitem} % for e.g. horizontal enumerate
\usepackage{wrapfig}

\usepackage{tikz}
\usetikzlibrary{calc,trees,positioning,arrows,chains,shapes.geometric,%
    decorations.pathreplacing,decorations.pathmorphing,shapes,%
    matrix,shapes.symbols}

%front page
\usepackage{wallpaper}
\usepackage{titling}

%bibliography
\usepackage[numbers]{natbib}
\bibliographystyle{plainnat}
\newcommand{\mcite}[1]{[\citenum{#1}, \citeauthor{#1} (\citeyear{#1})]}
\newcommand{\ncite}[1]{\cite{#1}}

%linespread and geometry
\linespread{1.3}

%commands
\newcommand{\Cal}{\mathcal} % to be deleted
\newcommand{\cl}{\mathcal}
\newcommand{\fk}{\mathfrak}
\newcommand{\bb}{\mathbb}
\newcommand{\hrm}{\mathrm}
\newcommand{\Q}{\bb{Q}}
\newcommand{\Z}{\bb{Z}}
\newcommand{\N}{\bb{N}}
\newcommand{\R}{\bb{R}}
\newcommand{\C}{\bb{C}}
\newcommand{\E}{\bb{E}}
\newcommand{\Rext}{\ol{\R}}
\newcommand{\Var}{\mathrm{Var}}
\newcommand{\Prob}{\mathds{P}} %fundamental probability measure
\newcommand{\idc}{\mathds{1}}
\newcommand{\ve}{\varepsilon} %abbreviation for epsilon
\newcommand{\Yp}{\Upupsilon} %abbreviation for Ypsilon
\newcommand{\difd}{\; \mathrm{d}} %differential d
\newcommand{\wt}{\widetilde}
\newcommand{\wh}{\widehat}
\newcommand{\ol}{\overline}
\newcommand{\ul}{\underline}
\newcommand{\Mid}{\;\middle\vert\;}
\newcommand{\id}{\text{id}}
\newcommand{\supp}{\text{supp}}
\newcommand{\defemph}[1]{\textbf{#1}} %first-mentions of names
\DeclarePairedDelimiter\ceil{\lceil}{\rceil}
\DeclarePairedDelimiter\floor{\lfloor}{\rfloor}
\DeclareMathOperator*{\argmax}{argmax}
\DeclareMathOperator*{\argmin}{argmin}
\newcommand{\defeq}{\vcentcolon=} %definition equality symbol
%add single eq. tag in align*
\newcommand\numberthis{\addtocounter{equation}{1}\tag{\theequation}}
\newcommand{\rleft}[1]{\rotatebox[origin=c]{90}{\ensuremath{#1}}}
\newcommand{\vrel}[3]{ % for vertical subseteq e.g.
\vcenter{\halign{\hfill##\hfill\cr
\ensuremath{#1}\cr
\rotatebox[origin=c]{270}{\ensuremath{#2}}\cr
\ensuremath{#3}\cr
}}}
\newcommand{\lar}{\leftrightarrow}
\newcommand{\Span}{\mathrm{span}}
\newcommand{\Gr}{\mathrm{Gr}}

%theorems
\theoremstyle{definition}
\newtheorem{thm}{Theorem}[chapter]
\newtheorem{lem}[thm]{Lemma}
\newtheorem{defn}[thm]{Definition}
\newtheorem{cor}[thm]{Corollary}
\newtheorem{rem}[thm]{Remark}
\newtheorem{prop}[thm]{Proposition}
\newtheorem{asm}{Assumption}
\newtheorem{example}[thm]{Example}
%\newtheorem{cond}{Condition}
\newtheorem{sett}{Setting}
\newtheorem{innercond}{Condition}
\newenvironment{cond}[1]
  {\renewcommand\theinnercond{#1}\innercond}
  {\endinnercond}
%cref
\crefname{algocf}{alg.}{algs.}
\Crefname{algocf}{Algorithm}{Algorithms}
\crefname{innercond}{}{}
\Crefname{innercond}{}{}

%allow page breaks in align
\allowdisplaybreaks



\begin{document}


\end{document}


\section{Decision models and value functions}

In this section we will develop general theory about
decision processes and value function (including Q-functions)
that is used across all sources considered in this paper.
We will also take up the question of optimal policy existence and 
prove this in different settings for reference
in the latter sections.

\subsection{History dependent decision process}
We define in this section a quite general framework.
We do this partly in the quest to have a united framework
to talk about results from a variety of sources,
and relate them to each other in generality.
And partly to avoid defining various concepts such as value functions
everytime a new context is considered.
A source which uses a setup which is almost as general can be found in
[ref. to Schal].
In this section recall that $\ul{\R} = \R \cup \{-\infty\}$,
$\ol{\R} = \R \cup \{\infty\}$ and
$\ol{\ul{\R}} = \R \cup \{\pm \infty\}$.

\begin{defn}[History dependent decision process]
  A \defemph{history dependent decision process} (HDP) is determined by
  \begin{enumerate}
    \item $(\Cal{S}_n, \Sigma_{\Cal{S}_n})_{n \in \N}$ a 
      measurable space of \defemph{states} for each timestep.
    \item $(\Cal{A}_n, \Sigma_{\Cal{A}_n})_{n \in \N}$ a 
      measurable space of \defemph{actions} for each timestep.
  \end{enumerate}
  for each $n \in \N$ we define the so called \defemph{history} spaces
  \[ \Cal{H}_1 = \Cal{S}_1, \quad
    \Cal{H}_2 = \Cal{S}_1\times \Cal{A}_1\times \Cal{S}_2,
    \quad \Cal{H}_3 = \Cal{S}_1 \times \Cal{A}_1 \times \Cal{S}_2 \times
  \Rext \times \Cal{A}_2 \times \Cal{S}_3 \]
  \[ \Cal{H}_n = \Cal{S}_1 \times \Cal{A}_1
    \times \Cal{S}_2 \times \ol{\ul{\R}} \times \Cal{A}_2
  \times \Cal{S}_3 \times \ol{\ul{\R}}\times \dots \times \Cal{S}_n \]
  \[
    \Cal{H}_\infty = \Cal{S}_1 \times \Cal{A}_1 \times \Cal{S}_2 \times
    \ol{\ul{\R}} \times \dots
  \]
  with associated product $\sigma$-algebras
  \begin{enumerate} \setcounter{enumi}{2}
    \item $(P_n)_{n \in \N}$ a sequence of
      $\Cal{H}_n \times \Cal{A}_n \leadsto \Cal{S}_{n+1}$ kernels
      called the \defemph{transition} kernels.
    \item $(R_n)_{n \in \N}$ a sequence of
      $\Cal{H}_{n+1} \leadsto \ol{\ul{\R}}$ kernels
      called the \defemph{reward} kernels.
  \end{enumerate}
  \label{sett:HDP}
\end{defn}
The name \emph{decision process} is used for many different processes
across litterature but many of them generalize to the above.
Some authors use the name \emph{dynamic progamming model} to refer to
such processes.
Notice the slight irregularity in the beginning of the history spaces:
We are missing a reward state after $\Cal{S}_1$. We could avoid
this by introducing some start reward we will do without.

\begin{asm}(Reward independence)
  $P_n, R_n$ and policies are only allowed to depend on the past
  states and actions, and not the rewards.
  \label{asm:rewardIndep}
\end{asm}

In all sources known to this writer \cref{asm:rewardIndep} is assumed.
This is a bit of a puzzle since it is obvious that one could
want to define algorithms (policies) that take into account which rewards
they received in the past.
We will also do this but stick to the standard and 
never attempt to evaluate ideal value functions of
policies that depend on rewards.
Thus we will let \cref{asm:rewardIndep} hold
from now on and throughout this paper.

The majority of sources considered in this paper also specialize
with the following:
\begin{asm}[One state and action space]
  $\Cal{S}_1 = \Cal{S}_2 = \dots \defeq \Cal{S}$
  $\Cal{A}_1 = \Cal{A}_2 = \dots \defeq \Cal{A}$
  \label{asm:oneStateActionSpace}
\end{asm}
We will do without this for the rest of this section in order to
present some results in the generality they deserve.
And later we will look at settings which do not specialize this way.
One could ask if it is possible to embed the general decision process into one
with \cref{asm:oneStateActionSpace} by setting
$\Cal{S} \defeq \bigcup_{i\in\N} \Cal{S}_i$ and
$\Cal{A} \defeq \bigcup_{i\in\N} \Cal{A}_i$ or similar.
One attempt at this can be found in [BS SOC, chp. 10],
but this will not be covered here. %consider covering it

Other ways to specialize include reducing one or both of the
transition and reward kernels to functions defined on
$\Cal{S} \times \Cal{A}$. These processes are often called
\emph{deterministic}, but the exact definitions vary across sources, and we will
instead specify each setting individually.

For a decision process we can define
\begin{defn}[Policy]
  A (randomized) \defemph{policy} $\pi = (\pi_n)_{n \in \N}$
  is a sequence of $\Cal{H}_n \leadsto \Cal{A}_n$ kernels.
  The set of all policies we denote $R\Pi$.
  The policy $\pi$ is called \defemph{semi Markov} if each $\pi_i$ only depends
  on the first and last state in the history
  and is called \defemph{Markov} if only the last.
  The sets are denoted $sM\Pi$ and $M\Pi$.
  Furthermore $\pi$ is called \defemph{deterministic} if all $\pi_i$
  are degenerate, i.e. for all $i$ we have
  $\pi_i(\{a_i\} \mid h_i) = 1$ for some $a_i \in \Cal{A}_{i}$.
  Under \cref{asm:oneStateActionSpace}
  it makes sense to make a (Markov) policy $(\pi, \pi, \dots)$,
  where $\pi$ only depends on the last state.
  Such a policy is called \defemph{stationary},
  and the set of them denoted $S\Pi$.
  We denote the deterministic version of the policy classes
  by the letter $D$.
\end{defn}
We have the following inclusions
\[ \begin{matrix}
  S\Pi &\subseteq M\Pi &\subseteq sM\Pi &\subseteq R\Pi
  \\ \rleft{\subseteq} & \rleft{\subseteq} & \rleft{\subseteq} & \rleft{\subseteq} 
  \\ DS\Pi &\subseteq DM\Pi &\subseteq DsM\Pi &\subseteq D\Pi
\end{matrix} \] 

\begin{prop}
A dynamic progamming model together with a policy $\pi$ defines a
probability kernel $\kappa_\pi : \Cal{S}_1 \to \Cal{H}_\infty$.
\end{prop}
\begin{proof}
  This is the Ionescu-Tulcea kernel generated by
  $\dots R_2 P_2 \pi_2 R_1 P_1 \pi_1$.
\end{proof}
This kernel yields a probability measure $\kappa_\pi \mu$ on $\Cal{H}_\infty$
for every $\mu \in \Cal{S}_1$. In particular for any $s \in \Cal{S}_1$
$\kappa_\pi \delta_s$ yields the measure $\kappa(\cdot \mid s)$
and we shall occasionally write this $\kappa_\pi s$ and
integration with respect to it $\E^\pi_s$.

Across litterature generally %todo: backup this claim
any function mapping a state space $\Cal{S}$ to $\ol{\ul{\R}}$
can be called a (state)
\defemph{value} function. Similarly any $\ul{\ol{\R}}$ valued function
on pairs of states and actions can be called (state)
\defemph{action value} or \defemph{Q}- function.
The idea behind such functions are commonly to estimate the
cumulative rewards associated with a state or state-action pair
and the trajectory of states it can lead to.
In order to define some standard value functions
we will need one of
the following conditions:

\begin{cond}{$F^-$}[Reward finity from above]
  $\int_{[0,\infty]} x \difd R_i(x \mid h) < \infty$ for all
  $h \in \Cal{H}_{i+1}$ and $i \in \N$
  \label{cond:F-}
\end{cond}
\begin{cond}{$F^+$}[Reward finity from below]
  $\int_{[-\infty,0]} x \difd R_i(x \mid h) > -\infty$ for all
  $h \in \Cal{H}_{i+1}$ and $i \in \N$
  \label{cond:F+}
\end{cond}
The letter \emph{F} comes from [todo ref Bertekas SOC].
When assuming either of (\cref{cond:F+}) or (\cref{cond:F-})
we ensure that the summation of finitely many rewards
has a well defined mean in $\Rext$,
and then the following definition makes sense 
\begin{defn}[Finite horizon value function]
  Let $\ul{R}_i : \Cal{H}_\infty \to \ol{\ul{\R}}$ be the projection onto the
  $i$th reward. Define
  \[ V_{n,\pi}(s) = \E_s^\pi \sum_{i=1}^n \ul{R}_i \]
  called the $k$th finite horizon value function.
  When $n=0$ $\forall \pi : V_{0,\pi} = V_0 \defeq 0$.
\end{defn}
The finite horizon value function
measures the expected total reward of starting in state $s$
and then follow the policy $\pi$ for $n$ steps.
This way it measures the \emph{value} of that particular state
given a policy and \emph{horizon} (number of steps).
We would like to extend this to an infinite horizon value function,
i.e. letting $n$ tend to $\infty$. To ensure that the integral is well-defined
we need one of the following conditions
\begin{cond}{P}[Reward non-negativity] $R_i([0,\infty] \mid h) = 1$,
  $\forall h \in \Cal{H}_{i+1}, i \in \N$
  \label{cond:P}
\end{cond}
\begin{cond}{N}[Reward non-positivity] $R_i([-\infty, 0] \mid h) = 1$
  $\forall h \in \Cal{H}_{i+1}, i \in \N$
  \label{cond:N} 
\end{cond}
\begin{cond}{D}[Discounting] There exist a bound $R_{\max} > 0$ and a
  $\gamma \in [0,1)$ called the \defemph{discount} factor such that
  $R_i([-R_{\max} \gamma^i, R_{\max} \gamma^i]) = 1$
  $\forall h \in \Cal{H}_{i+1}, i \in \N$
  \label{cond:D}
\end{cond}
Again the letters P, N and D are adopted from [todo ref Bertsekas SOC].
\begin{defn}
  We define the infinite horizon value function by
  \[ V_\pi(s) = \E_s^\pi \lim_{n \to \infty} \sum_{i=1}^n \ul{R}_i \]
\end{defn}
The infinite horizon value function $V_\pi$ measures the expected total
reward after following the policy $\pi$ an infinite number of steps.

\begin{rem}
  Whenever we are working with the finite horizon value function
  we will always assume that either ($F^+$) or ($F^-$) holds without
  stating this explicitly.
  If a result only holds under e.g. ($F^+$) we will of course be explicit
  about this be marking it accordingly with a ($F^+$).

  Similarly whenever we work with the infinite horizon value function we will
  always assume that at least one of (P), (N) or (D) holds.
  We will mark propositions and theorems
  by e.g. (\cref{cond:D}) (\cref{cond:P}) when
  the result only holds for if discounting \emph{or} reward non-negativity
  is assumed.
  Note that obviously (P) implies $F^+$ and (N) implies $F^-$.
\end{rem}

\begin{rem}
Since we are under \cref{asm:rewardIndep}, when talking about the finite
or infinite value functions,
we can actually reduce the reward kernels to functions
$r_i : \Cal{H}_{i+1} \to \Rext = h \mapsto \int r \difd R_i(r \mid h)$
(note that $r_i$ is measurable due to \cref{prop:intKerMeas}).
Another way of stating this is that the value functions are indifferent
to whether we use deterministic or stochastic rewards.
This however does not mean that we can dispose completely of stochastic
rewards, as they still make a difference to model-free algorithms that
do not know the reward kernel, and therefore cannot simply integrate it.
\end{rem}

For use later we mention some properties of these value functions.
\begin{prop} 
  When well-defined the value functions $V_{n,\pi}, V_\pi$ are measurable
  into $(\ul{\ol{\R}}, \ul{\ol{\bb{B}}})$.
\end{prop}
\begin{proof}  
  Use \cref{prop:intKerMeas}.
\end{proof}

\begin{prop}
  $\lim_{n\to\infty} V_{n, \pi} = V_\pi $
  for all $\pi \in R\Pi$.
  \label{prop:VnLimV}
\end{prop}
\begin{proof}
  By monotone or dominated convergence.
\end{proof}

\begin{prop} Under (D) for any $\pi \in R\Pi$ we have

  $\abs{V_{n,\pi}}, \abs{V_\pi} \leq R_{\max} (1 - \gamma) < \infty$.
  \label{prop:Vbounded}
\end{prop}
\begin{proof}
  For any $\pi \in R\Pi$
  \[ \abs{V_\pi(s)} \leq \E_s^\pi \sum_{i \in \N} \abs{\ul{R}_i}
    \leq \sum_{i \in \N} \gamma^{i-1} R_{\max}
  = R_{\max} / (1-\gamma) \]
  This also covers $V_{n, \pi}$.
\end{proof}
As this bound will occur again and again we denote it
\[ V_{\max} \defeq R_{\max}(1-\gamma) \]

\subsubsection{Optimal policies}

Let $(\Cal{S}_n, \Cal{A}_n, P_n, R_n)_{n \in \N}$ be a decision process.

\begin{defn}[Optimal value functions] 
  \begin{align*}
    V_n^*(s) \defeq & \; \sup_{\pi \in R\Pi} V_{n,\pi}(s) &
    V^*(s) \defeq & \; \sup_{\pi \in R\Pi} V_\pi(s)
  \end{align*}
  This is called the \defemph{optimal value function} (and the $n$th
  optimal value function).
  A policy $\pi^* \in R\Pi$ for which $V_{\pi^*} = V^*$ is called an
  \defemph{optimal policy}.
  If $V_{n, \pi^*} = V^*_n$ it is called $n$-optimal.
  \label{defn:optimalValue}
\end{defn}

\begin{prop} (D)

  $\abs{V^*_k},\; \abs{V^*} \leq V_{\max}$.
\end{prop}
\begin{proof}
  By \cref{prop:Vbounded} all terms in the suprema are within this bound.
\end{proof}

\begin{rem}
An interesting fact about the optimal value functions is that they
might not be Borel measurable [todo ref to counterexample]
even in the finite case.
After all we are taking a supremum over
sets of policies which have cardinality of at least the continuum.
However it is sometimes possible to show that they are.
We will take these discussions as they occur in various settings.
\end{rem}

At this point some central questions can be asked.
\begin{enumerate}
  \item To which extend does an optimal policy $\pi^*$ exist?
  \item Does $V_n^*$ converge to $V^*$?
  \item When can optimal policies be chosen to be Markov, deterministic, etc.?
  \item Can an algorithm be designed to efficiently find $V^*$ and
    $\pi^*$?
\end{enumerate}
These questions has been answered in a variety of settings.
We will try to address them in order by strength of assumptions
they require.

\subsubsection{Schäls theorem}
In a quite general setting, questions 1 and 2
was investigated by M. Schäl in 1974
[todo ref. to On Dynamic Programming:
Compactness of the space of policies, 1974].
Here some additional structure on our process is imposed:
\begin{sett}[Schäl]
  \begin{enumerate}
    \item $V_\pi < \infty$ for all policies $\pi \in R\Pi$.
    \item $(\Cal{S}_n, \Sigma_{\Cal{S}_n})$ is assumed to be standard Borel.
      I.e. $\Cal{S}_n$ is a non-empty Borel subset of a Polish space
      and $\Sigma_{\Cal{S}_n}$ is the Borel subsets of $S_n$.
    \item $(\Cal{A}_n, \Sigma_{\Cal{A}_n})$ is similarly assumed to be
      standard Borel.
    \item $\Cal{A}_n$ is compact.
    \item $\forall s \in \Cal{S}_1 :
      Z_n = \sup_{N \geq n} \sup_{\pi \in R\Pi} \sum_{t=n+1}^N
      \E_s^\pi r_n \to 0$ as $n \to \infty$.
  \end{enumerate}
  \label{sett:Schal}
\end{sett}

In this setting Schäl introduced two set of criteria for the existence
of an optimal policy:

\begin{cond}{S}
  \begin{enumerate}
    \item The function \[
	(a_1, a_2, \dots, a_n) \mapsto
	P_n(\cdot \mid s_1, a_1, s_2, a_2, \dots, s_n, a_n)
      \]
      is set-wise continuous (hence the name \defemph{S})
      for all $s_1, \dots, s_n \in \Cal{S}^{\ul{n}}$.
    \item $r_n$ is upper semi-continuous.
  \end{enumerate}
  \label{cond:S}
\end{cond}

\begin{cond}{W}
  \begin{enumerate}
    \item The function
      \[(h_n, a_n) \mapsto P_n(\cdot \mid h_n, a_n)\]
	is weakly continuous (hence the name \defemph{W}).
    \item $r_n$ is continuous.
  \end{enumerate}
  \label{cond:W}
\end{cond}

\begin{thm}[Schäl]
  When either (\cref{cond:S}) or (\cref{cond:W}) hold then
  \begin{enumerate}
    \item There exist an optimal policy $\pi^* \in R\Pi$.
    \item $V^*_n \to V^*$ as $n \to \infty$.
  \end{enumerate}
  \label{thm:SchalExi}
\end{thm}
\begin{proof}
  We refer to [todo ref: On Dynamic Programming: Compactness of the space of
  policies, M. Schäl 1974]. %todo: or do we?
\end{proof}

Schäls theorem tells us that optimal policies exist in a wide class
of decision processes. However in many cases we are looking at processes
in which the next state in independent of the history.
In such cases it makes sense to ask if optimal policies can be chosen
within the system of policy subclasses.
Such questions will be addressed in the next section.

\subsection{The Markov decision process and its operators}

\begin{defn}[Markov decision process]
  A \defemph{Markov decision process} (MDP) consists of
  \begin{enumerate}
    \item $(\Cal{S}, \Sigma_{\Cal{S}})$ a 
      measurable space of states.
    \item $(\Cal{A}, \Sigma_{\Cal{A}})$ a 
      measurable space of actions.
    \item $P : \Cal{S} \times \Cal{A} \leadsto \Cal{S}$
      a transition kernel.
    \item $R : \Cal{S} \times \Cal{A} \leadsto \ol{\ul{\R}}$
      a reward kernel.
    \item An optional disount factor $\gamma \in [0,1]$
      (when not discounting put $\gamma = 1$).
  \end{enumerate}
  \label{sett:MDP}
\end{defn}
This is a special case of the general decision process (\cref{sett:DP}) with
\begin{itemize}
  \item \Cref{asm:oneStateActionSpace} is satisfied i.e. 
    $\Cal{S}_1 = \Cal{S}_2 = \dots, \Cal{A}_1 = \dots = \Cal{A}$.
  \item $P_n$ depends only on $s_n$ and $a_n$ and does not
    differ with $n$. I.e. 
    $P_n(\cdot \mid s_1, \dots, s_n, a_n) = P(\cdot \mid s_n, a_n)$
    for all $n \in \N$.
  \item $R_n$ depends only on $s_n$ and $a_n$ and does not differ
    with $n$ except for a potential discount.
    I.e. $R = R_n/\gamma^{n-1}$ for all $n \in \N$
\end{itemize}
We will write $P$ instead of $P_n$ understanding
kernel compositions as if using $P_n$.

%todo write here how value functions in MDPs look under (D)

At this point it makes sense to define

\begin{defn}[The $T$-operators]
  For a stationary policy $\pi$ and measurable $V:\Cal{S} \to \ol{\ul{\R}}$
  with $V \geq 0$, $V \leq 0$ or $\abs{V} < \infty$
  we define the operators 
  \[ P_\pi V \defeq s \mapsto \int V(s') \difd P\pi(s' \mid s) \]
  \[ T_\pi V \defeq s \mapsto \int r(s, a)
  + \gamma V(s') \difd (P \pi)(a, s'\mid s) \]
  \[ T V \defeq s \mapsto \sup_{a \in \Cal{A}} T_a V(s) \]
  where $T_a = T_{\delta_a}$.
\end{defn}

We want to define

\begin{prop}[Properties of the $T$-operators]
  Let $\pi = (\pi_1, \pi_2, \dots)$ be a Markov policy.
  \begin{enumerate}
    \item The operators $P_\pi, T_\pi$ and $T$ commutes with limits.
    \item $V_{k, \pi} = T_{\pi_1} V_{k-1, (\pi_2, \dots)}
      = T_{\pi_1} \dots T_{\pi_k} V_0$.
    \item $V_\pi = \lim_{k \to \infty} T_{\pi_1} \dots T_{\pi_k} V_0$
    \item If $\pi$ is stationary $T_\pi V_\pi = V_\pi$.
    \item (D) $T$ and $T_\pi$ are $\gamma$-contractive
      on $\Cal{L}_\infty(\Cal{S})$.
    \item (D) $V_\pi$ is the unique bounded fixed point of $T_\pi$
      in $\Cal{L}_\infty(\Cal{S})$
  \end{enumerate} 
  \label{prop:propTV}
\end{prop}
\begin{proof}
  \leavevmode
  \begin{enumerate}
    \item By monotone or dominated convergence theorems.
      \label{commLimits}
    \item 
      \begin{align*}
	&T_{\pi_1} V_{k,(\pi_2, \dots)}(s_1)
	\\ &= \int r(s_1, a_1) + \gamma
	\int \sum_{i=2}^{k+1} \gamma^{i-2} r(s_i, a_i)
	\difd \kappa_{\pi_2, \dots} (a_2, s_3, a_3, \dots \mid s_2)
	\difd P \pi_1(a_1, s_2 \mid s_1)
	\\ &= \int \sum_{i=1}^{k+1} \gamma^{i-1} r(s_i, a_i)
	\difd \dots P \pi_2 P \pi_1 (a_1, s_2, \dots \mid s_1)
	\\ &= \int \sum_{i=1}^{k+1} \gamma^{i-1} r(s_i, a_i)
	\difd \kappa_\pi (a_1, s_2, \dots \mid s_1)
	\\ &= V_{k+1, \pi}(s_1)
      \end{align*}
      Now use this inductively.
    \item This is by 2. and \cref{prop:VnLimV}.
    \item By 3. $T_\pi V_\pi = T_\pi \lim_{k \to\infty} T_{\pi}^k V_0
      = \lim_{k \to\infty} T_\pi^{k+1} V_0 = V_\pi$.
    \item Let $V, V' \in \Cal{L}_\infty(\Cal{S})$
      and let $K = \norm{V - V'}_\infty$.
      Then since the rewards are bounded
      \[ \abs{T^\pi V - T^\pi V'}
	= \gamma \abs{\int V(s') - V'(s') \difd P\pi(s' \mid s)}
      \leq \gamma K \]
      For $T$ use the same argument and the fact that
      $\abs{\sup_x f(x) - \sup_{y} g(y)} \leq
      \abs{\sup_x f(x) - g(x)}$ for any $f,g : X \to \ul{\R}$.
    \item By 4., 5. and Banach fixed point theorem.
  \end{enumerate}
\end{proof}

\subsection{Q-functions}
\begin{defn}
  Let $\pi \in R\Pi$.
  Define
  \[ Q_{k, \pi}(s, a) = r(s, a) + \gamma \E_{P(\cdot \mid s, a)} V_{k, \pi}
  ,\qquad Q_\pi = r(s, a) + \gamma \E_{P(\cdot \mid s, a)} V_\pi \]
  \[ Q^*_k = \sup_{\pi \in R\Pi} Q_{k, \pi}
  , \qquad Q^* = \sup_{\pi \in R\Pi} Q_\pi \]
  Define $Q_0 = r$ then we make the convention that
  $Q^*_0 = Q_{0,\pi} = Q_0 = r$.
\end{defn}
The letter Q originates to a PhD thesis by C. Watkins from 1989
[todo ref C. Watkins, 1989]. Upon his definition he noted
\begin{displayquote}
  ``This is much simpler to calculate than [$V_\pi$]
  for to calculate [$Q_\pi$] it is only necessary to look one
  step ahead [\ldots]''
\end{displayquote}
A clear advantage of working with Q-function
$Q:\Cal{S}\times\Cal{A} \to \Rext$ rather than a value function
$V:\Cal{S}\to \Rext$,
is that finding the optimal action in state $s$
requires only a maximization over the Q-function itself:
$a = \argmax_{a \in \Cal{A}} Q(s,a)$.
This should be compared to finding a best action according to a value
function $V$:
$a = \argmax_{a \in \Cal{A}} r(s,a) + \gamma \E_{P(\cdot \mid s,a)} V$.
Besides being less simple,
this requires taking an expectation with respect to 
both the reward and transition kernel.
Later we will study settings where we are not allowed to know
the process kernels when attempting to find the optimal strategy.
In these situations the advantage of Q-functions is clear.
For now however the transition kernel will remain known and we
will in this section see how the results of state-value functions
translate to Q-functions.
The results in this section are original in the generality here presented,
as I was unable to find them elsewhere.

\begin{prop}
  Let $\pi = (\pi_1, \pi_2, \dots) \in R\Pi$ then
  $\lim_{k \to \infty} Q_{k, \pi} = Q_\pi$. Furthermore it holds that
  $\abs{Q_{k, \pi}}, \abs{Q_\pi}, \abs{Q^*_k}, \abs{Q^*} \leq V_{\max}$.
\end{prop}
\begin{proof}
  By dominated convergence or monotone convergence and \cref{prop:Vbounded}.
\end{proof}

In parallel to the operators for state-value functions we define
\begin{defn}[$T$ operators for Q-functions]
  For any stationary policy $\pi \in S\Pi$
  and measurable $Q:\Cal{S} \times \Cal{A} \to \ol{\ul{\R}}$ with
  $Q \geq 0, Q \leq 0$ or $\abs{Q} < \infty$ we define
  \[ P_\pi Q(s, a) = \int Q(s', a') \difd \pi P(s', a' \mid s, a) \]
  \[ T_\pi Q = r + \gamma P_\pi Q \]
  \[ T Q(s, a) = r(s, a) + \gamma
  \int \sup_{a' \in \Cal{A}} Q(s', a') \difd P(\cdot \mid s, a) \]
  where $T_a = T_{\delta_a}$.
\end{defn}
 
\begin{prop}[Properties of T-operators for Q-functions]
  Let $\pi = (\pi_1, \pi_2, \dots) \in R\Pi$.
  \leavevmode
  \begin{enumerate}
    \item If $\pi$ is Markov and $\mu$ is a stationary policy then
      $T_\mu Q_{k, \pi}
      = r + \gamma \E T_\mu V_{k, \pi}$
    \item $Q_{k, \pi} = T_{\pi_1} \dots T_{\pi_k} Q_0$. 
    \item For stationary $\pi \in S\Pi$ we have
      $T_\pi Q_\pi = Q_\pi$.
    \item (D) $T_\pi$ is $\gamma$-contractive on
      $\Cal{L}_\infty(\Cal{S}\times\Cal{A})$
      and $Q_\pi$ is the unique fixed point of $T_\pi$ in
      $\Cal{L}_\infty(\Cal{S}\times\Cal{A})$.
  \end{enumerate}
  \label{prop:TQ}
\end{prop}
\begin{proof}
  \leavevmode
  \begin{enumerate}
    \item This is essentially due to properties of the kernels. The idea is
      sketched here
      \begin{align*}
	T_\mu Q_{k, \pi} = r + \gamma \int r + \gamma V_{k, \pi}
	\difd P \difd \mu P
	= r + \gamma \int r + \gamma V_{k, \pi} \difd P \mu \difd P
	= r + \gamma \int T_\mu V_{k, \pi} \difd P
      \end{align*}
    \item Use 1. iteratively starting with
      $\mu = \pi_1, \pi = (\pi_2, \pi_3, \dots)$.
    \item By 2. $T_\pi Q_\pi = T_\pi (r + \gamma \E \lim_{k\to\infty} T_\pi^k V_0)
      = \lim_{k\to\infty} T_\pi (r + \gamma \E T_\pi^k V_0)
      = \lim_{k\to\infty} (r + \gamma \E T_\pi^{k+1} V_0)
      = r + \gamma \E \lim_{k\to\infty} T^{k+1}_\pi V_0
      = r + \gamma \E V_\pi = Q_\pi$.
    \item The contrativeness of $T_\pi$ follows from the same argument as for
      value functions. 2. and Banach fixed point theorem does the rest.
  \end{enumerate}
\end{proof}

\begin{defn}
  Let $\pi : \Cal{S} \leadsto \Cal{A}$ be a stationary policy. Define
  $A_s = \argmax_{a \in \Cal{A}} Q(s, a)$.
  If there exist a measurable subset $B_s \subseteq A_s$
  for every $s \in \Cal{S}$ such that
  \[ \pi \left( B_s \Mid s \right) = 1 \]
  then $\pi$ is said to be \defemph{greedy} with respect to $Q$ and is
  denoted $\pi_Q$.
\end{defn}

\begin{prop}[Properties of greedy policies]
  For any integrable $Q : \Cal{S} \times \Cal{A} \to \ol{\ul{\R}}$
  if $\pi_Q$ is greedy with respect to $Q$ then $T_{\pi_Q} Q = TQ$.
\end{prop}
\begin{proof}
  \begin{align*}
    T_{\pi_Q} Q &= r + \gamma \int Q(s, a) \difd \pi P(s, a \mid \cdot)
    \\ &= r + \gamma \int \int Q(s, a)
    \difd \pi_Q(a \mid s) \difd P(s \mid \cdot)
    \\ &= r + \gamma \int \max_{a \in \Cal{A}} Q(s, a)
    \difd P(s \mid \cdot)
    \\ &= T Q
  \end{align*}
\end{proof}

\subsection{Bertsekas-Shreve framework}
The theory described here is largely based on
[ref to Bertsekas-Shreve, Stochastic Optimal Control].
Their framework is cost-based as opposed to the this paper reward-based outset.
This means that positive and negative, upper and lower, supremum and infimum,
ect. are opposite to the source.
\begin{sett}[BS]
  \begin{itemize}
    \item We consider an MDP $(\Cal{S}, \Cal{A}, P, R, \gamma)$
      (see \cref{sett:MDP}).
    \item $\Cal{S}$ and $\Cal{A}$ are Borel spaces.
    \item $\Cal{A}$ is compact.
    \item $P(S \mid \cdot)$ is continuous for any $S \in \Sigma_{\Cal{S}}$.
    \item $r(s,a) = \gamma^{1-i} \int x \difd R(x \mid s, a)$ 
      is upper semicontinuous and uniformly bounded from above
      (least upper bound denoted $0 < R_{\max} < \infty$).
    \item The policies must consist of
      universally measurable probability kernels.
    \item One of (\cref{cond:P}), (\cref{cond:N}) or (\cref{cond:D})
      is always assumed.
  \end{itemize}
  \label{sett:BS}
\end{sett}
The original setup in [ref to Bertsekas-Shreve, Stochastic Optimal Control]
is slightly different than the setup here presented.
Besides having a state and action space, it also features a 
non-empty Borel space called the
\emph{disturbance space} $W$, a \emph{disturbance kernel}
$p: \Cal{S} \times \Cal{A} \to W$,
instead of a transition kernel which on the other hand is a deterministic
\emph{system function} $f : \Cal{S} \times \Cal{A} \times W \to \Cal{S}$
which should be Borel measurable.
Moreover it allows for constrains on the action space for each state.
This is made precise by a function $U:\Cal{S} \to \Sigma_{\Cal{A}}$
and a restriction on $R\Pi$ that all policies $\pi$ should satisfy
$\pi(U(s) \mid s) = 1$.
Lastly the rewards are interpreted as negative costs, and thus
$g$ is required to be semi \emph{lower}continuous.

By setting $P(\cdot \mid s, a) = f(s, a, p(\cdot \mid s, a))$
and maximizing rewards of upper semicontinuous instead of
minimizing lower semicontinuous ones, we fully capture
all aspects of the original process and its results,
except the for the action constrains. %todo make a more precise argument

Notice that \cref{sett:BS} implies (\cref{cond:F+}).
Throughout this section 
are always assumed. 

\begin{prop}
  Let $\Cal{X}, \Cal{Y}$ be separable and metrizable,
  $\kappa : \Cal{X} \to \Cal{Y}$ be a continuous probability kernel
  and $f:\Cal{X} \times \Cal{Y} \to \ul{\ol{\R}}$ be Borel-measurable
  satisfying one of
  $f \leq 0, f \geq 0, \abs{f} < \infty$.
  If $f$ is bounded from above (below) and upper (lower) semicontinuous
  then
  \[ x \mapsto \int f \difd \kappa(\cdot \mid x) \]
  is bounded from above (below) and upper (lower) semicontinuous. 
  \label{prop:BS7_31}
\end{prop}
\begin{proof}
  We refer to [BS SOC, prop. 7.31]. %todo do it yourself
\end{proof}

%proposition 8.6 Stoch. Opt. Control
\begin{prop}[Prop. 8.6 in BS]
  $V^*_k = T^k V_0$ and is upper semicontinuous.
  Furthermore for any $k \in \N$
  there exists a deterministic, Markov, Borel-measurable $k$-optimal policy
  $\pi^*_k = (\pi^*_{k,1}, \pi^*_{k,2}, \dots, \pi^*_{k,k}, \dots) \in DM\Pi$.
  These policies satisfy
  $\pi^*_{i} = (\pi^*_{k,k-i}, \dots, \pi^*_{k,k}, \dots)$ for any $i < k$.
  \label{prop:BSprop8_6}
\end{prop}

%cor. 9.17.2
\begin{thm}[Cor. 9.17.2 in BS]
  Under (\cref{cond:N}) or (\cref{cond:D})
  $V^* = \lim_{k\to\infty} V_k^*$ and is upper semicontinuous.
  Furthermore there exist a deterministic
  stationary, Borel-measurable policy $\pi^*$.
  \label{thm:BScor9.17.2}
\end{thm}

\subsubsection{Analytic setting}

\begin{sett}[BS Analytic]
  The same as \cref{sett:BS} except:
  $P$ is not necessarily continuous.
  $r$ is upper semianalytic.
  $\Cal{A}$ is not necessarily compact, but
  there exists a $k \in \N$ such that
  $\forall \lambda \in \R, n \geq k, s \in \Cal{S}$
  \[ A^\lambda_n(s) = \left\{ a \in \Cal{A} \Mid r(s, a)
  + \gamma \int V^*_n P(\cdot \mid s, a) \geq \lambda \right\} \]
  is a compact subset of $\Cal{A}$.
  \label{sett:BSA}
\end{sett}

\begin{thm}[Prop. 9.17 BS]
  Under \cref{sett:BSA} we have
  $V^* = \lim_{n \to \infty} V^*_n$ for all $s \in \Cal{S}$
  and there exists a optimal policy $\pi^*$ which is stationary
  and deterministic.
\end{thm}
\begin{proof}
  We refer to [todo ref to Bertsekas and Schreve, Stochastic Optimal Control:
  The Discrete-Time Case, prop. 9.17].
\end{proof}

\subsubsection{Implications for value-functions}
Let \cref{sett:BS} hold.

\begin{prop}
  $V^* = V_{\pi^*} = T_{\pi^*} V^* = T V^*$
  
  (D) $V^*$ is the unique fixed point of $T$ in $\Cal{L}_\infty(\Cal{S})$.
  \label{prop:VoptEqVpiOpt}
\end{prop}
\begin{proof}
  Since $\pi^*$ is optimal $V^* = V_{\pi^*}$ which by \cref{prop:propTV}
  equals $T_{\pi^*} V_{\pi^*}$.
  By \cref{thm:BScor9.17.2} and \cref{prop:BSprop8_6}
  $T V^* = T \lim_{k\to\infty} T^k V_0 =
  \lim_{k\to\infty} T^{k+1} V_0 = V^*$.
  If (D) holds $V^* \in \Cal{L}_\infty(\Cal{S})$ so by \cref{prop:propTV} 5.
  and 6. we are done.
\end{proof}

\begin{prop}
  \leavevmode
  \begin{enumerate}
    \item $Q^*_k = r + \gamma \E V^*_k$ and is upper semicontinuous.
    \item (N) (D) $Q^* = r + \gamma \E V^*$ and is upper semicontinuous.
    \item (N) (D) $\sup_{a \in \Cal{A}} Q^*(s, a) = V^*(s)$.
    \item (N) (D) $Q^* = \lim_{k\to\infty} Q_k^*$.
    \item (N) (D) $Q^* = Q_{\pi^*}$.
  \end{enumerate}
\end{prop}
\begin{proof}
  \leavevmode
  \begin{enumerate}
    \item Since $V_k^*$ is measurable due to \cref{prop:BSprop8_6}
      we see that
      $Q_k^* = \sup_{\pi \in R\Pi} (r + \gamma \E V_{k,\pi})
      \leq r + \gamma \E V_k^* = r + \gamma \E V_{\pi_k^*}
      \leq Q_k^*$.
      \Cref{prop:BS7_31} gives upper semicontinuity.
    \item Since $V^*$ is measurable due to \cref{thm:BScor9.17.2}.
      Now follow the argument for 1.
    \item Let $s \in \Cal{S}$ then $\sup_{a \in \Cal{A}} Q^*(s, a) = 
      \sup_{a \in \Cal{A}} (r(s, a) + \gamma \E_{P(\cdot \mid s, a)} V^*)
      = T V^*(s) = V^*(s)$.
    \item By monotone or dominated convergence and \cref{thm:BScor9.17.2}.
    \item By \cref{prop:VoptEqVpiOpt} and 2.
      $Q^* = r + \gamma \E V^* = r + \gamma \E V_{\pi^*} = Q_{\pi^*}$.
      \end{enumerate}
\end{proof}

\begin{prop}
  \leavevmode
  \begin{enumerate}
    \item $TQ^*_k = r + \gamma \E T V^*_k$ and if
      $\pi^* = (\pi^*_1, \pi^*_2 \dots)$
      is $k$-optimal then
      $Q^*_k = T_{\pi^*_1} \dots T_{\pi^*_k} r = T^k r$.
    \item $TQ^* = r + \gamma \E T V^*$ and $TQ^* = Q^*$.
    \item (D) $T$ is $\gamma$-contractive on
      $\Cal{L}_\infty(\Cal{S}\times\Cal{A})$
      and $Q^*$ is the unique fixed point of $T$ in 
      $\Cal{L}_\infty(\Cal{S}\times\Cal{A})$.
  \end{enumerate}
  \label{prop:TQfp}
\end{prop}

\begin{proof}
  \leavevmode
  \begin{enumerate}
    \item \begin{align*}
	TQ^*_k(s,a) &= T(r + \gamma \E V^*_k)(s,a)
	\\ &= r(s,a) + \gamma
	\int \sup_{a' \in \Cal{A}} (r(s',a')
	+ \gamma \E_{P(\cdot \mid s', a')} V^*_k)
	\difd P(s' \mid s,a)
	\\ &=r(s,a) + \gamma
	\int \sup_{a' \in \Cal{A}} \left(r(s',a') + \gamma
	\int V_k^*(s'') \difd P(s'' \mid s', a') \right)
	\difd P(s' \mid s,a)
	\\ &= r(s, a) + \gamma
	\int T V^*_k(s') \difd P(s' \mid s, a)
      \end{align*}
      To get $Q^*_k = T^k r$ use this inductively
      $Q^*_k = r + \gamma \E V^*_k = r+ \gamma TV^*_{k-1}
      = T Q^*_{k-1} = \dots$.
      The statement $Q^*_k = T_{\pi^*_1} \dots T_{\pi^*_k} r$
      is from \cref{prop:TQ}.
    \item The argument from 1. also implies this first statement in
      2. Now $TQ^* = r + \gamma \E TV^* = r + \gamma \E V^* = Q^*$
      by \cref{prop:VoptEqVpiOpt}.
    \item The argument is similar to \cref{prop:propTV} pt. 5.
  \end{enumerate}
\end{proof}

\begin{cor} (D)

  For any $Q \in \Cal{L}_\infty(\Cal{S} \times \Cal{A})$
  $T^k Q$ converges to $Q^*$ with rate $\gamma^k$.
  That is
  \[ \norm{T^k Q - Q^*}_\infty \leq \gamma^k \norm{Q - Q^*}_\infty \]
  \label{cor:QrateSimple}
\end{cor}
\begin{proof}
  This is directly from \cref{prop:TQfp} pt. 3.
\end{proof}

\begin{prop}
  \leavevmode
  \begin{enumerate}
    \item Let $\pi_i$ be greedy w.r.t. $Q_{i-1}^*$ then
      $(\pi_i, \pi_{i-1}, \dots, \pi_1)$ is $i$-optimal for any $i \in \N$.
    \item (N) (D) Any greedy strategy for $Q^*$ is optimal and such exist.
  \end{enumerate}
\end{prop}
\begin{proof}
  \begin{enumerate}
    \item Such greedy policies exist because $Q_{k, \pi}$ is upper
      semicontinuous by \cref{prop:BSprop8_6}.
      For induction base observe that
      $ Q_{1, \pi_1} = T_{\pi_1} Q_0 = T Q_0 = Q_1^*$.
      Now assume $Q_{i-1, {\pi_{i-1}, \dots, \pi_1}} = Q^*_{i-1}$.
      Then
      $Q_{i, (\pi_i, \dots, \pi_1)}
      = T_{\pi_i} Q_{i-1, (\pi_{i-1}, \dots, \pi_1)}
      = T_{\pi_i} Q^*_{i-1} = T Q_{i-1}^* = Q_i^*$.
    \item Since $Q$ is upper semicontinuous in the second entry
      the set $A_s = \argmax_{a \in \Cal{A}} Q(s, a)$ is non-empty
      and measurable for all $s$.
      Pick (by axiom of choice) an $a_s \in A_s$ for every $s \in \Cal{S}$.
      Then $\pi(\cdot \mid s) = \delta_{a_s}$ is greedy with respect to $Q$.
      %todo is pi measurable?
  \end{enumerate}
\end{proof}

\begin{rem}
  Most of the results of this section hold
  also under \cref{sett:BSA} with the addition
  that 'semicontinuous' is replaced by 'semianalytic'.
\end{rem}

Based on the results established so far we can as a non-practical
example design the following algorithm:

\begin{figure}[H]
\begin{algorithm}[H] %\label{algocf:fq} % this labels line, could not fix
\caption{Simple theoretical Q-iteration}
\KwIn{MDP $(\Cal{S}, \Cal{A}, P, R, \gamma)$, number of iterations $K$}
$\forall (s, a) \in \Cal{S} \times \Cal{A} :
r(s, a) \leftarrow \int x \difd R(x \mid s, a)$.

$\wt{Q}_0 \leftarrow r$

\For{$k = 0,1,2,\dots,K-1$}{
  $ \forall (s, a) \in \Cal{S} \times \Cal{A} :
  \wt{Q}_{k+1}(s, a) \leftarrow r(s, a)
  + \gamma \int \sup_{a' \in \Cal{A}} \wt{Q}_k(s', a') \difd P(s' \mid s, a)$
}
Define $\pi_K$ as the greedy policy w.r.t. $\wt{Q}_K$ \\
\KwOut{An estimator $\widetilde{Q}_K$ of $Q^*$ and policy $\pi_K$}
\label{alg:theoSimpleQ}
\end{algorithm}
\end{figure}

\begin{prop}(D)

  The output $\wt{Q}_K$ of \cref{alg:theoSimpleQ} converges to the optimal
  Q-function $Q^*$ with rate $\gamma^K$ concretely
  $\norm{\wt{Q}_K - Q^*}_\infty \leq \gamma^K \norm{Q^*}_\infty$.
  \label{prop:theoSimpleQConv}
\end{prop}
\begin{proof}
  This is by \cref{cor:QrateSimple}.
\end{proof}

\subsubsection{Finite Q-iteration}
Concluding on the results so far
we have showed how if one knows the dynamics
of a stationary decision process satisfying rather broad criteria, 
such as continuity and compactness,
the optimal policy and state-value function can be found
simply by iteration over the $T$-operator and picking a greedy strategy
(see \cref{prop:theoSimpleQConv}).
Of course this is practical computationally, only if
the resulting $Q$ functions can be represented and computed in finite
space and time.
This is trivially the case when
\begin{asm}
  $\Cal{S}\times\Cal{A}$ is finite.
  \label{asm:finite}
\end{asm}
Say $\abs{\Cal{S}} = k$ and $\abs{\Cal{A}} = \ell$.
In this case the transition operator $P$ can be represented as a
matrix of \emph{transition probabilities}
\[ P \defeq \begin{pmatrix}
    P(s_1 \mid s_1, a_1) & \dots & P(s_k \mid s_1, a_1)
    \\ \vdots & \vdots & \vdots
    \\ P(s_1 \mid s_k, a_\ell) & \dots & P(s_k \mid s_k, a_\ell)
\end{pmatrix} \]
then the algorithm becomes

\begin{algorithm}[H] %\label{algocf:fq} % this labels line, could not fix
\caption{Simple finite Q-iteration}
\KwIn{MPD $(\Cal{S}, \Cal{A}, P, R, \gamma)$, number of iterations $K$}
Set $ r \leftarrow \left(\int r \difd R(\cdot \mid s_1, a_1),
\dots, \int r \difd R(\cdot \mid s_k, a_\ell) \right)^T $

and $ \wt{Q}_0 \leftarrow r$.

\For{$k = 0,1,2,\dots,K-1$}{
  Set $m(\wt{Q}_k) \leftarrow (\max_{a \in \Cal{A}} Q(s_1, a), \dots,
  \max_{a \in \Cal{A}}Q(s_k, a))^T$

  Update action-value function:
  \[ \wt{Q}_{k+1} \leftarrow
    r + \gamma P m(\wt{Q}_k)
  \]
}
Define $\pi_K$ as the greedy policy w.r.t. $\wt{Q}_K$ \\
\KwOut{An estimator $\widetilde{Q}_K$ of $Q^*$ and policy $\pi_K$}
\label{alg:finiteSimpleQ}
\end{algorithm}

\begin{prop}
  The output $\wt{Q}_K$ from \cref{alg:finiteSimpleQ} is
  $K$-optimal and
  $\norm{\wt{Q}_K - Q^*}_\infty \leq \gamma^K \norm{Q^*}_\infty$.
\end{prop}
\begin{proof}
  See \cref{prop:theoSimpleQConv}.
\end{proof}


%todo check measurability (universal) issues in the above sections

\subsection{Approximation}

In this section we will look at what happens if we
instead use approximations of the $Q$-functions and $T$ operator.
We first look at a naive approach using $Q$-functions.

Let $\wt{Q}_0$ be any bounded Q-function.
Suppose we approximate $T\wt{Q}_0$ by a Q-function $\wt{Q}_1$
to $\ve_1 > 0$ precision and then approximate $T\wt{Q}_1$ and so on
getting a sequence of Q-functions satisfying
\[ \abs{T\wt{Q}_{k-1} - \wt{Q}_k} \leq \ve_k, \forall k \in \N \]

First observe that
\begin{align*}
  \abs{T^k \wt{Q}_0 - \wt{Q}_k}
  &\leq \abs{T^k \wt{Q}_0 - T \wt{Q}_{k-1}} + \abs{T\wt{Q}_{k-1} - \wt{Q}_k}
  \\ &\leq \gamma \abs{T^{k-1} \wt{Q}_0 - \wt{Q}_{k-1}}
  + \abs{T\wt{Q}_{k-1} - \wt{Q}_k}
\end{align*}

Using this iteratively we get
\[ \abs{T^k \wt{Q}_0 - \wt{Q}_k} \leq \sum_{i=1}^k \gamma^{k-i} \ve_i
\defeq \ve_a(k) \]

Then we can bound
\begin{align*}
  \abs{Q^* - \wt{Q}_k}
  &\leq \abs{Q^* - T^k \wt{Q}_0} + \abs{T^k \wt{Q}_0 - \wt{Q}_k}
  \\ &\leq \gamma^k \abs{Q^* - \wt{Q}_0}
  + \ve_a(k)
\end{align*}

These terms are sometimes called the \emph{algorithmic}
and \emph{approximation} errors.

The algorithmic error quickly while the other depends on our
step-wise approximations. For example
$\ve_i(k) = \ve$ we easily get the bound
$ \ve_a(k) = \ve \frac{1-\gamma^k}{1-\gamma} \leq \frac{\ve}{1-\gamma} $
Or if $\ve_i \leq c\gamma^i$ we get $\ve_a(k) \leq ck \gamma^k \to 0$ as
$k \to \infty$.
Generally if one can show that $\ve_i \to 0$ we have
\begin{prop} $ \sum_{i-1}^k \gamma^{k-i} \ve_i \to 0 $
  whenever $\ve_k \to 0$ as $k \to \infty$.
\end{prop}
\begin{proof}
  Let $\ve > 0$. Find $N$ such that $\ve_n \leq \ve (1-\gamma)/2$ 
  for all $n>N$ and find $M>N$ such that
  $\gamma^M \leq
  \ve \gamma^N \left( \sum_{i=1}^N \gamma^{N-i} \ve_i \right)^{-1}$.
  Then for all $m>M$
  \begin{align*}
    \sum_{i=1}^m \gamma^{m-i} \ve_i
    &\leq \gamma^{m-N} \sum_{i=1}^N \gamma^{N-i} \ve_i
    + \sum_{i=N+1}^m \gamma^{m-i} \ve (1-\gamma)/2
    \leq \ve/2 + \ve/2 \leq \ve
  \end{align*}
\end{proof}

Let $\Cal{F}$ be a class of functions

\begin{thm}[Universal Approximation Theorem for ANNs]
  Let $\sigma: \R \to \R$ be non-constant, bounded and continuous function.
  Let $\ve > 0$ and $f \in C(I_m)$.
  Then there exists a ANN $F$ with one hidden layer
  and activation function $\sigma$ such that
  \[ \norm{F - f}_\infty < \ve \]
\end{thm}



\section{Hidden dynamics}

In this section we will look at what can be done when the process dynamics
are unknown.
In this case we cannot calculate directly neither $r$, $T_\pi Q$ nor
$TQ$ because the transition and reward kernels $P,R$ are unknown.

It is clear that \cref{alg:theoSimpleQ} will not work without
modification in this case. Simply because $R$ and $P$ are not
available.
To make the scheme work anyway we could simply avoid taking expectations
and use the random outcomes of the kernels.
Leading to

\begin{figure}[H]
\begin{algorithm}[H] %\label{algocf:fq} % this labels line, could not fix
  \caption{Random theoretical Q-iteration (example of thought)}
\KwIn{MDP $(\Cal{S}, \Cal{A}, P, R, \gamma)$, number of iterations $K$}
$\forall (s, a) \in \Cal{S} \times \Cal{A} :
\wt{Q}_0(s, a) \leftarrow X \sim R(\cdot \mid s, a)$.

\For{$k = 0,1,2,\dots,K-1$}{
  $ \forall (s, a) \in \Cal{S} \times \Cal{A} :
  \wt{Q}_{k+1}(s, a) \leftarrow r'
  + \gamma \sup_{a' \in \Cal{A}} \wt{Q}_k(s', a')$

  where $r' \sim R(\cdot \mid s, a), s' \sim P(\cdot \mid s, a)$.
}
Define $\pi_K$ as the greedy policy w.r.t. $\wt{Q}_K$ \\
\KwOut{An estimator $\widetilde{Q}_K$ of $Q^*$ and policy $\pi_K$}
\label{alg:theoRandomQ}
\end{algorithm}
\end{figure}
We immedially run into problems in the uncountable case, because
drawing uncountably many times from a distribution is not easily
defined in a sensible way.
Even in the finite case where the $\wt{Q}_k$s
are well defined, they cannot converge if $R$ is not deterministic.
Therefore this approach is not attractive in a continuous or
stochastic setting.

There are broadly two ways of dealing with such 

\subsection{Finite case}

A common way to overcome the problem of convergence
is called \emph{temporal difference} (TD) learning and is based on the following
update scheme
\begin{equation}
  \wt{Q}_{k+1}(s, a) \leftarrow (1-\alpha_k) \wt{Q}_k(s, a)
  + \alpha_k (r' + \gamma \cdot \max_{a' \in \Cal{A}} \wt{Q}_k(s', a'))
  \label{eq:tdeq}
\end{equation}
Here $r'$ and $s'$ are the reward and next-state drawn from the
reward and transition kernels,
and $\alpha_k \in [0,1]$ is the so-called \defemph{learning rate}
(of the $k$th step).
The 'temporal difference' is also the name of term
$ \alpha_k ( r' + \gamma \cdot \max_{a \in \Cal{A}} \wt{Q}_k(s', a')
- \wt{Q}_k(s, a) )$ occuring from rearranging \cref{eq:tdeq}.
Usually the learning rate is fixed before running the algoritm
(does not depend on the history) and is set to decay
from 1 to 0 in some fashion as $k \to \infty$.

We will now look at a convergence result originally obtained by Watkins and Dayan
[todo ref Watkins and Dayan, 1992] of a TD algorithm using Q-functions,
which was extended slightly by [Jaakola, Jordan, Singh, 1993].

\begin{algorithm}[H] %\label{algocf:fq} % this labels line, could not fix
  \caption{Finite asynchronos Q-learning}
  \KwIn{MDP $(\Cal{S}, \Cal{A}, P, R, \gamma)$ such that
    $\abs{\Cal{S}}\abs{\Cal{A}} < \infty$,
    number of iterations $K$,
    state-action pairs $(s_1,a_1, \dots, s_K, a_K)$,
    learning rates $(\alpha_1', \dots, \alpha_K')$,
    initial $\wt{Q}_0 : \Cal{S}\times\Cal{A} \to \R$
  }
  Put $\alpha_k(s, a) \leftarrow \begin{cases}
    \alpha'_k & (s, a) = (s_k, a_k)
    \\ 0 & (s, a) \neq (s_k, a_k)
  \end{cases}$.

  \For{$k = 1,2,\dots,K$}{
    Sample $r' \sim R(\cdot \mid s_k, a_k),
    s' \sim P(\cdot \mid s_k, a_k)$

    Update action-value function:
    \[ \wt{Q}_k \leftarrow
      \wt{Q}_{k-1} +
      \alpha_k (r' + \max_{a' \in \Cal{A}} \wt{Q}_{k-1}(s', a'))
    \]
  }
  Define $\pi_K$ as the greedy policy w.r.t. $\wt{Q}_K$ \\
  \KwOut{An estimator $\widetilde{Q}_K$ of $Q^*$ and policy $\pi_K$}
  \label{alg:finAsyncQL}
\end{algorithm}
Note that only the value of the pair $(s_k, a_k)$ are updated in each
step of the algorithm
(since $\alpha_k(s, a) = 0$ for all $(s,a)\neq(s_k, a_k)$).

\begin{thm}[Watkins, Dayan] %todo add year
  Let $s_1, a_1, s_2, a_2, \dots \in
  \Cal{S} \times \Cal{A} \times \Cal{S} \times \Cal{A} \times \dots$
  be random variables, and $\alpha_1, \alpha_2, \dots \in [0,1]$.
  The output $\wt{Q}_K$ of \cref{alg:simpleQL} converges to $Q^*$
  provided
  \begin{enumerate}
    \item $\Prob\left(\sum_{i=1}^\infty \alpha_i(s,a) = \infty\right) = 1,
      \Prob\left(\sum_{i=1}^\infty \alpha_i^2(s,a) < \infty\right) = 1$.
    \item $\Var(R(\cdot \mid s, a)) < \infty$ for all $(s, a) \in
      \Cal{S}\times\Cal{A}$.
    \item If $\gamma = 1$ all policies lead to a reward-free terminal
      state almost surely.
  \end{enumerate}
\end{thm}
In the original formulation the sums of learning rates were supposed to
converge \emph{uniformly}. However this is equivalent to this formulation
because of the fact that
$\Prob(\sup_{(s, a) \in \Cal{S} \times \Cal{A}} \abs{f_n(s, a)} \to 0) = 1 \iff
\Prob(\abs{f_n(s, a)} \to 0) = 1, \forall (s, a) \in \Cal{S} \times \Cal{A}$
whenever $\Cal{S}, \Cal{A}$ is finite.
Notice that the first condition implies that all state-action pairs
must occur infinitely often almost surely.
Also notice that the second condition is automatically fulfilled under
(D) since then $\Var(R(\cdot \mid s, a)) \leq \E (2R_{\max})^2 = 4 R_{\max}^2$.

%% Szepesvári Asymptotic convergence-rate of Q-learning ¿1996?
In a special case of the same setup, convergence rates where
established by [Szepesvári 1996]. %todo ref
\begin{thm}[Szepesvári]
  Let $t \in \N$ and
  $s_1, a_1, s_2, \dots, s_t, a_t$ be sampled i.i.d. from
  $p \in \Cal{P}(\Cal{S} \times \Cal{A})$.
  Set the learning rates such that
  $\alpha'_k
  = |\{ i \in [k-1] \mid (s_i, a_i) = (s_k, a_k) \}|^{-1}$,
  i.e. the reciprocal of the frequency of $(s_k, a_k)$ at step $k$.
  Let $\beta = \max_{x \in \Cal{S} \times \Cal{A}} p(x) /
  \min_{x \in \Cal{S} \times \Cal{A}} p(x)$.
  Then for some $B > 0$ the following holds asymptotically almost surely
  \begin{equation}
    \abs{\wt{Q}_t - Q^*} \leq B \frac{1}{t^{\beta (1-\gamma)}}
    \label{eq:Szepesvari1}
  \end{equation}
  and
  \begin{equation}
    \abs{\wt{Q}_t - Q^*} \leq B \sqrt{\frac{\log \log t}{t}}
    \label{eq:Szepesvari2}
  \end{equation}
\end{thm}
Here \cref{eq:Szepesvari1} is tightest when $\gamma > 1 - \beta/2$
otherwise \cref{eq:Szepesvari2} is tighter.
%todo find out if we wanna do proof

A paper [MH, 2003] proves that $Q$-learning is PAC-learnable
given some additional assumptions.

\begin{algorithm}[H] %\label{algocf:fq} % this labels line, could not fix
  \caption{Finite synchronos Q-learning}
  \KwIn{MDP $(\Cal{S}, \Cal{A}, P, R, \gamma)$ such that
    $\abs{\Cal{S}}\abs{\Cal{A}} < \infty$,
    number of iterations $K$,
    learning rates $(\alpha_1, \dots, \alpha_K)$,
    initial $\wt{Q}_0 : \Cal{S}\times\Cal{A} \to \R$
  }

  \For{$k = 1,2,\dots,K$}{
    Sample $r' \sim R(\cdot \mid s_k, a_k),
    s' \sim P(\cdot \mid s_k, a_k)$

    Update action-value function:
    \[ \wt{Q}_k \leftarrow
      \wt{Q}_{k-1} +
      \alpha_k (r' + \max_{a' \in \Cal{A}} \wt{Q}_{k-1}(s', a'))
    \]
  }
  Define $\pi_K$ as the greedy policy w.r.t. $\wt{Q}_K$ \\
  \KwOut{An estimator $\widetilde{Q}_K$ of $Q^*$ and policy $\pi_K$}
  \label{alg:finSyncQL}
\end{algorithm}


\begin{thm}[Mansour 2003]
  Assume (P) and (D). Let $\alpha_k = 1/(k+1)^\omega$ where
  $\omega \in (1/2,1]$.
  Fix $C>0$, a sufficiently large constant.
  Let $\ve, \delta >0$ and define
  \[ A = \frac{4V^2_{\max} \log(2\abs{\Cal{S}}\abs{\Cal{A}} V_{\max}/\delta
  (1-\gamma)\ve)}{(1-\gamma)^2\ve^2}, \quad B= 2 \log(V_{\max}/\ve)/(1-\gamma) \]
  The following hold for any $\psi > 0$.

  If the synchronos algorithm (\cref{alg:finSyncQL}) is run with
  \[ K \geq C \begin{cases}
      A^{1/\omega} + B^{1/(1-\omega)} & \omega \in (1/2,1)
      \\ \frac{(2 + \psi)^B}{\psi^2} \left(A +
      \frac{4 V^2_{\max} \log(1/\psi)}{(1-\gamma)^2 \ve^2} \right)
      & \omega = 1
    \end{cases}
  \]
  then with probability at least $1-\delta$ we have
  $\norm{\wt{Q}_K - Q^*}_\infty < \ve$.

  If the asynchronos algorithm (\cref{alg:finAsyncQL}) with 
  \[ K \geq C \begin{cases}
      (L^{1 + 3 \omega} A)^{1/\omega} + (L B)^{1/(1-\omega)}
      & \omega \in (1/2,1)
      \\ \frac{(L + \psi L + 1)^B}{\psi^2}
      \left(A + \frac{4V^2_{\max} \log(1/\psi)}{(1-\gamma)^2\ve^2} \right)
	& \omega = 1
  \end{cases} \]
  and the state-action pairs $(s_1, a_1, \dots, s_K, a_K)$ are drawn
  from a distribution such that every pair in $\Cal{S}\times\Cal{A}$
  appears in every sequence of length at least $L > 0$,
  then with probability at least $1-\delta$ we have
  $\norm{\wt{Q}_K - Q^*}_\infty < \ve$.
  \label{thm:mansour2003}
\end{thm}
In [MH] $L$ is called the \emph{covering rate}.

An interesting side note to \cref{thm:mansour2003} is that
one can use the bounds to give hints at how to tune the learning rate
by changing $\omega$. Optimizing for different scenarios yield
different learning theoretically optimal values for $\omega$.
For example if we want to optimize for the bound on $K$ for 
$\gamma \to 1$ using the synchronos algorithm,
we get the following rate (treating other variables as constant)
$K \geq C'(1/(1-\gamma)^{4/\omega} + 1/(1-\gamma)^{1/(1-\omega)}$.
Thus picking $\omega = 4/5$ is optimal.
As another example: running the asynchronos algorithm and wanting to
minimize for large covering rates $L$. We get
$K \geq C''(L^{2+1/\omega} + L^{1/(1-\omega)})$
which is optimized for $\omega \approx 0.77$.
It is then shown (in [MH]) experimentally that $\omega = 0.85$ is 
optimal using a way of generating random finite MDPs.
It seems this value has since been adopted as a kind of standard
value for the learning rate
(see e.g. [Devraj Meyn, 2018]) % todo find more examples

\subsubsection{History dependent setting}

\begin{sett}[Finite HDP]
  \leavevmode
  \begin{enumerate}
    \item A history dependent decision process (see \cref{sett:HDP}),
      with a single \emph{finite} state space,
      a single finite action space $(\Cal{S}, \Cal{A})$,
      and transition and reward kernels $(P_n, R_n)_{n \in \N}$.
      Define $\Cal{H}^* \defeq \bigcup_{i \in \N} \Cal{H}_n$,
      the space of finite histories.
    \item $(P_n)_{n \in \N}$ is viewed as a single kernel
      $P : \Cal{H}^* \times \Cal{A} \leadsto \Cal{S}$.
    \item $(R_n)_{n \in \N}$ is deterministic and viewed as a single function
      $r : \Cal{H}^* \to \R$.
      This is discounted by $\gamma \in [0,1)$ in accordance to
      condition (D). That is
      $r(h_n)$ is bounded in the interval
      $[-\gamma^{n-2}R_{\max}, \gamma^{n-2} R_{\max}]$ for any
      $h_n \in \Cal{H}_n$.
      Furthermore $r$ depends only on $s_1 a_1 \dots s_k r_{k-1} a_k$
      when evaluated on
      $h_{k+1} = s_1 a_1 \dots r_{k-1} a_k s_{k+1} \in \Cal{H}_{k+1}$.
  \end{enumerate}
  \label{sett:HDP_MH}
\end{sett}

\begin{rem}
  Note that the finite \cref{sett:HDP_MH} is a special case of
  \cref{sett:Schal} considered by [Schäl, 1974],
  because Polishness and compactness of $\Cal{S}, \Cal{A}$ is readily
  implied by using the discrete topology in the finite state and action
  spaces, and the fact that (D) implies pt. 5 in \cref{sett:Schal}.
  Further the conditions (S) and (W) of Schal are also both implied by
  the discreteness.
  This implies by \cref{thm:SchalExi} the existence of an optimal
  $\pi^* \in R\Pi$ and that $V^*_n \to V^*$.
\end{rem}

Within \cref{sett:HDP_MH} Q-functions are generalized so that
they are taking values in $\Cal{H}^* \times \Cal{A}$.
We likewise generalize the $T$ function by
\[ TQ(h, a) \defeq r' +
  \gamma \sum_{s\in\Cal{S}} \max_{a' \in \Cal{A}} Q(hr'as, a') P(s \mid ha),
\qquad r' = r(h, a)\]

The optimal Q-function $Q^*$ is defined in [Majeed, Hutter] as
the fixed point of the $T$ operator in
$\Cal{L}_\infty(\Cal{H}^* \times \Cal{A})$.
%todo prove that this is the normal sup definition

Now a function $\phi : \Cal{H}^* \to \Cal{X}$ is introduced
which maps a history to a new finite space $\Cal{X}$.
The intuition here is that $x_n = \phi(h_{n-1} r_{n-2} a_{n-1} s_n)$ is the
state $s_n$ as it is perceived by the agent.
This is called \defemph{partial observability}.
$\phi$ is a assumed to be surjective so $\Cal{X}$ is a finite space of
reduced size in comparison to $\Cal{S}$.
In applications this could be a partially observable environment
or a latent space.

This way we are now considering a class of problems
which is wider than a 
history dependent decision process (HDP).
Namely a partially observable HDP or shortened: POHDP.
A HDP under \cref{sett:HDP_MH} is the subclass of POHDP where
$\Cal{S} = \Cal{X}$ and $\phi = \id_\Cal{S}$.

Let $\phi_{hra}(s) = \phi(hras)$. Then we can define a kernel
\begin{align*} p_h & : \{\phi(h)\} \times \Cal{A} \leadsto \Cal{X}
  \\ p_h & (x' \mid xa)
  = \sum_{s:\phi(hr'as)=x'} P(s \mid ha), \quad r'=r(h,a)
\end{align*}
or expressed as an image measure
$p_h(\cdot \mid xa) = \phi_{hr'a}(P(\cdot \mid ha))$.
and further function $q_h^*$ by the equation
\[ q_h^*(x, a) = r' + \gamma \sum_{x' \in \Cal{X}} 
\max_{a' \in \Cal{A}} q^*_h(x', a')
p_h(x' \mid xa), \quad r'=r(h,a) \]

\begin{asm}[State-uniformity condition]
  For any $h,h' \in \Cal{H}^*$ we have
  \[ \phi(h) = \phi(h') \implies Q^*(h, \cdot) = Q^*(h', \cdot) \]
  \label{asm:stateUniformity}
\end{asm}

A process as in \cref{sett:HDP_MH} together with the state-uniformity condition
is by [MH] called a \emph{Q-Value uniform decision process} (QDP).

\begin{thm}[Hutter, 2016]
  Under \cref{asm:stateUniformity} we have
  $q^*_{h'}(\phi(h), a) = Q^*(h, a)$ for any $h'\in \Cal{H}^*$.
\end{thm}

With this as a motivation we will try to use
the standard TD update step as for an MDP environment:
\begin{equation}
  q_{t+1}(x, a) = q_t(x, a) + \alpha_t(x, a)
  \left(r' + \gamma \max_{a \in \Cal{A}} q_t(x', a') - q_t(x, a)\right),
  \quad x = \phi(h), r' = r(h,a)
  \label{eq:updateStepMH}
\end{equation}

\begin{thm}
  Within \cref{sett:HDP_MH} assume
  \begin{enumerate}
    \item State-uniformity (\cref{asm:stateUniformity}).
    \item Any state is reached eventually under any policy
      (called \emph{state-process ergodicity} in [MH]).
    \item Learning rate satisfies
      \[ \sum_{t=0}^\infty \alpha_t(x, a) = \infty, \quad
      \sum_{t=0}^\infty \alpha_t(x, a)^2 < \infty \]
  \end{enumerate}
  Then starting with any $q_0 : \Cal{X} \times \Cal{A} \to \R$
  the update step \cref{eq:updateStepMH} yields a sequence
  $(q_t)_{t \in \N}$ which converges to the optimal $q^* = Q^*$.
\end{thm}

It seems relevant to ask how restrictive the state-uniformity assumption is.
[MH] answers this by an array of examples showing the following
relations of the classes of decision processes:

\begin{figure}[H]
  \centering
  \begin{tikzpicture}
    \draw (0,0) circle (70pt) ;
    \node at (0,2) { POHDP };
    \draw (0,-0.5) circle [x radius = 2cm, y radius = 1.3cm]; % (40pt) ;
    \node at (0,0.1) { POMDP };
    \draw (0,-1) circle (17pt) ;
    \node at (0,-0.9) { MDP };
    \draw[dashed] (0, -0.3) circle [x radius=1cm, y radius=2cm];
    \node at (0,1.2) { QDP };
  \end{tikzpicture}
  \caption{Classes of finite decision processes considered in [MH].}
  \label{fig:DPMH}
\end{figure}

Recall that QDP is a partially observable HDP under state-uniformity
(\cref{asm:stateUniformity}).

One point remains unclear after reading [MH]:
Why is $q^*$ and $Q^*$ well defined by their recursive definition
and how are they related to the optimal value function
$V^*(s) = \sup_{\pi \in R\Pi} \E_s^\pi
\sum_{i=1}^\infty \gamma^{i-1} r_i $
(see \cref{defn:optimalValue})
of a general HDP?
A sensible thing to ask would be that
$Q^*(h, a) = r(h, a) + \gamma \E_{P(\cdot \mid h a)} V^*$.
However we will not go further into these details.

\subsection{Results for continuous settings}

\subsubsection{Linear function approximation}

This section is based on the 2007 article by [Melo, Ribeiro].

\begin{sett}[Melo, Rebeiro]
  \leavevmode
  \begin{enumerate}
    \item An MDP $(\Cal{S}, \Cal{A}, P, R, \gamma)$ (see \cref{sett:MDP}).
    \item Discounted, i.e. (D) holds with $\gamma \in [0,1)$.
    \item $\Cal{S} \subseteq \R^w$ is compact.
    \item $\Cal{A}$ is finite.
    \item $r_i$ is upper semicontinuous \label{item:MRlast}.
  \end{enumerate}
  \label{sett:MR}
\end{sett}

\begin{rem}
  \Cref{item:MRlast} was actually not part of the assumptions in
  \citet{Nobody06}.
  We include it here in order to ensure the existence of an
  optimal policy and thus measurability of $V^*$.
\end{rem}

Let $\left\{ \xi_1, \dots, \xi_M \right\}$ be a finite set of linearly
independent measurable action value functions,
$\xi_i : \Cal{S} \times \Cal{A} \to \R,\; \forall i \in [M]$.
Denote $\Cal{Q} \defeq \Span\left\{ \xi_i \mid i \in [M] \right\}$
and for $\theta \in \R^M$
\[ Q_\theta(s, a) = \sum_{i=1}^M \theta_i \xi_i(s, a) = \xi^T \theta \]
We would now like to find the best approximation
$q^* \in \Cal{Q}$ to $Q^*$ within the span.
If we measure distance by the $\Cal{L}_2$-norm this
$\rho_\Cal{Q} Q^*$ where $\rho_\Cal{Q}$ is the orthogonal projection on
$\Cal{Q}$.

The gradient of $Q_\theta$ over $\theta$ is
\[ \nabla_\theta Q_\theta(s, a) = \xi(s, a) \]
This gives the idea for a temporal difference with approximation from
$\Cal{Q}$ using the update step
\[ \theta_{k+1} = \theta_k + \alpha_k \xi(s_k, a_k)
  \left( r_k + \gamma \max_{b \in \Cal{A}} Q_{\theta_k}(s_{k+1}, b)
- Q_{\theta_k}(s_k, a_k) \right) \]

\begin{figure}[H]
\begin{algorithm}[H] %\label{algocf:fq} % this labels line, could not fix
  \caption{Q-learning with linear approximation}
  \KwIn{MDP $(\Cal{S}, \Cal{A}, P, R, \gamma)$,
    policy $\pi$,
    number of iterations $K$,
    learning rates $(\alpha_1, \dots, \alpha_K)$,
    initial $\theta_1 \in \R^M$
  }

  \For{$k = 1,2,\dots,K$}{
    Sample $a_k \sim \pi(\cdot \mid s_k)$,
    $s_{k+1} \sim P(\cdot \mid s_k, a_k)$,
    $r_k \sim R(\cdot \mid s_k, a_k)$.

    Update action-value parameter:
    \[ \theta_{k+1} = \theta_k + \alpha_k \xi(s_k, a_k)
      \left( r_k + \gamma \max_{b \in \Cal{A}} Q_{\theta_k}(s_{k+1}, b)
    - Q_{\theta_k}(s_k, a_k) \right) \]
  }
  Define $\wt{\pi}_K$ as the greedy policy w.r.t.
  $\wt{Q}_K \defeq Q_{\theta_{K+1}}$.

  \KwOut{An estimator $\wt{Q}_K$ of $Q^*$ and policy $\wt{\pi}_K$}
  \label{alg:QLlinear}
\end{algorithm}
\end{figure}

In order to understand the results of the analysis of \cref{alg:QLlinear}
found in [MR],
we need to define some concepts from ergodic theory.

Let $\kappa : \Cal{X} \leadsto \Cal{X}$ be a transition kernel.
Let $\mathfrak{P} = \kappa^\infty : \Cal{X} \leadsto \Cal{X}^\infty$.
And denote by
$\mathfrak{P}_x = \mathfrak{P}\delta_x \in \Cal{P}(\Cal{X}^\infty)$
the probability measure for the process starting at $x \in \Cal{X}$.
Let $\rho_i : \Cal{X}^\infty \to \Cal{X}$ be projection on the
$i$th space.
Define for any $A \in \Sigma_{\Cal{X}}$ the function
$\tau_A : \Cal{X}^\infty \to \ol{\N} = \inf\{ i \in \N \mid \rho_i \in A \}$.
Intuitively this function records the earliest time where the process
enter the set $A \subseteq \Cal{X}$.
Define the function
$\eta_A : \Cal{X}^\infty \to \ol{\N} = \sum_{i \in \N} 1_A \circ \rho_i$.
This function records the total number of times in which the process is
inside the set $A$.
Let $\varphi \in \Cal{P}(\Cal{X})$ be a probability measure on $\Cal{X}$.

\begin{defn}[Invariant measure]
  A countably additive measure $\mu \in \Cal{P}(\Cal{X})$ is said
  to be \defemph{invariant} w.r.t $\kappa$ if $\kappa \circ \mu = \mu$.
\end{defn}

\begin{defn}[Positivity]
  \leavevmode

  $\mathfrak{P}$ is called \defemph{positive} if it admits an $\kappa$-invariant
  probability measure $\mu$.
\end{defn}

\begin{defn}[Irreducibility]
  $\mathfrak{P}$ is called $\varphi$-irreducible
  $\mathfrak{P}_x(\tau_A < \infty) > 0$
  for all $A \in \Sigma_\Cal{X}$
  with $\varphi(A) > 0$
  and all $x \in \Cal{X}$.
\end{defn}

\begin{defn}[Harris recurrency]
  $\mathfrak{P}$ is called $\varphi$-Harris recurrent if
  it it $\varphi$-irreducible and
  $\mathfrak{P}_x(\eta_A = \infty) = 1$ for all $A \in \Sigma_\Cal{X}$ with
  $\varphi(A) > 0$ and all $x \in \Cal{X}$.
\end{defn}

\begin{defn}[Geometric ergodicity]
  A Markov process $\mathfrak{P}$ is called \defemph{geometrically ergodic} if
  it is positive with invariant measure $\mu$, $\varphi$-Harris recurrent
  for some $\varphi \in \Cal{P}(\Cal{X})$ and $\exists t>1$ such that
  \[ \sum_{i=1}^\infty t^i \norm{P^n_x - \mu}_{TV} < \infty,
  \quad \forall x \in \Cal{X} \]
\end{defn}

Since the $P,R$ of our MDP is reward independent we can view the
MDP as a stationary process $\mathfrak{P}$ on $\Cal{S}$
generated by kernel $P\pi$ for a policy $\pi \in S\Pi$.

\begin{thm}[Melo, Rebeiro]
  Let $(\Cal{S}, \Cal{A}, P, R, \gamma)$ be an MDP as of \cref{sett:MR}.
  Let $\pi \in S\Pi$ be a stationary process
  and $\mathfrak{P}$ the process kernel derived by $P\pi$.
  Assume that $\mathfrak{P}$ is geometrically ergodic with invariant
  measure $\mu$ and that
  $\pi(a \mid s) > 0$ for all $a \in \Cal{A}$ and $\mu_\pi$-almost all
  $s \in \Cal{S}$.
  Assume that $\sum_{i=1}^M \abs{\xi_i} \leq 1$.
  Then if \cref{alg:QLlinear} is run with learning rates from a sequence
  $\{ \alpha_k \}_{k \in \N}$ satisfying $\alpha_k \in [0,1]$ and
  \[ \sum_{k = 1}^\infty \alpha_k = \infty, \qquad
  \sum_{k = 1}^\infty \alpha_k^2 < \infty \]
  we have that
  \[ \theta_k \to \theta^* \]
  with probability 1, and $Q_{\theta^*}$ satisfies
  \[ Q_{\theta^*} = \rho_\Cal{Q} T Q_{\theta^*} \]
  Furthermore the orthogonal projection is expressible as
  \[ \rho_\Cal{Q} Q = \xi^T
    \frac{\E_{\pi\mu}\left( \xi Q \right)}{\E_{\pi\mu} (\xi \xi^T)}
  \]
  (recall that $\pi \mu(S \times A)
  = \int_S \pi(A \mid s) \difd \mu(s)$).
\end{thm}




\section{Deep fitted Q-iteration}
\subsection{Introduction}

\subsubsection{Differences in notation}
% Notational differences between this paper
% and [YangXieWang]

Because $\sigma$ is used ambigously in \cref{thm:main}
we denote the probability distribution $\sigma$
from \ncite{F20} p. 20 by $\nu$ instead.
I avoid the shorthand defined in
\ncite{F20} p. 26 bottom:
$\norm{f}_n^2 = 1/n \cdot \sum_{i=1}^n f(X_i)^2$.
and use $p$-norms instead.
The conversion to the notation used here becomes
$\norm{f}_n \leadsto \norm{f}/n$.
The letter $r$ is used in \ncite{F20} to denote the euclidean dimension of
the state space, while here we use $w$.

\subsubsection{The decision model}

\begin{sett}[Fan et al.]
  \leavevmode
  \begin{enumerate}
    \item We're considereing an MDP (\cref{sett:MDP}).
      That is a state and action space
      $(\Cal{S}, \Cal{A})$ and a transition and reward kernel $P, R$
      which only depends on the previous state-action pair.
    \item $S \subseteq \R^w$ is a compact subset of a euclidean space.
    \item $\Cal{A}$ is finite.
    \item Discounted factor satisfy $0 < \gamma < 1$.
  \end{enumerate}
\end{sett}

%todo establish connection to Bertsekas model, e.g. is this u.s.c?


\subsubsection{ReLU Networks}
\begin{defn}[Sparse ReLU Networks]
  For $s,V \in \R$ a (s,V)-\defemph{Sparse ReLU Network} is an ANN $f$
  with all activation functions being \emph{ReLU}
  i.e. $\sigma_{ij} = \max(\cdot, 0)$
  and with weights $(W_\ell, v_\ell)$ satisfying
  \begin{multicols}{3}
    \begin{itemize}
      \item $\max_{\ell \in [L+1]} \norm{\widetilde{W}_\ell}_\infty \leq 1$
      \item $\sum_{\ell = 1}^{L+1} \norm{\widetilde{W}_\ell}_0 \leq s$
      \item $\max_{j \in [d_{L+1}]} \norm{f_j}_\infty \leq V$
    \end{itemize}
  \end{multicols}
  Here $\widetilde{W}_\ell = (W_\ell, v_\ell)$.
  The set of them we denote $\Cal{F}\left(L, \{d_i\}_{i=0}^{L+1},s,V \right)$.
  \label{def:sparseReLU}
\end{defn}
The idea to work with this particular subclass of neural networks come from
\ncite{SH17}, which establishes the following lemma

\begin{lem}[Approximation of Hölder Smooth Functions by ReLU networks]
    Let $m,M \in \Z_+$ with $N \geq \max\{(\beta + 1)^r, (H + 1) e^r\}$,
    $L = 8 + (m + 5) (1 + \ceil{\log_2(r + \beta)})$, 
    $d_0 = r, d_j = 6(r + \ceil{\beta}) N, d_{L+1} = 1$.
    Then for any $g \in \Cal{C}_r \left( [0,1]^r, \beta, H \right)$
    there exists a ReLU network
    $f \in \Cal{F}\left(L, \{d_j\}_{j=0}^{L+1}, s, \infty \right)$
    with $s \leq 141 (r + \beta + 1)^{3 + r} N (m+6)$
    such that
    \begin{equation*}
      \norm{f - g}_\infty \leq (2 H + 1) 6^r N (1 + r^2 + \beta^2) 2^{-m}
      + H 3^{\beta} N^{-\beta/r}
    \end{equation*}
    \label{lem:holderapprox} 
    %todo: decide whether to do this proof
  \end{lem} 
  \vspace*{-\baselineskip}

\subsubsection{Fitted Q-Iteration}
The algorithm analysed by [Fan et al] is
\begin{figure}[H]
\begin{algorithm}[H] %\label{algocf:fq} % this labels line, could not fix
  \caption{Fitted Q-Iteration Algorithm}
  \KwIn{MDP $(\Cal{S}, \Cal{A}, P, R, \gamma)$, function class $\Cal{F}$,
    sampling distribution $\nu$, number of iterations $K$,
  number of samples $n$, initial estimator $\widetilde{Q}_0$}
  \For{$k = 0,1,2,\dots,K-1$}{
    Sample i.i.d. observations $\{(S_i, A_i), i \in [n]\}$ from $\nu$
    obtain $R_i \sim R(S_i, A_i)$ and $S'_i \sim P(S_i, A_i)$ \\
    Let $Y_i = R_i + \gamma \cdot \max_{a \in \Cal{A}} \widetilde{Q}_k(S'_i, a)$ \\
    Update action-value function:
    \[ \widetilde{Q}_{k+1} \leftarrow
      \argmin_{f \in \Cal{F}} \frac{1}{n}
    \sum_{i=1}^n (Y_i - f(S_i, A_i))^2 \]
  }
  Define $\pi_K$ as the greedy policy w.r.t. $\widetilde{Q}_K$ \\
  \KwOut{An estimator $\widetilde{Q}_K$ of $Q^*$ and policy $\pi_K$}
  \label{alg:fqi}
\end{algorithm}
\end{figure}



\subsection{Notational deviations from [TODO ref YangXieWang]}
% Notational differences between this paper
% and [YangXieWang]

Because $\sigma$ is used ambigously in \cref{thm:main}
we denote the probability distribution $\sigma$
from [YangXieWang, thm. 6.2, p. 20] by $\nu$ instead.

I avoid the shorthand defined in
[YangXieWang, p. 26 bottom]:
$\norm{f}_n^2 = 1/n \cdot \sum_{i=1}^n f(X_i)^2$.
and use $p$-norms instead.
The conversion to my notation thus becomes
$\norm{f}_n \leadsto \norm{f}/n$.

\subsection{Assumptions}


\subsubsection{Hölder Smoothness} %todo spell Holder properly
\begin{defn}[Hölder smoothness]
  For $f : \Cal{S} \to \R$ we define
  \begin{equation}
    \norm{f}_{C_w} \defeq 
    \sum_{|{\alpha}| < \beta}
    \norm{\partial^{\alpha}f}_\infty +
    \sum_{\norm{{\alpha}}_1 = \floor{\beta}}
    \sup_{x \neq y} \frac{|\partial^\alpha (f(x) - f(y))|}
  {\norm{x-y}_\infty^{\beta-\floor{\beta}}}
  \end{equation}
  Where $\alpha = (\alpha_1, \dots, \alpha_w) \in \N_0^w$.
  And $\partial^k$ is the partial derivative w.r.t. the $k$th variable.
  If $\norm{f}_{C_w} < \infty$ then $f$ is \defemph{Hölder smooth}.
  Given a compact subset $\Cal{D} \subseteq \R^w$
  the space of Hölder smooth functions on $\Cal{D}$ with norm bounded by
  $H > 0$ is denoted
  \[ C_w(\Cal{D}, \beta, H) \defeq
  \left\{ f : \Cal{D} \to \R \Mid \norm{f}_{C_w} \leq H \right\} \]
\end{defn}

\begin{defn}
  Let $t_j, p_j \in \N$, $t_j\leq p_j$ and $H_j, \beta_j > 0$ for $j \in [q]$.
  We say that $f$ is a \defemph{composition of Hölder smooth functions} when
  \[ f = g_q \circ \dots \circ g_1 \]
  for some functions $g_j : [a_j, b_j]^{p_j} \to [a_{j+1}, b_{j+1}]^{p_{j+1}}$
  that only depend on $t_j$ of their inputs
  for each of their components $g_{jk}$,
  and satisfies $g_{jk} \in C_{t_j}([a_j, b_j]^{t_j}, \beta_j, H_j)$, 
  i.e. they are Holder smooth.
  We denote the class of these functions
  \[ \Cal{G}(\{p_j, t_j, \beta_j, H_j\}_{j \in [q]}) \]
\end{defn}

\begin{defn}
  Define
  \[ \Cal{F}_0 = \left\{ f : \Cal{S} \times \Cal{A} \to \R \Mid
  f(\cdot, a) \in \Cal{F}(s, V) \; \forall a \in \Cal{A} \right\} \]
  and
  \[ \Cal{G}_0 = \left\{ f : \Cal{S} \times \Cal{A} \to \R
      \Mid f(\cdot, a) = \Cal{G}(\{p_j, t_j, \beta_t, H_j\}_{j \in [q]})
  \; \forall a \in \Cal{A} \right\} \]
\end{defn}

\begin{asm}\label{asm:A1}
  $ T \Cal{F}_0 \subseteq \Cal{G}_0$.
  I.e. t is assumed that $T f \in \Cal{G}_0$ for any $f \in \Cal{F}_0$, 
  so when using the Bellman optimality operator on our sparse ReLU networks,
  we should stay in the class of compositions of Holder smooth functions.
\end{asm}

If also $\Cal{G}_0$ is well approximated by functions in $\Cal{F}_0$
then this assumption implies that $\Cal{F}_0$ is approximately closed
under the Bellman operator $T$ and thus that $Q^*$ is close to $\Cal{F}_0$.

We now look at a simple example where \cref{asm:A1} holds:
Seting $\Cal{D}=[0,1]^r$, $q=1$ 
and taking both the expected reward function and transition kernel
to be Hölder smooth.

\begin{example}
  Assume for all $a \in \Cal{A}$ that
  $P(\cdot \mid s,a)$ is absolutely continuous w.r.t. $\lambda^k$
  (the $k$ dimensional Lebesgue measure)
  with density $p(\cdot \mid s, a)$,
  that for all $s' \in \Cal{S}$ we have
  $s \mapsto p\left(s' \Mid s, a \right)$
  and $s \mapsto r(s, a)$ are both Hölder smooth in the class
  $C_w([0,1]^r, \beta, H)$.
  Then
  \[ T \Cal{F}_0 \subseteq C_w([0,1]^r, \beta, (1 + \gamma V_{\max}) H) \] 
  To see this let
  Let $f \in \Cal{F}_0$ and $\alpha \in \N_0^w$.
  Observe that
  \begin{align*}
    \partial^\alpha (Tf)(s, a)
    = & \; \partial^\alpha_s \left( r(s, a) \right)
    + \gamma \int_{\Cal{S}}\partial^\alpha_s \left[ \max_{a' \in \Cal{A}}
    f(s', a') p\left(s' \Mid s, a\right) \right] \difd s' 
    \\ \leq & \; \partial^\alpha_s \left( r(s, a) \right)
    + \gamma V_{\max} \sup_{s' \in \Cal{S}} \partial_s^\alpha
    p\left(s' \Mid s, a\right)
  \end{align*}
  similarly
  \begin{align*}
    \partial^\alpha (Tf)(s, a) - \partial^\alpha (Tf)(s', a)
    \leq & \; \partial^\alpha_s \left( r(s, a) \right)
    - \partial^\alpha_s \left( r(s', a) \right)
    \\ & \; + \gamma V_{\max} \sup_{s'' \in \Cal{S}}
    \left( \partial_s^\alpha p(s'' \Mid s, a)
    - \partial_s^\alpha p(s'' \Mid s', a) \right)
  \end{align*}
  Thus
  \begin{align*}
    \norm{Tf}_{C_w} \leq & \; \sum_{\abs{\alpha}<\beta} \left(
      \norm{\partial^\alpha r(\cdot, a)}_\infty
      + \gamma V_{\max} \sup_{s \in \Cal{S}} \norm{\partial^\alpha
    p(s \mid \cdot, a)}_\infty \right)
    \\ + & \; \sum_{\norm{\alpha}_1 = \floor{\beta}} \sup_{x \neq y}
    \left(
      \frac{\abs{\partial^\alpha (r(x, a) - r(y, a))}}
      {\norm{x - y}_{\infty}^{\beta - \floor{\beta}}}
      + \gamma V_{\max} \sup_{s \in \Cal{S}} \frac{
      \abs{\partial^\alpha (p(s \mid x, a) - p(s \mid y, a))}}
      {\norm{x - y}_{\infty}^{\beta - \floor{\beta}}}
    \right)
    \\ \leq & \; H + \gamma V_{\max} H = (1 + \gamma V_{\max}) H
  \end{align*}
  
\end{example}


\subsubsection{Concentration coefficients}

\begin{defn}[Concentration coefficients] \label{defn:ccoefs}
  Let $\nu_1, \nu_2 \in \Cal{P}(\Cal{S}\times \Cal{A})$ be probability measures,
  absolutely continuous w.r.t. $\lambda^w \otimes \mu_\Cal{A}$
  (the product of the $w$-dimensional Lebesgue measure and the counting measure
  on $\Cal{A}$).
  Define
  \[ \kappa(m, \nu_1, \nu_2) = \sup_{\pi_1, \dots, \pi_m}
    \left[ \E_{v_2} \left( \frac{\mathrm{d} (P_{\pi_m} \dots P_{\pi_1} \nu_1)}
  {\mathrm{d} \nu_2} \right)^2 \right]^{1/2} \]
\end{defn}

\begin{asm}\label{asm:A2}
  Let $\nu$ be the sampling distribution from the algorithm, and $\mu$ the distribution
  over which we measure the error in the main theorem, then we assume
  \[ (1 - \gamma)^2 \sum_{m\geq 1} \gamma^{m-1} m \kappa(m, \mu, \nu)
  = \phi_{\mu, \nu} < \infty \]
\end{asm}



\subsection{Main theorem}
\begin{thm}[Yang, Xie, Wang] \label{thm:main}
    Let $\mu$ be any distribution over $\Cal{S} \times \Cal{A}$.
  Make \cref{asm:A1} and \cref{asm:A2} with the
  constants $\phi_{\mu, \nu} > 0$, $q \in \N$ and
  $\{p_j, t_j, \beta_j, H_j\}_{j \in [q]}$. Furthermore assume
  that there exists a constant $\xi > 0$ such that
  \[ \max \left\{ \sum_{j=1}^q (t_j + \beta_j + 1)^{3 + t_k},
      \sum_{j=1}^q \log (t_j + \beta_j),
      \max_{j \in [q]} p_j
  \right\} \leq (\log n)^\xi \] 
  Set $\beta^*_j = \beta_j \prod_{\ell = j+1}^q \min(\beta_\ell, 1)$
  for $j\in [q-1]$, $\beta^*_q = 1$,
  $\alpha^* = \max_{j \in [q]} t_j/(2\beta^*_j + t_j)$, 
  $\xi^* = 1 + 2\xi$ and $\kappa^* = \min_{j\in [q]} \beta^*_j/t_j$.
  Then there exists a class of ReLU networks
  \[ \Cal{F}_0 = \{f : \Cal{S} \times \Cal{A} \to \R : f(\cdot, a) \in 
  \Cal{F}(\wt{L}, \{\wt{d}_j\}_{j=0}^{\wt{L}+1},\wt{s}) \mid a \in \Cal{A} \} \]
  with structure satisfying
  \[ \wt{L} \lesssim (\log n)^{\xi^*},
    \wt{d}_0 = r, \wt{d}_j \leq 6 n^{\alpha^*} (\log n)^{\xi^*},
  d_{L+1} = 1, \wt{s} \lesssim n^{\alpha^*} \cdot (\log n)^{\xi^*} \]
  such that when running \cref{alg:fqi} with $\Cal{F}_0$
  and $n$ is sufficiently large
  \[ \norm{Q^* - Q^{\pi_K}}_{1, \mu} \leq \;
    C_{\varepsilon} \frac{\phi_{\mu, \nu} \gamma}{(1-\gamma)^2} V_{\max}^2
    n^{\max\{-2\alpha^*\kappa^*, (\alpha^*  - 1)/2 \}} \log(n)^{1+2\xi^*}
    + \frac{4 \gamma}{(1-\gamma)^2} R_{\max} \gamma^K
  \]
  where $C_{\varepsilon}>0$ is a constant not depending on $n$ or $K$.
\end{thm}



\subsection{Proofs}
The proof of \cref{thm:main} combines two results.
The first on the error propagation and the second on the error
ocurring in a single step.

\begin{thm}[Error Propagation]\label{thm:errorprop}
	Let $\{\wt{Q}_i\}_{0\leq i\leq K}$ be the iterates of the fitted Q-iteration algorithm.
	Then
	\begin{equation*}
	  \norm{Q^* - Q^{\pi_K}}_{1, \mu}
	  \leq \frac{2 \phi_{\mu,\nu} \gamma}{(1-\gamma)^2} \cdot \varepsilon_{\max}
	  + \frac{4 \gamma^{K+1}}{(1-\gamma)^2} \cdot R_{\max}
	\end{equation*}
	Where
	\[ \varepsilon_{\max}
	= \max_{k \in [K]} \norm{T\wt{Q}_{k-1} - \wt{Q}_k}_{2,\nu} \] 
\end{thm}


\begin{thm}[One-step Approximation Error]\label{thm:oneStep}
  Let
  \begin{itemize}
    \item $\Cal{F} \subseteq \Cal{B}(\Cal{S}\times\Cal{A}, V_{\max})$
      be a class ofbounded measurable functions
    \item $\nu \in \Cal{P}(\Cal{S},\Cal{A})$ be a probability measure
    \item $(S_i, A_i)_{i\in[n]}$ be $n$ i.i.d. samples following $\nu$
    \item $(R_i, S'_i)_{i\in[n]}$ be the rewards and next states
    	corresponding to the samples
    \item $Q \in \Cal{F}$ be fixed
    \item $Y_i = R_i + \gamma \max_{a \in \Cal{A}} Q(S'_i, a)$
    \item $\wh{Q} = \argmin_{f \in \Cal{F}} \frac{1}{n}
      \sum_{i=1}^n (f(S_i, A_i) - Y_i)^2$
    \item $\epsilon \in (0,1],\; \delta > 0$ be fixed
    \item $\Cal{N}(\delta, \Cal{F}, \norm{\cdot}_\infty)$
      a minimal $\delta$-covering of $\Cal{F}$ w.r.t. $\norm{\cdot}_\infty$
    \item $N_\delta = \abs{\Cal{N}(\delta, \Cal{F}, \norm{\cdot}_\infty)}$
      the number of elements in this covering
  \end{itemize}
  Then
  \[(1+\epsilon)^2+\omega(\Cal{F})+C\cdot V_{\max}^2/(n+\epsilon)\cdot N_\delta
  + C' \cdot V_{\max} \cdot \delta \]
  where $C = 64, C'=8$ and
  \[ \omega(\Cal{F})
  = \sup_{g\in \Cal{F}} \inf_{f \in \Cal{F}} \norm{f - T g}_{2, \nu}^2 \]
\end{thm}

The proofs of \cref{thm:errorprop} and \cref{thm:oneStep} are
found below, but first 
we will show how to combine them to obtain
\cref{thm:main}.





\begin{proof}[Proof of main theorem] %todo write intro
  Using \cref{thm:errorprop} we get
  \begin{equation}
    \norm{Q^* - Q^{\pi_K}}_{1,\mu} \leq
    \frac{2 \phi_{\mu, \nu} \gamma}{(1-\gamma)^2} +
    \frac{4 \gamma^{K+1}}{(1-\gamma)^2} R_{\max}
    \label{eq:mp1}
  \end{equation}
  where $\varepsilon_{\max} =
  \max_{k \in [K]} \norm{T \wt{Q}_{k-1} - \wt{Q}_k}_{2, \nu}$.
  %todo: specify constants for \Cal{F}_0 !
  Using \cref{thm:oneStep} with $Q = \wt{Q}_{k-1}$,
  $\Cal{F} = \Cal{F}_0$, $\epsilon = 1$ and $\delta = 1/n$, we get
  \begin{align}
    \varepsilon_{\max} \leq & 6 n^{-1} C_2^2 V_{\max}^2 \log(N_\delta)
    + 2 \omega(\Cal{F}_0)
    + 8 \sqrt{2} V_{\max} n^{-1/2} \sqrt{\log N_0}
    + 16 V_{\max} n^{-1}
    \label{eq:mp2}
  \end{align}
  where $N_0 = \abs{\Cal{N}(1/n, \Cal{F}_0, \norm{\cdot}_\infty)}$.
  The remains only to wound $\omega(\Cal{F}_0)$ and $N_0$,
  starting with $\omega(\Cal{F}_0)$.

  \textbf{Step 1}. %todo: proper ref below
  We want to employ the following lemma by [Schmidt-Hieber 2019, thm. 5, p. 22]
  \begin{lem}[Approximation of Hölder Smooth Functions]
    Let $m,M \in \Z_+$ with $N \geq \max\{(\beta + 1)^r, (H + 1) e^r\}$,
    $L = 8 + (m + 5) (1 + \ceil{\log_2(r + \beta)})$, 
    $d_0 = r, d_j = 6(r + \ceil{\beta}) N, d_{L+1} = 1$.
    Then for any $g \in \Cal{C}_r \left( [0,1]^r, \beta, H \right)$
    there exists a ReLU network
    $f \in \Cal{F}\left(L, \{d_j\}_{j=0}^{L+1}, s, \infty \right)$
    with $s \leq 141 (r + \beta + 1)^{3 + r} N (m+6)$
    such that
    \begin{equation*}
      \norm{f - g}_\infty \leq (2 H + 1) 6^r N (1 + r^2 + \beta^2) 2^{-m}
      + H 3^{\beta} N^{-\beta/r}
    \end{equation*}
    \label{lem:holderapprox} 
    %todo: decide whether to do this proof
  \end{lem} 
  \vspace*{-\baselineskip}
  to each Hölder smooth part of $g$ and then piece it together somehow,
  using that ReLU networks are easily stitched together into bigger
  ReLU networks.
  Therefore the first step is to refit our
  Hölder Smooth compositions in $\Cal{G}_0$ to be defined on a hyper-cube instead.
  This is a relatively simple procedure:

  Let $f \in \Cal{G}_0$ then $f(\cdot, a) \in
  \Cal{G}(\{p_j, t_j, \beta_j, H_j\})$ for all $a \in \Cal{A}$.
  Therefore $f(\cdot, a) = g_q \circ \dots \circ g_1$ where
  the (sub-)components $(g_{jk})_{k=1}^{p_{j+1}} = g_j$ satisfy
  \begin{equation}
    g_{jk} \in C_{t_j}([a_j, b_j]^{t_j}, \beta_j, H_j)
    , \qquad j \in [q], k \in [p_{j+1}]
  \end{equation}
  Here $a_1 = 0, b_1=1$ and,
  $a_j < b_j \in \R$ are some real numbers for $2 \leq j \leq q$.
  Notice that the Hölder smooth condition implies that
  $g_{jk}([a_j, b_j]^{t_j}) \subseteq [-H_j, H_j]$.
  Define
  \begin{align}
    h_1 = & g_1/(2H_1) + 1/2 \notag
    \\ h_j(u) = & g_j(2H_{j-1} u - H_{j-1})/(2H_j) + 1/2,
    & j \in \{2, \dots, q-1\} \notag
    \\ h_q(u) = & g_q(2H_{q-1}u - H_{q-1})
  \end{align}
  Then $g_q \circ \dots \circ g_1 = h_q \circ \dots \circ h_1$ and
  \begin{align}
    h_{1k} \in &\; C_{t_1}([0,1]^{t_1}, \beta_1, 1) \notag
    \\ h_{jk} \in &\; C_{t_j}([0,1]^{t_j}, \beta_j, (2H_{j-1})^{\beta_j}),
    & j \in \{2, \dots, q-1\} \notag
    \\ h_q \in &\; C_{t_q}([0,1]^{t_q}, \beta_q, H_q(2H_{q-1})^{\beta_q})
  \end{align}
  Define %todo: define N
  $\eta = \log\left((2W + 1) 6^{t_j} N / (W 3^{\beta_j} N^{-\beta_j/t_j}) \right)$,
  and $m = \eta \ceil{\log_2 n}$ 
  \begin{equation}
    W \defeq \max \left( \left\{ (2 H_{j-1})^{\beta_j} \mid 1\leq j\leq q-1 \right\}
    \cup \left\{ H_q(2H_{q-1})^{\beta_q}, 1\right\} \right)
  \end{equation}
  By \cref{lem:holderapprox} there exists a ReLU network
  \begin{equation}
    \wh{h}_{jk} \in \Cal{F} \left( L_j + 2, \left\{ t_j, \wt{d}_j p_{j+1}, \dots,
    \wt{d}_j p_{j+1}, p_{j+1} \right\}, (\wt{s}_j + 4) \cdot p_{j+1} \right)
  \end{equation}
  where $\wt{d}_j = 6(t_j + \ceil{\beta_j})N$ and
  $\wt{s}_j \leq 141 (t_j + \beta_j + 1)^{3 + t_j} N (m + 6)$
  such that
  \begin{equation}
    \norm{\wh{h}_{jk} - h_{jk}}_\infty \leq (2 W + 1) 6^{t_j} N 2^{-m}
    + W 3^{\beta_j} N^{-\beta_j/t_j} \leq 2 W 3^{-\beta_j} N^{-\beta_j/t_j}
    \label{eq:hhatbound}
  \end{equation} 
  since $n\leq 4 > e$.
  Since $h_{j+1}$ is defined on $[0, 1]^{t_{j+1}}$ but $\wt{h}_j$ takes values
  in $\R$ we need to restrict $\wt{h}_j$ somehow to stitch
  the two together (by function composition). This is easily done by
  \begin{lem}
    Restriction to $[0, 1]$ is expressible as a two-layer ReLU network
    with 4 non-zero weights.
  \end{lem}
  \begin{proof}
    Namely $\tau(u) = 1 - (1 - u)_+ = \min \left\{
    \max \left\{ u, 0 \right\}, 1 \right\}$. %todo elaborate
  \end{proof}
  Now define
  $\wt{h}_{jk} = \tau \circ \wh{h}_{jk}$
  (and $\wt{h}_{j} = (\wt{h}_{jk})_{k \in [p_{j + 1}]}$).
  Then
  \begin{equation}
    \wt{h}_{jk} \in \Cal{F}\left( L_j + 2, \left\{ t_j,
  \wt{d}_j, \dots, \wt{d}_j, 1 \right\}, (\wt{s}_j + 4) p_{j+1} \right)
  \end{equation}
  and since $h_{jk}([0, 1]^{t_j}) \in [0, 1]$ by \cref{eq:hhatbound}
  \begin{align}
    \norm{\wt{h}_{jk} - h_{jk}}_\infty
    = &\; \norm{\tau \circ \wh{h}_{jk} - \tau \circ h_{jk}}_\infty
    \\ \leq &\; \norm{\wh{h}_{jk} - h_{jk}}_\infty
    \\ \leq &\; 2 W 3^{-\beta_j} N^{-\beta_j/t_j}
  \end{align}

  \textbf{Step 2}.
  Now define $\wt{f}:\Cal{S} \to \R$
  as $\wt{f} = \wt{h}_1 \circ \dots \circ \wh{h}_1$.
  If we set
  $\wt{L} \defeq \sum_{j=1}^q (L_j + 2)$,
  $\wt{d} \defeq \max{j \in [q]} \wt{d}_j p_{j+1}$
  and $\wt{s} \defeq \sum_{j=1}^q (\wt{s}_j + 4) p_{j+1}$.
  Then $\wt{f} \in \Cal{F}\left( \wt{L}, \left\{ 
  r, \wt{d}, \dots, \wt{d}, 1\right\}, \wt{s} \right)$.
  %TODO: continue from here

\end{proof}

Now for \cref{thm:errorprop}.

Before we proceed to prove \cref{thm:errorprop},
we will establish a couple of lemmas.

\begin{lem}\label{lem:tlemma}
  $T Q \geq T_\pi Q$ for any policy $\pi : \Cal{S} \to \Cal{P}(\Cal{A})$
  and any action value function $Q: \Cal{S} \times \Cal{A} \to \R$.
\end{lem}
\begin{proof}
  This is an easy consequence of the definitions (\cref{defn:opQ})
  \begin{align*}
    (TQ)(s, a) &= r(s, a) + \gamma \int \max_{a'} Q(s', a')
    \difd P(s' \mid s, a)
    \\ &\geq r(s, a) + \gamma \int \int Q(s', a'')
    \difd \pi(a'' \mid s') \difd P(s' \mid s, a)
    \\ &= T_\pi Q(s,a)
  \end{align*}
  since $\max_{a'} Q(s', a') \geq Q(s', a'')$ for any $a'' \in \cl{A}$.
\end{proof}

The next lemma (last before proof of \ref{thm:oneStep}) is about the
relation between the next-step operator $P_\tau$ and the
concentration coefficients.

We recall here some details regarding composition of kernels and measures
discussed in \cref{rem:altComp}.
A stationary policy $\tau : \cl{S} \leadsto \cl{A} \in S\Pi$ composed
with the transition kernel $P : \cl{S} \times \cl{A} \leadsto \cl{S}$
yields a kernel $\tau P : \cl{S} \times \cl{A} \leadsto \cl{S} \times \cl{A}$.
The $\circ$-composition of kernels is forgets histories and so if
$\tau' \in S\Pi$ we have that 
$(\tau' P) \circ (\tau P) : \cl{S} \times \cl{A} \leadsto \cl{S} \times \cl{A}$.
Lastly the kernel-measure $\circ$-composition by a probability measure
$\mu \in \cl{P}(\cl{S} \times \cl{A})$ we have that
$(\tau P) \circ \mu \in \cl{P}(\cl{S} \times \cl{A})$.

\begin{lem}\label{lem:MRN}
  Let $f:\Cal{S}\times\Cal{A} \to \R$ be an action-value function,
  $\tau_1, \dots, \tau_m$ be policies
  and $\mu \in \Cal{P}(\Cal{S}\times\Cal{A})$ be a probability measure.
  Then
  \begin{align*}
    \E_\mu [(P_{\tau_m} \dots P_{\tau_1})(f)]
    \leq \kappa(k - i + j; \mu, \nu) \norm{f}_{2,\nu}
  \end{align*}
  For any measure $\nu \in \Cal{P}(\Cal{S}\times\Cal{A})$ which is
  absolutely continuous w.r.t.
  $(\tau_m P) \circ \dots \circ (\tau_1 P) \circ \mu$.
  Here $\kappa$ is the concentration coefficients defined in \cref{defn:ccoefs}.
\end{lem}
\begin{proof} \label{proof:C1}
  Recall that
  \begin{align*}
    \kappa(m; \mu, \nu) &\defeq \sup_{\pi_1, \dots, \pi_m \in S\Pi} \left[
      \E_{\nu} \abs{\frac{\dif\; ((\pi_m P) \circ \dots \circ (\pi_1 P) \circ \mu)}
  {\dif \nu}}^2 \right]^{1/2}
    \\ &= \sup_{\pi_1, \dots, \pi_m \in S\Pi}
    \norm{\frac{\dif\; ((\pi_m P) \circ \dots \circ (\pi_1 P) \circ \mu)}
    {\dif \nu}}_{2, \nu}
  \end{align*}
  Now
  \begin{align}
    \E_\mu [ P_{\tau_m} \dots P_{\tau_1} f]
    &= \int P_{\tau_{m-1}} \dots P_{\tau_2} f \difd \tau_m P \difd \mu
    \\ &= \int f \difd (\tau_1 P) \circ \dots \circ (\tau_m P) \circ \mu
    \label{eq:preradniko}
    \\ &= \int f \frac{\difd (\tau_1 P) \circ \dots \circ (\tau_m P) \circ \mu}
    {\dif \nu} \dif \nu
    \label{eq:postradniko}
    \\ &\leq \norm{\frac{(\tau_1 P) \circ \dots \circ (\tau_m P) \circ \mu}
    {\dif \nu}}_{2,\nu}
    \cdot \norm{f}_{2, \nu} \label{eq:cspost}
    \\ &\leq \kappa(m, \mu, \nu) \norm{f}_{2,\nu}
  \end{align}
  Where \cref{eq:postradniko} is due to the Radon-Nikodym theorem
  (\cref{thm:radonNiko})
  and \cref{eq:cspost} is Cauchy-Schwarz.
\end{proof}

We now turn to the proof of \cref{thm:errorprop}.

\begin{proof}[Proof of \cref{thm:errorprop}] 
  First some things to keep in mind during the proof.
  Recall that $V_{\max} = R_{\max} / (1 - \gamma)$ and that
  $\pi_Q$ is the greedy policy w.r.t. $Q$.
  Denote 
  \[ \pi_i = \pi_{\wt{Q}_i},
    \; Q_{i+1} = T \wt{Q}_{i},
  \; \varrho_{i} = Q_{i} - \wt{Q}_{i},
\; \mbox{ for } i \in \{0,\dots,K+1\} \]
  Note that for any policy $\pi$,
  $P_{\pi}$ is linear and 1-contrative on
  $\Cal{L}^\infty(\Cal{S} \times \Cal{A})$. %todo: proof
  Also \[ T_{\pi} Q_{\pi} = Q_{\pi}, \;
    T Q = T_{\pi_Q} Q, \;
  T Q^* = Q^* = Q_{\pi^*} \]
  where $\pi^*$ is greedy w.r.t. $Q^*$. 
  Also if $f,f':\Cal{S}\times \Cal{A} \to \R$ are measurable we have
  \begin{equation}
    f \geq f' \implies P_{\pi} f \geq P_{\pi} f'
    \label{eq:Pmonotone}
  \end{equation}
  
  The proof consists of four steps.

  \textbf{Step 1}
  We start by relating $Q^* - Q_{\pi_K}$, the quantity of interest,
  to $Q^* - \wt{Q}_K$, which is more related to the output of the algorithm.
  Using \cref{lem:tlemma} we can make the upper bound
  \begin{align}
    Q^* - Q_{\pi_K} &= T_{\pi^*} Q^* - T_{\pi_K} Q_{\pi_K} \notag
    \\ &= T_{\pi^*} Q^* + (T_{\pi^*} \wt{Q}_K - T_{\pi^*} \wt{Q}_K)
    + (T \wt{Q}_K - T \wt{Q}_K)- T_{\pi_K} Q_{\pi_K} \notag
    \\ &= (T_{\pi^*} \wt{Q}_K - T \wt{Q}_K)
    + (T_{\pi^*} Q^* - T_{\pi^*} \wt{Q}_K) 
    + (T \wt{Q}_K - T_{\pi_K} Q_{\pi_K}) \notag
    \\ &\leq (T_{\pi^*} Q^* - T_{\pi^*} \wt{Q}_K) 
    + (T \wt{Q}_K - T_{\pi_K} Q_{\pi_K}) \notag
    \\ &= (T_{\pi^*} Q^* - T_{\pi^*} \wt{Q}_K) 
    + (T_{\pi_K} \wt{Q}_K - T_{\pi_K} Q_{\pi_K}) \notag
    \\ &= \gamma P_{\pi^*} (Q^* - \wt{Q}_K)
    + \gamma P_{\pi_K} (\wt{Q}_K - Q_{\pi_K}) \notag
    \\ &= \gamma (P_{\pi^*} - P_{\pi_K})(Q^* - \wt{Q}_K)
    + \gamma P_{\pi_K} (Q^* - Q_{\pi_K}) \label{preU}
  \end{align}
  This implies
  \begin{equation} (I - \gamma P_{\pi_K})(Q^* - Q_{\pi_K})
    \leq \gamma (P_{\pi^*} - P_{\pi_K})(Q^* - \wt{Q}_K)
    \label{eq:beforeU}
  \end{equation}
  Since $\gamma P_{\pi_K}$ is $\gamma$-contractive,
  $U = (I - \gamma P_{\pi_K})^{-1}$ exists as a bounded operator on
  $\Cal{L}^\infty(\Cal{S}\times \Cal{A})$ and equals
  \begin{equation}
    U = \sum_{i=0}^\infty \gamma^i (P_{\pi_K})^i
    \label{eq:defnU}
  \end{equation}
  From \cref{eq:defnU} and \cref{eq:Pmonotone}
  we also see that $f \geq f' \implies U f \geq U f'$ for any
  $f, f' : \Cal{S}\times \Cal{A} \to \R$.
  Therefore we can apply $U$ on both sides of \cref{eq:beforeU} to obtain 
  \begin{equation} Q^* - Q_{\pi_K} \leq \gamma U (P_{\pi^*}(Q^* - \wt{Q}_K)
  - P_{\pi_K} (Q^* - \wt{Q}_K)) \label{eq:qq1} \end{equation} 

  \textbf{Step 2}
  Using \cref{lem:tlemma} for any $i \in [K]$ we can get an upper bound
  \begin{align}
    Q^* - \wt{Q}_{i+1} &= Q^* + (T\wt{Q}_i - T \wt{Q}_i) - \wt{Q}_{i+1}
    + (T_{\pi^*}\wt{Q}_i - T_{\pi^*}\wt{Q}_i) \notag
    \\ &= (Q^* - T_{\pi^*} \wt{Q}_i) + (T \wt{Q}_i - \wt{Q}_{i+1})
    + (T_{\pi^*}\wt{Q}_i - T \wt{Q}_i) \notag
    \\ &= (T_{\pi^*} Q^* - T_{\pi^*} \wt{Q}_i) + \varrho_{i+1}
    + (T_{\pi^*}\wt{Q}_i - T \wt{Q}_i) \notag
    \\ &\leq T_{\pi^*} Q^* - T_{\pi^*} \wt{Q}_i + \varrho_{i+1} \notag
    \\ &= \gamma P_{\pi^*} (Q^* - \wt{Q}_i) + \varrho_{i+1} \label{Qst_high}
  \end{align}
  and a lower bound
  \begin{align}
    Q^* - \wt{Q}_{i+1} &= Q^* + (T\wt{Q}_i - T \wt{Q}_i) - \wt{Q}_{i+1}
    + (T_{\pi_i}Q^* - T_{\pi_i}Q^*) \notag
    \\ &= (T_{\pi_i}Q^* - T_{\pi_i} \wt{Q}_i) + \varrho_{i+1}
    + (T Q^* - T_{\pi_i} Q^*) \notag
    \\ &\geq T_{\pi_i}Q^* - T_{\pi_i} \wt{Q}_i + \varrho_{i+1} \notag
    \\ &= \gamma P_{\pi_i} (Q^* - \wt{Q}_i) + \varrho_{i+1} \label{Qst_low}
  \end{align}
  Applying \cref{Qst_high} and \cref{Qst_low} iteratively we get  
  \begin{align}
    Q^* - \wt{Q}_K \leq \gamma^K (P_{\pi^*})^K (Q^* - \wt{Q}_0)
    + \sum_{i=0}^{K-1} \gamma^{K-1-i} (P_{\pi^*})^{K-1-i} \varrho_{i+1}
    \label{eq:Qsk_high}
  \end{align}
  and
  \begin{align}
    Q^* - \wt{Q}_K \geq \gamma^K (P_{\pi_{K-1}} \dots P_{\pi_0})(Q^* - \wt{Q}_0)
    + \sum_{i=0}^{K-1}\gamma^{K-1-i}(P_{\pi_{K-1}}\dots P_{\pi_{i+1}})
    \varrho_{i+1} \label{eq:Qsk_low}
  \end{align}

  \textbf{Step 3}
Combining \cref{eq:Qsk_high} and \cref{eq:Qsk_low} with \cref{eq:qq1} we get
  \begin{equation}
    \begin{split}
      Q^* - Q_{\pi_K} \leq U^{-1} \bigg(
	\gamma^{K+1}((P_{\pi^*})^{K+1} - P_{\pi_K} \dots P_{\pi_0})(Q^* - \wt{Q}_0)
	\\ + \sum_{i=0}^{K-1} \gamma^{K-i}
      ((P_{\pi^*})^{K-i} - P_{\pi_K} \dots P_{\pi_{i+1}}) \varrho_{i+1} \bigg)
    \end{split}
    \label{eq:step3_1}
  \end{equation}
  For shorthand define constants
  \begin{equation} \alpha_i = \frac{(1-\gamma) \gamma^{K-i-1}}{1 - \gamma^{K+1}}
    \; \mbox{ for } 0 \leq i \leq K-1 \mbox{ and }
    \alpha_K = \frac{(1-\gamma) \gamma^K}{1 - \gamma^{K+1}}
  \end{equation}
  (note that $\sum_{i=0}^K \alpha_i = 1$) and operators
  \begin{align}
    O_i = (1-\gamma)/2 U^{-1} [(P_{\pi^*})^{K-i}
    + (P_{\pi_K} \dots P_{\pi_{i+1}})]
    \\ O_K = (1-\gamma)/2 U^{-1} [(P_{\pi^*})^{K+1}
    + (P_{\pi_K} \dots P_{\pi_0})]
  \end{align}
  Then by \cref{eq:step3_1}
  \begin{align}
    \abs{Q^* - Q_{\pi_K}} \leq \frac{2 \gamma (1- \gamma^{K+1})}{(1-\gamma)^2}
    \left[ \sum_{i=0}^{K-1} \alpha_i O_i \abs{\varrho_{i+1}}
    + \alpha_K O_K \abs{Q^* - \wt{Q}_0} \right]
  \end{align}
  So by linearity of expectation
  \begin{align}
    \norm{Q^* - Q_{\pi_K}}_{1, \mu} &= \E_\mu \abs{Q^* - Q_{\pi_K}}
    \\ &\leq \frac{2 \gamma (1- \gamma^{K+1})}{(1-\gamma)^2}
    \left[ \sum_{i=0}^{K-1} \alpha_i \E_\mu (O_i \abs{\varrho_{i+1}})
    + \alpha_K \E_\mu (O_K \abs{Q^* - \wt{Q}_0}) \right]
    \label{eq:step3_final}
  \end{align}
  With the bound on rewards we (crudely) estimate
  \begin{equation}
    \E_\mu O_K \abs{Q^* - \wt{Q}_0} \leq 2 V_{\max} = 2 R_{\max} / (1-\gamma)
    \label{eq:step3_crude}
  \end{equation}
  The remaining difficulty lies in $\E_\mu(O_i\abs{\varrho_{i+1}})$.

  \textbf{Step 4}
  Using the sum expansion of $U$ we get
  \begin{align}
    \E_\mu (O_i \abs{\varrho_{i+1}})
    &= \frac{1 - \gamma}{2} \E_\mu \left( U^{-1} [(P_{\pi_K})^{K-i} +
    P_{\pi_K} \dots P_{\pi_{i+1}}] \abs{\varrho_{i+1}} \right)
    \\ &= \frac{1 - \gamma}{2} \E_\mu \left( \sum_{j=0}^\infty
      [(P_{\pi_K})^j (P_{\pi_K})^{K-i}
      + (P_{\pi_K})^{j+1} P_{\pi_{K-1}} \dots P_{\pi_{i+1}}]
    \abs{\varrho_{i+1}} \right)
    \\ &= \frac{1 - \gamma}{2} \sum_{j=0}^\infty
      \E_\mu \left( [(P_{\pi_K})^j (P_{\pi_K})^{K-i}
      + (P_{\pi_K})^{j+1} P_{\pi_{K-1}} \dots P_{\pi_{i+1}}] 
      \abs{\varrho_{i+1}} \right)
  \end{align}
  Notice that there are $K-i+j$ $P$-operators on both terms
  in the sum. Therefore were can employ \cref{lem:MRN} twice.
  Moreover define
  $\varepsilon_{\max} = \max_{i \in [K]} \norm{\varrho_i}_{2,\nu}$.
  Then
  \begin{align}
    \E_\mu(O_i \abs{\varrho_{i+1}}) &\leq (1-\gamma)
    \sum_{j=0}^\infty \gamma^j \kappa(K-i+j;\mu,\nu) \norm{\varrho_{i+1}}_{2,\nu}
    \notag
    \\ &\leq \varepsilon_{\max} (1-\gamma)
    \sum_{j=0}^\infty \gamma^j \kappa(K-i+j;\mu,\nu) 
    \label{eq:step4_1}
  \end{align} 
  Using \cref{eq:step3_final}, \cref{eq:step3_crude} and \cref{eq:step4_1}
  \begin{equation}
    \begin{split}
    \norm{Q^* - Q_{\pi_K}}_{1,\mu} \leq
    \frac{2 \gamma (1- \gamma^{K+1})}{1-\gamma} 
    \left[ \sum_{i=0}^{K-1} \sum_{j=0}^\infty
    \alpha_i \gamma^j \kappa(K-i+j; \mu, \nu) \right] \varepsilon_{\max}
    \\ + \frac{4 \gamma (1-\gamma^{K+1})}{(1-\gamma)^3} \alpha_K R_{\max}
  \end{split}
  \label{eq:step4_2}
  \end{equation} 
  Focusing on the first term on RHS of \cref{eq:step4_2}, if we 
  then we can take the norm out of the sum as a constant. We are left with
  \begin{align}
    & \sum_{i=0}^{K-1} \sum_{j=0}^\infty \alpha_i \gamma^j \kappa(K-i+j;\mu,\nu)
    \notag
    \\ &= \sum_{i=0}^{K-1} \sum_{j=0}^\infty  
    \frac{(1-\gamma) \gamma^{K-i+j-1}}{1-\gamma^{K+1}} \kappa(K-i+j;\mu,\nu)
    \notag
    \\ &= \frac{1-\gamma}{1-\gamma^{K+1}} \sum_{j=0}^\infty \sum_{i=0}^{K-1} 
    \gamma^{K-i+j-1} \kappa(K-i+j;\mu,\nu) \notag
    \\ &\leq \frac{1-\gamma}{1-\gamma^{K+1}} \sum_{m=0}^\infty
    \gamma^{m-1} \cdot m \cdot \kappa(m; \mu, \nu) \notag
    \\ &\leq \frac{1}{1-\gamma^{K+1}(1-\gamma)} \phi_{\mu,\nu}
    \label{eq:step4_3}
  \end{align}
  Where the last inequality is due to \cref{asm:A2}.
  
  Combining \cref{eq:step4_2} and \cref{eq:step4_3} we arrive at
  \begin{equation}
    \norm{Q^* - Q_{\pi_K}}_{1,\mu} \leq
    \frac{2\gamma \cdot \phi_{\mu,\nu}}{(1-\gamma)^2} \cdot \varepsilon_{\max}
    + \frac{4\gamma^{K+1}}{(1-\gamma)^2} \cdot R_{\max}
  \end{equation}
\end{proof}



Finally we show \cref{thm:oneStep}.
% Proof of Theorem 6.2 (section C.2, page 36 in YangXieWang)
% Questions to ponder:
% Should we have two 'delta's?



\begin{prop}
  \label{lem:norm12ineq}
  Let $v$ be a random vector in $\R^n$ then
  \[ \E \norm{v}_1 \leq \sqrt{n} \sqrt{ \E \norm{v}_2^2 } \]
\end{prop}

\begin{proof}
  Denote $v$'s coordinates $v=(v_1, \dots, v_n)$.
  Cauchy-Schwarz applied to some vector $w$ and $(1, \dots, 1)$ yields
  \[ \norm{w}_1 \leq \sqrt{n} \norm{w}_2 \]
  Now let $w = (\E v_1, \dots, \E v_n)$.
  Then by linearity of expectation and Jensens inequality
  \[ \E \norm{v} = \norm{w} \leq \sqrt{n} \sqrt{\sum_{i=1}^n (\E v_i)^2}
  \leq \sqrt{n} \sqrt{\E \sum_{i=1}^n v_i^2} = \sqrt{n} \sqrt{\E \norm{v}_2^2} \]
\end{proof}

\begin{proof}[Proof of \cref{thm:oneStep}]
  First some introductory fixing of notation and variables.
  Fix a minimal $\delta$-covering of $\Cal{F}$
  with centers $f_1, \dots, f_{N_\delta}$.
  Define
  \[ \wt{Q} := \argmin_{f \in \Cal{F}} \norm{f - TQ}_{\nu}^2 \]
  \[ k^* := \argmin_{k \in [N_\delta]} \norm{f_k - \wh{Q}}_\infty \]
  and $ X_i := (S_i, A_i) $.
  Notice that $\wt{Q}$ differs from $\wh{Q}$ in that
  $\wt{Q}$ approximates $TQ$ w.r.t. $\norm{\cdot}^2_\nu$ while
  $\wh{Q}$ approximates $Y = (Y_1, \dots, Y_n)$ in mean squared error over
  $X = (X_1, \dots, X_n)$.
  %We define a shorthand for the average by
  %$\ol{f(X)} := 1/n \cdot \sum_{i=1}^n f(X_i)$.
  %In general we shall loosely write \emph{vectored} equations like
  %$ Y = R + \gamma \max_{a \in \Cal{A}} Q(S',a) $ 
  %(this is then meant to be equivalent to the definition of $Y$).
  We shall be loose about applying functions to vectors
  (of random variables)
  in the sense that they are applied entry-wise.
  We use $\norm{\cdot}_p$ to denote the (finite dimensional) $p$-norm
  ($p$ ommitted when $p=2$).
  When talking about $p$-norms on the random variables we always specify
  the distribution (e.g. $\norm{\cdot}_\nu$).
  When the sample (e.g. $X$) is clear from context we omit it writing
  $\norm{f} = \norm{f(X)}$.
  
  \textbf{Step 1}
  By definion (of $\wh{Q}$) for all $f \in \Cal{F}$ we have
  $\norm{\wh{Q}(X) - Y}^2 \leq \norm{f(X) - Y}^2$, leading to
  \begin{align}
    & \norm{Y}^2 + \norm{\wh{Q}}^2 - 2 Y\cdot \wh{Q}
    \leq \norm{Y}^2 + \norm{f}^2 - 2 Y\cdot f
    \\ \iff & \norm{\wh{Q}}^2 + \norm{TQ}^2 - 2 \wh{Q}\cdot TQ 
    \leq \norm{f}^2 + \norm{TQ}^2 - 2 f\cdot TQ + 2 Y\cdot \wh{Q}
    - 2 Y \cdot f - 2 \wh{Q}\cdot TQ + 2f\cdot TQ
    \\ \iff & \norm{\wh{Q} - TQ}^2
    \leq \norm{f - TQ}^2 + 2(Y - TQ)\cdot(\wh{Q} - f) 
    \\ \iff & \norm{\wh{Q} - TQ}^2
    \leq \norm{f - TQ}^2 + 2 \xi \cdot (\wh{Q} - f)
    \label{eq:c2first}
  \end{align}
  Where $ \xi_i := Y_i - TQ(X_i) $ and $\xi := (\xi_1, \dots, \xi_n)$.
  Now we proof a minor lemma
  \begin{prop}
    $\E(\xi_i g(X_i)) = 0$ for any function $g:\R\to \R$.
    \label{lem:YTQ}
  \end{prop}
  \begin{proof}
    Recall that $X_i = (S_i, A_i)$,
    \begin{align*}
      Y_i = R_i + \gamma \max_{a \in \Cal{A}} Q(S_{i+1}, a)
    \end{align*}
    where $S_{i+1} \sim P(X_i)$, $R_i \sim R(X_i)$ and
    \begin{align*}
      TQ(X_i) = \E_{X_i} R'_i
      + \gamma \E_{X_i} Q(S', \argmax_{a \in \Cal{A}} Q(S', a))
    \end{align*}
    where $S' \sim P(X_i)$, $R'_i \sim R(X_i)$.
    Since $S'$ and $S_{i+1}$ are i.i.d.
    \begin{align*}
      \E_{X_i} \xi_i & = \E_{X_i} \left( Y_i - TQ(X_i) \right)
      \\ & = \E_{X_i} R_i - \E_{X_i} R'_i
      + \gamma \left( \E_{X_i} \left( \max_{a \in\Cal{A}} Q(S_{i+1}, a) \right)
      - \E_{X_i} \argmax_{a \in \Cal{A}} \left(  Q(S', a) \right) \right)
      \\ & = 0 
    \end{align*}
    Therefore $\E(\xi_i g(X_i)) = 0$.
  \end{proof}
  By this lemma we can deduce
  \begin{equation}
    \E \left( \xi \cdot (\wh{Q} - f) \right)
    = \E \left( \xi \cdot (\wh{Q} - TQ) \right)
  \end{equation}
  To bound this we insert $f_{k^*}$ by the triangle inequality
  \begin{equation}
    \abs{\E \left( \xi \cdot (\wh{Q} - TQ) \right) }
    \leq \abs{\E \left( \xi \cdot (\wh{Q} - f_{k^*}) \right) } 
    + \abs{\E \left( \xi \cdot (f_{k^*} - TQ) \right) }
    \label{eq:c2triangle1}
  \end{equation}
  We now bound these two terms. The first by Cauchy-Schwarz
  \begin{equation}
    \abs{\E \xi \cdot (\wh{Q} - f_{k^*})}
    \leq \E \left( \norm{\xi} \norm{\wh{Q} - f_{k^*}} \right)
    \leq \E (\norm{\xi}) \sqrt{n} \delta
    \leq 2 n V_{\max} \delta
    \label{eq:c2firstcs}
  \end{equation}
  where we have used that $\norm{\wh{Q} - f_{k^*}}_\infty \leq \delta$ so
  \begin{equation}
    \norm{\wh{Q} - f_{k^*}}^2
    = \sum_{i=1}^n (\wh{Q}(X_i) - f_{k^*}(X_i))^2
    \leq \sum_{i=1}^n \delta^2
    = n \delta^2
  \end{equation}
  and that $\abs{Y_i}, TQ(X_i) \leq V_{\max}$ so
  \begin{equation}
    \norm{\xi}^2 = \sum_{i=1}^n (Y_i - TQ(X_i))^2 
    \leq \sum_{i=1}^n (2 V_{\max})^2
    = 4 V_{\max}^2 n
  \end{equation}
  To bound the second term in \cref{eq:c2triangle1} define
  \begin{equation}
  Z_j := \xi \cdot (f_j - TQ) \norm{f_j - TQ}^{-1}
  \end{equation}
  %TODO more steps here probably
  Then
  \begin{align}
    \E \left( \xi \cdot (f_{k^*} - TQ) \right)
    &= \E \left( \norm{f_{k^*} - TQ} \abs{Z_{k^*}} \right)
    \label{eq:long1_1}
    \\ &\leq 
    \E \left( \left(\norm{\wh{Q} - TQ} + \norm{\wh{Q} - f_{k^*}} \right)
    \abs{Z_{k^*}} \right) 
    \label{eq:long1_2}
    \\ &\leq 
    \E \left( \left(\norm{\wh{Q} - TQ} + n \delta \right)
    \abs{Z_{k^*}} \right) 
    \label{eq:long1_3}
    \\ &\leq 
    \left( \E \left(\norm{\wh{Q} - TQ} + n \delta \right)^2 \right)^{1/2}
    \left( \E Z_{k^*}^2 \right)^{1/2} 
    \label{eq:long1_4}
    \\ &\leq 
    \E \left(\norm{\wh{Q} - TQ} + n \delta \right) 
    \left( \E Z_{k^*}^2 \right)^{1/2} 
    \label{eq:long1_5}
    \\ &\leq 
    \left(\sqrt{ \E \norm{\wh{Q} -TQ}_2^2 } + n \delta \right) 
    \left( \E Z_{k^*}^2 \right)^{1/2} 
    \label{eq:long1_6}
    \\ &\leq \left(\sqrt{\E \norm{\wh{Q} -TQ}_2^2} + n \delta \right)
    2 V_{\max} 
    \label{eq:long1_8}
  \end{align}
  Where \cref{eq:long1_1} to \cref{eq:long1_2} is by the triangle inequality,
  \cref{eq:long1_5} to \cref{eq:long1_6} is \cref{lem:norm12ineq}
  and \cref{eq:long1_6} to \cref{eq:long1_8} is due to the following
  \begin{prop}
    \[ |Z_j| \leq 2 V_{\max} \]
  \end{prop}
  \begin{proof} For any $i \in [n]$
    \[ |\xi_i| = |Y_i - \wh{Q}(X_i)| \leq |Y_i| + |\wh{Q}(X_i)|
    \leq 2 V_{\max} \]
    Thus
    \begin{align*}
      \xi \cdot (f_j - TQ)
      & \leq \norm{\xi} \cdot \norm{f_j - TQ} \cdot \norm{f_j - TQ}^{-1}
      \\ & \leq 2 V_{\max} 
    \end{align*}
  \end{proof}
  Combining \cref{eq:c2first}, \cref{eq:c2triangle1},
  \cref{eq:c2firstcs} and \cref{eq:long1_8}
  \begin{align}
    \E \norm{\wh{Q} - TQ}^2 & \leq \E \norm{f - TQ}^2 + 4 n V_{\max} \delta
    + \left( \sqrt{\E \norm{\wh{Q} - TQ}^2} + \sqrt{n} \delta \right)
    2 V_{\max} 
    \\ & = 2 \sqrt{\E \norm{\wh{Q} - TQ}^2} V_{\max} 
    + 6 n \delta V_{\max} + \E \norm{f - TQ}^2
    \label{eq:c2step1comb}
  \end{align}
  \begin{lem} Let $a,b>0, \kappa \in (0,1]$ then
    \[ a^2 \leq 2ab + c \implies a^2 \leq (1 + \kappa)^2 b^2 / \kappa
    + (1 + \kappa) c \]
    \label{lem:abc}
  \end{lem}
  \begin{proof} $0 \leq (x - y) = x^2 + y^2 - 2xy \implies 2xy \leq x^2 + y^2$
    for any $x, y \in \R$ so
    \begin{align*}
      2ab & = 2 \sqrt{\frac{\kappa}{1+\kappa}} a \sqrt{\frac{1+\kappa}{\kappa}} b
      \\ & \leq \frac{\kappa}{1+\kappa} a^2 + \frac{1 + \kappa}{\kappa} b^2
    \end{align*}
  \end{proof}
  By \cref{lem:abc} applied to \cref{eq:c2step1comb}
  \begin{align}
    \frac{1}{n} \E \norm{\wh{Q} - TQ}^2
    & \leq \frac{(1+\kappa)^2}{\kappa} \frac{1}{n} V_{\max}^2
    + (1 + \kappa) \left( 6 \delta V_{\max}
    + \frac{1}{n} \E \norm{f - TQ}^2 \right)
    \label{eq:c2step1final0}
    \\ & \leq \frac{(1+\kappa)^2}{\kappa} \frac{1}{n} V_{\max}^2
    + (1 + \kappa) \left( 6 \delta V_{\max}
    + \inf_{f \in \Cal{F}} \frac{1}{n} \E \norm{f - TQ}^2 \right)
    \label{eq:c2step1final}
  \end{align}
  Here we can take infimum since
  \cref{eq:c2step1final0} holds for all $f \in \Cal{F}$.

  \textbf{Step 2} Here we link up $\norm{\wh{Q} - TQ}_{\sigma}^2$ 
  with $\E \frac{1}{n} \norm{\wh{Q} -TQ}^2$.
  \begin{align}
    \abs{ \left( \wh{Q}(x) - TQ(x) \right)^2
    - \left( f_{k^*}(x) - TQ(x) \right)^2 }
    &= \abs{ \wh{Q}(x) - f_{k^*}(x) } \cdot
    \abs{ \wh{Q}(x) + f_{k^*}(x) - 2 TQ(x) }
    \\ & \leq 4 V_{\max} \delta
  \end{align}
  Using this twice we can say
  \begin{align}
    & (\wh{Q}(\wh{X}_i) - TQ(\wh{X}_i)^2 
    \\ \leq {} & (\wh{Q}(\wt{X}_i) - TQ(\wt{X}_i))^2
    - (f_{k^*}(\wt{X}_i) - TQ(\wt{X}_i))^2
    + (f_{k^*}(\wt{X}_i) - TQ(\wt{X}_i))^2 
    \\ \leq {} & (f_{k^*}(\wt{X}_i) - TQ(\wt{X}_i))^2 
    + (\wh{Q}(X_i) - TQ(X_i))^2
    - (\wh{Q}(X_i) - TQ(X_i))^2
    \notag \\ & + (f_{k^*}(X_i) - TQ(X_i))^2
    - (f_{k^*}(X_i) - TQ(X_i))^2 + 4 V_{\max} \delta
    \\ \leq {} & (\wh{Q}(X_i) - TQ(X_i))^2
    + (f_{k^*}(\wt{X}_i) - TQ(\wt{X}_i))^2
    - (f_{k^*}(X_i) - TQ(X_i))^2
    + 8 V_{\max} \delta
    \label{eq:c2step2long1}
  \end{align}
  Thus we get
  \begin{align}
    & \norm{\wh{Q} - TQ}_\sigma^2
    \\ = {} & \E \frac{1}{n} \sum_{i=1}^n (\wh{Q}(\wt{X}_i) - TQ(\wt{X}_i))^2
    \\ \leq {} & \E \frac{1}{n} \sum_{i=1}^n \left( (\wh{Q}(X_i) - TQ(X_i))^2 
      + (f_{k^*}(\wt{X}_i) - TQ(\wt{X}_i))^2
    - (f_{k^*}(X_i) - TQ(X_i))^2 \right)
    + 8 V_{\max} \delta 
    \\ = {} & \frac{1}{n} \norm{\wh{Q} - TQ}^2
    + \frac{1}{n} \sum_{i=1}^n h_{k^*}(X_i, \wt{X}_i)
    + 8 V_{\max}
  \end{align}
  Where we define
  \begin{equation}
    h_j(x, y) \defeq \left( f_j(y) - TQ(y) \right)^2
    - \left( f_j(x) - TQ(x) \right)^2
    \label{eq:c2step2hdef}
  \end{equation}
  
\end{proof}





\section{Further directions}


\subsubsection{Suboptimality of policies}
Relating to decision processes and value functions
Through out the paper we discuss a wide array of
approximations of $Q^*$.
The default strategy is then to accept some close-enough approximation $\wt{Q}$
and then pick the greedy policy $\wt{pi}$ with respect to $\wt{Q}$.
We then measure our deviation from optimality in terms of the distance
$\norm{Q^* - \wt{Q}}_\infty$.
However we do not estimate the deviation of
$Q_{\wt{\pi}}$ from $Q^*$ which is actually a better measure
of the sub-optimality of $\wt{\pi}$ when comparing to $\pi^*$.
To this end it could be interesting to some how establish relations
between
$\norm{Q^* - Q_{\wt{\pi}}}_\infty$ and $\norm{Q^* - \wt{Q}}_\infty$.


\section{Appendices}
\subsection{Disambiguation}




\begin{itemize}
	\item $[q] = \{1,\dots,q\}$ for $q\in \N$.
	\item $C_{\bb{K}}(X) = \{ f : X \to \bb{K} \mid f \; \text{continuous} \}$,
		$\bb{K} \in \{ \R, \C \}$. $C(X) = C_\R(X)$.
	\item ANN: artificial neural network see \cref{def_ANN}.
\end{itemize}

\subsection{Lemmas for Fan et al.}

\begin{lem} For $x > 0$.
  \[ \int_x^\infty e^{-t^2/2} \dif t \leq \frac{1}{x} e^{-x^2/2} \]
  \label{lem:normtail}
\end{lem}

\begin{proof}
  Observe that for $t \geq x > 0$ we have $1 \leq t/x$ so
  \begin{align*}
    \int_x^\infty e^{-t^2/2} \dif t
    &\leq \int_x^\infty \frac{t}{x} e^{-t^2/2} \dif t
    \\ &\leq \frac{1}{x} e^{-x^2/2}
  \end{align*}
\end{proof}


\begin{thm}[Banach fixed point theorem]
  Let $(\Cal{X}, d)$ be a complete metric space
  and $T:\Cal{X} \to \Cal{X}$ be a contraction,
  i.e. $d(Tx, Ty)<\gamma d(x, y)$ for some $0 < \gamma < 1$
  and all $x,y \in \Cal{X}$.
  Then $T$ has a unique fixed point $x^*$ and for every $x\in \Cal{X}$
  it holds that $T^k x \to x^*$ as $k \to \infty$, with rate
  $d(T^k x, x^*) < \gamma^k d(x, x^*)$.
\end{thm}

\subsection{Other notes}

\begin{defn}[Interior]
  For a subset $A \subseteq \cl{X}$
  of a topological space $(\cl{X}, \cl{O}_\cl{X})$
  the \defemph{interior} $A^\circ \subseteq A$ of $A$ is the
  union of all open sets $U \in \cl{O}_\cl{X}$ which are contained in $A$.
  That is
  \[A^\circ = \bigcup_{U \in \cl{U}} U, \text{ where } \cl{U}
  = \left\{ U \in \cl{O}_\cl{X} \Mid U \subseteq A \right\}\]
  \label{defn:interior}
\end{defn}

\begin{defn}[Order Topology]
  Given a totally ordered set $(\cl{X}, <)$ the \defemph{order topology}
  is the topology generated by the subbase of sets on the form
  \[ \left\{ x \mid a < x \right\},\; a \in \cl{X}\; \hrm{and} \;
  \left\{ x \mid x < b \right\}, b \in \cl{X} \]
  \label{defn:orderTop}
\end{defn}

\begin{defn}[$\sigma$-algebra]
  A $\sigma$-\defemph{algebra} $\Sigma$ on a set $\cl{X}$ is a pavement (family of
  subsets of $\cl{X}$) $\Sigma \subseteq 2^\cl{X}$ (where $2^\cl{X}$ denotes the
  powerset of $\cl{X}$) satisfying
  \begin{itemize}
    \item $\emptyset,\; \cl{X} \in \Sigma$.
    \item $A \in \Sigma \implies \cl{X} \setminus A \in \Sigma$.
    \item If $A_1, A_2, \dots \in \Sigma$
      are a countable collection of subsets of $\cl{X}$
      in $\Sigma$ then $\bigcup_{i \in \N} A_i \in \Sigma$.
  \end{itemize}
  The pair $(\cl{X}, \Sigma)$ of a set and a $\sigma$-algebra on it is
  called a \defemph{measurable space}.
  \label{defn:sigmaAlg}
\end{defn}

\begin{thm}
  For any pavement $\Gamma \subseteq 2^\cl{X}$ of a set $\cl{X}$ there exists
  a \emph{smallest} $\sigma$-algebra $\Sigma \subseteq 2^\cl{X}$ on $\cl{X}$
  satisfying
  \begin{enumerate}
    \item $\Gamma \subseteq \Sigma$.
    \item For any $\sigma$-algebra $\Sigma'$ for which $\Gamma \subseteq \Sigma'$
      it holds that $\Sigma \subseteq \Sigma'$.
  \end{enumerate}
  This smallest $\sigma$-algebra is denoted $\sigma(\Gamma)$.
\end{thm}

\begin{defn}[Borel $\sigma$-algebra]
  For a topological space the \defemph{Borel} $\sigma$-algebra is the smallest
  $\sigma$-algebra containing all open sets.
  \label{defn:BorelAlg}
\end{defn}

\begin{defn}[Product $\sigma$-algebra]
  Let $(\cl{X}_i, \cl{A}_i)_{i \in I}$ be a collection of measurable spaces.
  the product $\sigma$-algebra
  \[ \bigotimes_{i \in I} \cl{A}_i \]
  is the smallest $\sigma$-algebra making all coordinate projections
  $\rho_i : \prod_{j \in I} \cl{X}_j \to \cl{X}_i$
  measurable.
  In particular if $\abs{I} = 2$
  \[ \cl{A}_1 \otimes \cl{A}_2 = \sigma \left(
      \left\{ A_1 \times \cl{X}_2 \mid A_1 \in \cl{A}_1 \right\} \cup
  \left\{ \cl{X}_1 \times A_2 \mid A_2 \in \cl{A}_2 \right\} \right) \]
  \label{defn:prodSigmaAlg}
\end{defn}

\begin{defn}[Dynkin class]
  Let $D$ be a pavement of $X$,
  that is a collection of subsets of $X$.
  $D$ is called a \defemph{Dynkin class} if
  \begin{enumerate}
    \item $X \in D$,
    \item If $A, B \in D$ and $A \subseteq B$ then $B \setminus A \in D$,
    \item If $A_1, A_2, \dots \in D$ with $A_n \subseteq A_{n+1}$ for
      all $n \in \N$ then $\bigcup_{n=1}^\infty A_n \in D$.
  \end{enumerate}
  \label{defn:DynkinClass}
\end{defn}

\begin{thm}[Dynkins $\pi$-$\lambda$ theorem]
  Let $P$ be a pavement of $X$ which is stable under finite intersections
  (such are called $\pi$-systems) and $D$ a Dynkin class
  (see \cref{defn:DynkinClass}).
  If $P \subseteq D$ then $\sigma(P) \subseteq D$
  where $\sigma(P)$ is the smallest $\sigma$-algebra containing $P$.
  \label{thm:DynkinPiLambda}
\end{thm}

\begin{defn}[Measure]
  Given a measurable space $(\cl{X}, \Sigma)$ a \defemph{measure}
  is a function $\mu : \Sigma \to [0, \infty]$ satisfying
  \begin{enumerate}
    \item $\mu(\emptyset) = 0$
    \item $\mu\left( \bigcup_{i \in \N} A_i \right) =
      \sum_{i \in \N} \mu(A_i)$
      for any countable collection of mutually disjoint sets
      $A_1, A_2, \dots \in \Sigma$.
  \end{enumerate}
  If there exists a sequence of subsets
  $A_1 \subseteq A_2 \subseteq \dots \subseteq
  \cl{X}$ with $\bigcup_{i \in \N} A_i = \cl{X}$ and $\mu(A_i) < \infty$ for all
  $i \in \N$ then $\mu$ is called $\sigma$\defemph{-finite}.
  If $\mu(\cl{X}) < \infty$ then $\mu$ is called \defemph{finite},
  and if furthermore $\mu(\cl{X}) = 1$ then $\mu$ is called a
  \defemph{probability measure}.
\end{defn}

\begin{thm}[Carathéodory's extension theorem]
  Let $\cl{X}$ be a set and $\cl{S} \subset 2^\cl{X}$ be a pavement of $\cl{X}$
  satisfying
  \begin{enumerate}
    \item $\emptyset \in \cl{X}$
    \item $S,\; T \in \cl{S} \implies S \cap T \in \cl{X}$
    \item For $S,\; T \in \cl{S}$ there exists finitely many disjoint
      subsets $S_1, S_2, \dots, S_n \in \cl{S}$ so that
      $ S \setminus T = \bigcup_{i=1}^n S_i$.
  \end{enumerate}
  ($\cl{S}$ is then called a \emph{semi-ring}).
  Let $\mu : \cl{S} \to [0,\infty]$ be a function satisfying
  \begin{enumerate}[label=\roman*.]
    \item $\mu(\emptyset) = 0$
    \item For a countable mutually disjoint collection of subsets
      $S_1, S_2, \dots \in \cl{S}$ it holds that
      $\mu\left( \bigcup_{i \in \N} S_i \right)
      = \sum_{i \in \N} \mu(S_i)$.
  \end{enumerate}
  Then $\mu$ has an extension to a measure $\mu$ on $\sigma(\cl{S})$.
  Furthermore if there exists an increasing sequence of subsets
  $S_1 \subseteq S_2 \subseteq \dots \in \cl{S}$ of $\cl{S}$
  satisfying $\bigcup_{i \in \N} S_i = \cl{X}$ and
  $\mu(S_i) < \infty$ for all $i \in \N$ then
  the extension is unique.
  In particular if $\cl{X} \in \cl{S}$ and $\mu(\cl{X})=1$ then
  $\mu$ extends uniquely to a probability measure on $(\cl{X}, \sigma(\cl{S}))$.
  \label{thm:caratheo}
\end{thm}

\begin{defn}[Measurable function]
  A functions $f : \cl{X} \to \cl{Y}$ between two measurable spaces
  are called \defemph{measurable} if
  \[ f^{-1}(\Sigma_\cl{Y}) =
    \left\{ f^{-1}(B) \mid B \in \Sigma_\cl{Y} \right\}
  \subseteq \Sigma_\cl{X} \]
  The set of such functions we denote
  $\cl{M}(\Sigma_\cl{X}, \Sigma_\cl{Y})$ or $\cl{M}(\cl{X}, \cl{Y})$.
  \label{defn:measFunc}
\end{defn}

\begin{defn}[Almost sure uniform convergence of random processes]
  A sequence of random processes $X_n : \Cal{X} \times \Omega \to \R$
  is said to converge \defemph{almost surely
  uniformly} to $X: \Cal{X} \times \Omega \to \R$ if and only if
  \[ \Prob(\sup_{x \in \Cal{X}} \abs{X_n(x) - X(x)} \to 0) = 1 \]
\end{defn}

\begin{defn} [Uniform convergence in probability of random processes]
  A sequence of random processes $X_n: \Cal{X} \times \Omega \to \R$
  is said to converge \defemph{uniformly
  in probability} to $X : \Cal{X} \times \Omega \to \R$ if and only if
  \[ \sup_{x \in \Cal{X}} \abs{X_n(x) - X(x)} \overset{P}{\to} 0 \]
  \label{defn:uniformConvProb}
\end{defn}

\begin{defn}
  A sequence of events $A_1, A_2, \dots \subseteq \Omega$
  is said to be \defemph{asymptotically almost sure}
  if $\Prob(A_k) \to 1$ for $k \to \infty$.
  \label{defn:aas}
\end{defn}

\begin{example}
  For example if $U_1, U_2, \dots \sim \hrm{Unif}(0,1)$ are i.i.d. random
  variables, $X_k = \max_{i \in [k]} U_i$ for $k \in \N$ and $\ve > 0$ then
  the events $(A_k)_{k \in \N} = (X_k > 1 - \ve)_{k \in \N}$
  are asymptotically almost sure
  since $\Prob(A_k) \to 1$ as $k \to \infty$.
  The property
  $X_k > 1 - \ve$ is then said to hold \emph{asympotically almost surely}.
  \label{example:aas}
\end{example}

\begin{prop}
  $\id_{\cl{P}(X)} = \mu \mapsto \kappa \circ \mu$
  where $\kappa(\cdot \mid x) = \delta_x(\cdot)$.
  Thus $\kappa$ can be seen as an identity mapping on $\cl{P}(X)$.
  \label{prop:identityKernel}
\end{prop}
\begin{proof}
  \[ \kappa \mu (A) = \int \delta_x(A) \difd \mu(x) = \mu (A) \]
\end{proof}

\begin{defn}[Lipschitz continuity]
  Let $(\cl{X}, d_\cl{X}), \; (\cl{Y}, d_\cl{Y})$ be metric spaces.
  A function $f: \cl{X} \to \cl{Y}$ is
  said to \defemph{Lipschitz} with constant $L > 0$ if
  \[ d_\cl{Y}(f(x),f(y)) \leq L d_\cl{X}(x,y) \]
  \label{defn:Lipschitz}
\end{defn}

\begin{defn}[Differentiability in one variable]
  A function $f : A \to \R$ where $A \subseteq \R$ is an open subset of the
  real numbers is \defemph{differentiable}
  at $x \in \R$ if the \defemph{derivative}
  \[ f'(x) \defeq \lim_{x_n \to x} \frac{f(x) - f(x_n)}{x - x_n} \]
  exists, is finite and is the same for any sequence
  $(x_n)_{n \in \N} \subseteq A$
  converging to $x$
  with $x_n \neq x$ for all $n \in \N$.
  If $f$ is differentiable at $A$ if it is differentiable for every
  $x \in A$.
  If $f' : A \to \R$ is continuous then we write $f \in C^1(A)$.
  If $f'' = (f')' : A \to \R$ exists and is continuous we write
  $f \in C^2(A)$. Like this for $k \in \N_0$ we say that $C^k$
  is the set of $k$ times continuously
  differentiable functions, 
  and we write $f^{(k)}$ for the $k$th derivative,
  when $k=0$ we have $C^0(A) = C(A)$ the set of continuous functions and
  $f^{(0)} = f$.
  This extends to $C^\infty$, called the set of
  \defemph{smooth} functions, for any element is continuously differentiable
  $n$ times for any $n \in \N_0$.
  \label{defn:diffR}
\end{defn}

\begin{defn}[Partial derivatives]
  Let $f : U \to \R$ where $U \subseteq \R^n$ 
  is open be a function
  satisfying for some $x = (x_1, \dots, x_n) \in U$ that
  $f_{x, i} = x_i \mapsto f(x_1, \dots, x_i, \dots, x_n) \in C^1(\rho_i(U))$
  where $\rho_i : U \to \R$ is projection onto the $i$th coordinate.
  The partial derivative of $f$ with respect to the $i$th variable at $x$ is the
  function $\delta_i f(x) \defeq f_{x, i}'(x_i)$.
  For $k \in \N_0$
  if $f_{x, i} \in C^k$ then write
  $\delta_i^k f(x) \defeq f^{(k)} f_{x, i} (x_i)$ whenever this exists.
  If $\alpha = (\alpha_1, \dots, \alpha_n) \in \N_0^n$ we denote by
  $\delta^\alpha f(x) \defeq \delta_1^{\alpha_1} \dots \delta_n^{\alpha_n} f(x)$.
  \label{defn:partialDer}
\end{defn}
\begin{rem}
  A standard result called \emph{Schwartz's theorem} say that the order
  in which partial derivatives are taken does not matter when
  these such derivates are continuous.
\end{rem}

\begin{defn}[Differentiability in $\R^n$]
  A function $f: U \to \R$ defined on an open set $U \subseteq \R^n$
  is said to be $C^k$ for $k \in \N_0$ if
  the partial derivatives
  $\partial^\alpha f : U \to \R$ exists and is continuous for all
  $\alpha \in \N_0^n$ with $\norm{\alpha}_1 = \alpha_1 + \dots + \alpha_n \leq k$.
  \label{defn:diffRn}
\end{defn}

\begin{defn}[Absolutely continuity of measures]
  Let $\mu, \nu \in \cl{P}(\cl{X})$ be $\sigma$-finite measures
  then $\mu$ is said to be \defemph{absolutely continuous} with respect to 
  $\nu$, written $\mu << \nu$ if for all $A \in \Sigma_\cl{X}$ we have
  $\nu(A)=0 \implies \mu(A)=0$.
  \label{defn:absContMeas}
\end{defn}

\begin{thm}[Radon-Nikodym]
  Let $\mu, \nu \in \cl{P}(\cl{X})$ with $\mu << \nu$.
  Then there exists a positive measurable function
  $f : \cl{X} \to [0, \infty)$
  such that $\mu(A) = \int_A f \difd \nu$.
  This function is denoted $f = \frac{\difd \mu}{\difd \nu}$.
  %By some authors also called the \emph{Rattata-Nidoking} theorem.
  \label{thm:radonNiko}
\end{thm}

\begin{thm}[Banach fixed point theorem]
  Let $(\Cal{X}, d)$ be a complete metric space
  and $T:\Cal{X} \to \Cal{X}$ be a contraction,
  i.e. $d(Tx, Ty)<\gamma d(x, y)$ for some $0 < \gamma < 1$
  and all $x,y \in \Cal{X}$.
  Then $T$ has a unique fixed point $x^*$ and for every $x\in \Cal{X}$
  it holds that $T^k x \to x^*$ as $k \to \infty$, with rate
  $d(T^k x, x^*) < \gamma^k d(x, x^*)$.
  \label{thm:BanachFP}
\end{thm}

\subsection{Geometric ergodicity}
\label{sec:geometricErgo}

To understand geometric ergodicity we need
to define some concepts from ergodic theory.
Let $\kappa : \cl{X} \leadsto \cl{X}$ be a transition kernel.
Let $\fk{P} = \kappa^\infty : \cl{X} \leadsto \cl{X}^\infty$.
And denote by
$\fk{P}_x = \fk{P}\delta_x \in \cl{P}(\cl{X}^\infty)$
the probability measure for the process starting at $x \in \cl{X}$.
Let $\rho_i : \cl{X}^\infty \to \cl{X}$ be projection on the
$i$th space.
Define for any $A \in \Sigma_{\cl{X}}$ the function
$\tau_A : \cl{X}^\infty \to \ol{\N} = \inf\{ i \in \N \mid \rho_i \in A \}$.
Intuitively this function records the earliest time where the process
enter the set $A \subseteq \cl{X}$.
Define the function
$\eta_A : \cl{X}^\infty \to \ol{\N} = \sum_{i \in \N} 1_A \circ \rho_i$.
This function records the total number of times in which the process is
inside the set $A$.
Let $\varphi \in \cl{P}(\cl{X})$ be a probability measure on $\cl{X}$.

\begin{defn}[Invariant measure]
  A countably additive measure $\mu \in \cl{P}(\cl{X})$ is said
  to be \defemph{invariant} w.r.t $\kappa$ if $\kappa \circ \mu = \mu$.
\end{defn}

\begin{defn}[Positivity]
  \leavevmode

  $\fk{P}$ is called \defemph{positive} if it admits an $\kappa$-invariant
  probability measure $\mu$.
\end{defn}

\begin{defn}[Irreducibility]
  $\fk{P}$ is called $\varphi$-irreducible
  $\fk{P}_x(\tau_A < \infty) > 0$
  for all $A \in \Sigma_\cl{X}$
  with $\varphi(A) > 0$
  and all $x \in \cl{X}$.
\end{defn}

\begin{defn}[Harris recurrency]
  $\fk{P}$ is called $\varphi$-Harris recurrent if
  it it $\varphi$-irreducible and
  $\fk{P}_x(\eta_A = \infty) = 1$ for all $A \in \Sigma_\cl{X}$ with
  $\varphi(A) > 0$ and all $x \in \cl{X}$.
\end{defn}

\begin{defn}[Geometric ergodicity]
  A Markov process $\fk{P}$ is called \defemph{geometrically ergodic} if
  it is positive with invariant measure $\mu$, $\varphi$-Harris recurrent
  for some $\varphi \in \cl{P}(\cl{X})$ and $\exists t>1$ such that
  \[ \sum_{i=1}^\infty t^i \norm{P^n_x - \mu}_{TV} < \infty,
  \quad \forall x \in \cl{X} \]
\end{defn}








\bibliography{bib}{}

\end{document}
