\subsection{Reinforcement Learning}
In Reinforcement Learning (RL) we are concerned with finding an optimal policy
for an agent in some environment.
Typically (also in the case of Q-learning) this environment is a
Markov decision process

\begin{defn}
	A Markov decision process (MDP) $(\Cal{S}, \Cal{A}, P, R, \gamma)$
	consists of
\begin{itemize}
  \item $(\Cal{S}, \Sigma_{\Cal{S}})$ a measurable space of \defemph{states}.
  \item $(\Cal{A}, \Sigma_{\Cal{A}})$ a measurable space of \defemph{actions}.
  \item $P(\cdot \mid \cdot) : \Sigma_{\Cal{S}} \times (\Cal{S} \times \Cal{A})
    \to [0,1]$ a probability kernel (of \defemph{transition} probabilities).
  \item $R(\cdot \mid \cdot) : \bb{B} \times (\Cal{S} \times \Cal{A})
    \to [0,1]$ a probability kernel (of \defemph{reward} probabilities).
  \item $\gamma \in (0,1)$ a \defemph{discount} factor.
\end{itemize}
\end{defn}

In order for this to make sense we here include
\begin{defn}[Probability kernel]
  Let $(X, \Sigma_X), (Y, \Sigma_Y)$ be measurable spaces.
  A function $\kappa(\cdot \mid \cdot) : \Sigma_Y \times X \to [0,1]$
  is a \defemph{probability kernel} provided
  \begin{itemize}
    \item $B \mapsto \kappa(B \mid x) \in \Cal{P}(\Sigma_Y)$
      that is $\kappa(\cdot \mid x)$ is a probability measure
      for any $x \in X$.
    \item $x \mapsto \kappa(B \mid x) \in \Cal{M}(\Sigma_X, \Sigma_Y)$
      that is $\kappa(B \mid \cdot)$ is ($\Sigma_X/\Sigma_Y$-) measurable
      for any $B \in \Sigma_Y$.
  \end{itemize}
\end{defn}

Note that both $P$ and $R$ to be stochastic
and that $R$ can depend on the action as well as the state.
This is perhaps the most general way to define an MDP,
generalizing some definitions. Common variations include that
$R$ depends on $\Cal{S}$ only,
$R$ is deterministic, or
$P$ is deterministic. %todo backup this claim by a reference or three
\begin{defn}[Policy]
A (\defemph{randomized}, \defemph{stationary}) \defemph{policy}
$\pi$ is probability kernel
\[\pi(\cdot \mid \cdot) : \Sigma_{\Cal{A}} \times \Cal{S} \to [0,1] \]
\end{defn}

An MDP together with a policy and an initial distribution
$\mu \in \Cal{P}(\Cal{S})$
give rise to a countable stochastic process,
$ (X_i)_{i\in \N} = (S_i, A_i, R_i)_{i\in \N}$ 
that is a probability measure $P^\pi_\mu$. on
$(\Cal{S} \times \Cal{A} \times \R)^\N $.
See [ref. to Feinberg On Meas. and Repr. of Str. Meas. p. 31-32,
and pos. Ionescu Tulcea] %todo ref
for full details of how this is constructed.
Intuitively $S_1$ is drawn from $\mu$,
then for all $i \in \N$
$A_i$ is drawn from $\pi(\cdot \mid S_i)$,
a reward is then drawn from $R(\cdot \mid S_i, A_i)$,
then $S_{i+1}$ is drawn from $P(\cdot \mid S_i, A_i)$ and so on.
%todo: explain more about measures
We let $E^\pi_\mu$ denote the expectation taken w.r.t $P^\pi_\mu$.
When $\mu$ is a Dirac measure $\delta_x$, i.e. $\mu(\{x\}) = 1$ for some $x$,
we shall generally write $x$ instead of $\mu$,
E.g. $\E^\pi_s$ the expectation taken with respect to $P^\pi_{\delta_s}$

\subsection{Q-Learning}

Fix an MDP $(\Cal{S}, \Cal{A}, P, R, \gamma)$ and a policy $\pi$.

Generally in value iteration and Q-learning,
any function $\Cal{S} \to \R$ is called a (\defemph{state})
\defemph{value} function. Similarly any function
$\Cal{S} \times \Cal{A} \to \R$ is called a (\defemph{state})
\defemph{action value} or \defemph{Q}- function.
The idea behind such functions are (usually) to estimate the cumulated
rewards associated with a state or state-action pair.
One has the reduced the problem of finding a good strategy
to a one step problem of choosing the action that leads to the
highest value. Of course this value of a state will depend
strongly on the policy being followed. 

The \defemph{ideal} value function
w.r.t. $\pi$,
$V^\pi : \Cal{S} \to \R$
\[ V^\pi(s) \defeq \E^\pi_s \sum_{t = 1}^\infty \gamma^t R_t \]
where $R_t$ are the projections of the random process generated from the
MDP with starting distribution $\delta_s$. The value function gives the expected 
accumulated reward when starting in state $s$ and following policy $\pi$.

The \defemph{ideal} Q-function 
w.r.t. $\pi$,
$Q^\pi : \Cal{S} \times \Cal{A} \to \R$
\[ Q^\pi(s, a) \defeq r(s, a) +
\gamma \E (V^\pi(S) \mid S \sim P(\cdot \mid s, a)) \]
where $r(s,a) \defeq \E(R \mid R \sim R(\cdot \mid s, a))$.

One could think that it is a bit superfluous to define both a value and an
action value function. 
%todo: prove something about the equivalence
According to [todo: ref] the main reason to work with Q-functions
is that it is more difficult to work with value function for several reasons.
Firstly to know what is the best action $a^*$ given a state $s$
and a value function $V$, one has to calculate for each action $a$
the distribution of the next state $s'$ and take expectation over
$V(s')$. This either requires full knowledge of the transition kernel
(this falls outside the so called model-free approaches)
or some way of estimating it, and in both cases at computational cost.
Whereas Q-functions simply requires finding
$\argmax_{a \in \Cal{A}} Q(s,a)$.

We define the operator
\[ (P^\pi Q)(s, a) \defeq \E (Q(S', A') \mid S' \sim P(\cdot \mid s, a),
A' \sim \pi(\cdot \mid S')) \]
Intuitively this operator yields the expected state-action-value function
when looking \emph{one step ahead} following the policy $\pi$ and taking
expectation of $Q$.

We define the operator $T^\pi$ called the Bellman operator by
\[ (T^\pi Q)(s, a) \defeq r(s, a) + \gamma (P^\pi Q)(s, a) \]
This operator adjust the $Q$ function to look more like $Q^\pi$ making one
iteration of reward-propagation discounting with $\gamma$.
Indeed it is easily seen that $Q^\pi$ is a fixed point for $T^\pi$.
%todo: prove this

A \emph{greedy} policy $\pi$ with respect to a state-action value function $Q$
is a policy which deterministically chooses an action with maximal value of $Q$
for each state.
That is $\pi(s) = \delta_a$ for some $a \in \argmax_a Q(s,a)$.
We then write $\pi = \pi_Q$. With this we can define the operator $T$:
\[ T Q \defeq T^{\pi_Q} Q \]
called the Bellman \emph{optimality} operator.

The optimal Q-function is defined as
\[ Q^*(s, a) \defeq \sup_\pi Q^\pi(s, a) \]
where the supremum is taken over all policies.

One can show that there is a policy $\pi^*$ such that $Q^* = Q^{\pi^*}$.
This is the optimal policy - the goal of RL. 

Note that $V^\pi$, $Q^\pi$ and $Q^*$ are usually infeasible to calculate to
machine precision, unless $\Cal{S} \times \Cal{A}$ is finite and not very big.


The Bellman optimality \emph{equation} can then be written $Q^* = TQ^*$.




\subsection{Artificial Neural Networks}

\begin{defn}\label{def_ANN}
	An \textbf{ANN} (Artificial Neural Network) with structure
	$\{ d_i \}_{i=0}^{L+1} \subseteq \N$,
	activation functions $\sigma_i = (\sigma_{ij} : \R \to \R)_{j=1}^{d_i}$
	and weights $\{ W_i \in M^{d_i \times d_{i-1}}, v_i \in \R^{d_i} \}_{i=1}^{L+1}$
	is the function $F:\R^{d_0} \to \R^{d_{L+1}}$ 
	\[ F = w_{L+1} \circ \sigma_L \circ w_L \circ \sigma_{L-1} \circ \dots \circ w_1 \]
	where $w_i$ is the affine function $x \mapsto W_i x + v_i$ for all $i$.

	Here $\sigma_i(x_1, \dots, x_{d_i})
	= (\sigma_{i1}(x_1), \dots, \sigma_{id_{i}}(x_{d_{i}}))$.

	$L \in \N_0$ is called the number of hidden layers.

	$d_i$ is the number of neurons or nodes in layer $i$.
\end{defn}

An ANN is called \emph{deep} if there are two or more hidden layers.

\begin{defn}[Sparse ReLU Networks]
  For $s,V \in \R$ a (s,V)-\defemph{Sparse ReLU Network} is an ANN $f$
  with any structure $\{d_i\}_{i\in [L+1]}$,
  all activation functions being \emph{ReLU} i.e. $\sigma_{ij} = \max(\cdot, 0)$
  and any weights $(W_\ell, v_\ell)$
  satisfying
  \begin{multicols}{3}
    \begin{itemize}
      \item $\max_{\ell \in [L+1]} \norm{\widetilde{W}_\ell}_\infty \leq 1$
      \item $\sum_{\ell = 1}^{L+1} \norm{\widetilde{W}_\ell}_0 \leq s$
      \item $\max_{j \in [d_{L+1}]} \norm{f_j}_\infty \leq V$
    \end{itemize}
  \end{multicols}
  Here $\widetilde{W}_\ell = (W_\ell, v_\ell)$.
  The set of them we denote $\Cal{F}(s,V)$.
\end{defn}

\subsection{Fitted Q-Iteration}

We here present the algorithm which everything in this paper revolves around:

\begin{algorithm}[H] %\label{algocf:fq} % this labels line, could not fix
	\caption{Fitted Q-Iteration Algorithm}
	\KwIn{MDP $(\Cal{S}, \Cal{A}, P, R, \gamma)$, function class $\Cal{F}$,
		sampling distribution $\nu$, number of iterations $K$,
		number of samples $n$, initial estimator $\widetilde{Q}_0$}
	\For{$k = 0,1,2,\dots,K-1$}{
		Sample i.i.d. observations $\{(S_i, A_i), i \in [n]\}$ from $\nu$
		obtain $R_i \sim R(S_i, A_i)$ and $S'_i \sim P(S_i, A_i)$ \\
		Let $Y_i = R_i + \gamma \cdot \max_{a \in \Cal{A}} \widetilde{Q}_k(S'_i, a)$ \\
		Update action-value function:
		\[ \widetilde{Q}_{k+1} \leftarrow
			\argmin_{f \in \Cal{F}} \frac{1}{n}
			\sum_{i=1}^n (Y_i - f(S_i, A_i))^2 \]
	}
	Define $\pi_K$ as the greedy policy w.r.t. $\widetilde{Q}_K$ \\
	\KwOut{An estimator $\widetilde{Q}_K$ of $Q^*$ and policy $\pi_K$}
\end{algorithm}



