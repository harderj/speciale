% preamble speciale Jacob Harder
% 31. jan. 2020

%packages

\usepackage[utf8]{inputenc} %utf8 is probably good
\usepackage{amsmath}
\usepackage{amssymb}
\usepackage{amsthm}
\usepackage{graphicx} %for including images
\usepackage{float} %for exact placement of figure (and more?)
\usepackage{mathtools} %for \mathclap (stacking under sums)
\usepackage{dsfont} %for boldface numbers
\usepackage{bm} %vectors in bold
\usepackage[ruled,vlined]{algorithm2e}
\usepackage{cleveref}
\usepackage{ifthen}
\usepackage{commath}
\usepackage[a4paper,width=150mm,top=25mm,bottom=25mm]{geometry}
\usepackage{upgreek}
%\usepackage[inline]{enumitem}
\usepackage{multicol}
%\usepackage{fancyhdr} %maybe later..
%\pagestyle{fancy}
\usepackage{mathabx}
\usepackage{csquotes}
\usepackage{tikz}
\usepackage{outline} %for subitems in lists

%front page
\usepackage{wallpaper}
\usepackage{titling}

%bibliography
\usepackage[numbers]{natbib}
\bibliographystyle{plainnat}
\newcommand{\mcite}[1]{[\citenum{#1}, \citeauthor{#1} (\citeyear{#1})]}
\newcommand{\ncite}[1]{\cite{#1}}

%linespread and geometry
\linespread{1.3}

%commands
\newcommand{\Cal}{\mathcal}
\newcommand{\cl}{\mathcal}
\newcommand{\fk}{\mathfrak}
\newcommand{\bb}{\mathbb}
\newcommand{\Q}{\bb{Q}}
\newcommand{\Z}{\bb{Z}}
\newcommand{\N}{\bb{N}}
\newcommand{\R}{\bb{R}}
\newcommand{\C}{\bb{C}}
\newcommand{\E}{\bb{E}}
\newcommand{\Rext}{\ol{\ul{\R}}}
\newcommand{\Var}{\mathrm{Var}}
\newcommand{\Prob}{\mathds{P}} %fundamental probability measure
\newcommand{\idc}{\mathds{1}}
\newcommand{\ve}{\varepsilon} %abbreviation for epsilon
\newcommand{\Yp}{\Upupsilon} %abbreviation for Ypsilon
\newcommand{\difd}{\; \mathrm{d}} %differential d
\newcommand{\wt}{\widetilde}
\newcommand{\wh}{\widehat}
\newcommand{\ol}{\overline}
\newcommand{\ul}{\underline}
\newcommand{\Mid}{\;\middle\vert\;}
\newcommand{\id}{\text{id}}
\newcommand{\supp}{\text{supp}}
\newcommand{\defemph}[1]{\textbf{#1}} %first-mentions of names
\DeclarePairedDelimiter\ceil{\lceil}{\rceil}
\DeclarePairedDelimiter\floor{\lfloor}{\rfloor}
\DeclareMathOperator*{\argmax}{argmax}
\DeclareMathOperator*{\argmin}{argmin}
\newcommand{\defeq}{\vcentcolon=} %definition equality symbol
%add single eq. tag in align*
\newcommand\numberthis{\addtocounter{equation}{1}\tag{\theequation}}
\newcommand{\rleft}[1]{\rotatebox[origin=c]{90}{\ensuremath{#1}}}
\newcommand{\vrel}[3]{ % for vertical subseteq e.g.
\vcenter{\halign{\hfill##\hfill\cr
\ensuremath{#1}\cr
\rotatebox[origin=c]{270}{\ensuremath{#2}}\cr
\ensuremath{#3}\cr
}}}
\newcommand{\lar}{\leftrightarrow}
\newcommand{\Span}{\mathrm{span}}

%theorems
\theoremstyle{definition}
\newtheorem{lem}{Lemma}
\newtheorem{defn}{Definition}
\newtheorem{thm}{Theorem}
\newtheorem{cor}{Corollary}
\newtheorem{rem}{Remark}
\newtheorem{prop}{Proposition}
\newtheorem{asm}{Assumption}
\newtheorem{example}{Example}
%\newtheorem{cond}{Condition}
\newtheorem{sett}{Setting}
\newtheorem{innercond}{Condition}
\newenvironment{cond}[1]
  {\renewcommand\theinnercond{#1}\innercond}
  {\endinnercond}

%cref
\crefname{algocf}{alg.}{algs.}
\Crefname{algocf}{Algorithm}{Algorithms}
\crefname{innercond}{}{}
\Crefname{innercond}{}{}

