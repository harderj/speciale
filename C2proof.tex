
\begin{lem}\label{lem:tlemma}
  $T Q \geq T^\pi Q$ for any policy $\pi : \Cal{S} \to \Cal{P}(\Cal{A})$
  and any action value function $Q: \Cal{S} \times \Cal{A} \to \R$.
\end{lem}
\begin{proof}
  \begin{align*}
    (TQ)(s, a) &= \E \left( R(s, a) + \gamma \max_{a'} Q(S', a')
    \mid S' \sim P(\cdot \mid s, a) \right)
    \\ &\geq \E \left( R(s, a) + \gamma Q(S', A')
    \mid S' \sim P(\cdot \mid s, a), A' \sim \pi(\cdot \mid S') \right)
    \\ &= T^\pi Q(s,a)
  \end{align*}
\end{proof}

\begin{lem}\label{lem:MRN}
  Let $f:\Cal{S}\times\Cal{A} \to \R$ be an action-value function,
  $\tau_1, \dots, \tau_m$ be policies
  and $\mu \in \Cal{P}(\Cal{S}\times\Cal{A})$ be a probability measure.
  Then
  \begin{align*}
    \E_\mu [(P^{\tau_m} \dots P^{\tau_1})(f)]
    \leq \kappa(k - i + j; \mu, \nu) \norm{f}_{2,\nu}
  \end{align*}
  For any measure $\nu \in \Cal{P}(\Cal{S}\times\Cal{A})$ which is
  absolutely continuous w.r.t. $(P^{\tau_m} \dots P^{\tau_1})(\mu)$.
	Here $\kappa$ is the concentration coefficients defined in \cref{defn:ccoefs}.
\end{lem}
\begin{proof}
  Recall that
  \begin{align*}
    \kappa(m; \mu, \nu) := \sup_{\pi_1, \dots, \pi_m} \left[
      \E_{\nu} \abs{\frac{\dif (P^{\pi_m}\dots P^{\pi_1}\mu)}{\dif \nu}}^2
      \right]^{1/2}
      \\ = \sup_{\pi_1, \dots, \pi_m}
      \norm{\frac{\dif (P^{\pi_m}\dots P^{\pi_1}\mu)}{\dif \nu}}_{2, \nu}
  \end{align*}
  Thus
  \begin{align}
    \E_\mu [(P^{\tau_m} \dots P^{\tau_1})(f)]
    &= \int (P^{\tau_m} \dots P^{\tau_1})(f) \dif \mu
    \\ &= \int f \dif (P^{\tau_m} \dots P^{\tau_1} \mu)
    \label{eq:preradniko}
    \\ &= \int f \frac{\dif(P^{\tau_m} \dots P^{\tau_1} \mu)}{\dif \nu} \dif \nu
    \label{eq:postradniko}
    \\ &\leq \norm{\frac{\dif(P^{\tau_m} \dots P^{\tau_1} \mu)}{\dif \nu}}_{2,\nu}
    \cdot \norm{f}_{2, \nu} \label{eq:cspost}
    \\ &\leq \kappa(m, \mu, \nu) \norm{f}_{2,\nu}
  \end{align}
  Where \cref{eq:postradniko} is due to the Radon-Nikodym theorem
  and \cref{eq:cspost} is Cauchy-Schwarz.
\end{proof}

\begin{proof}[Proof of \cref{thm:errorprop}] 
  First some things to keep in mind during the proof.
  Recall that $V_{\max} = R_{\max} / (1 - \gamma)$ and that
  $\pi_Q$ is the greedy policy w.r.t. $Q$.
  Denote 
  \[ \pi_i = \pi_{\wt{Q}_i},
    \; Q_{i+1} = T \wt{Q}_{i},
  \; \varrho_{i} = Q_{i} - \wt{Q}_{i},
\; \mbox{ for } i \in \{0,\dots,K+1\} \]
  Note that for any policy $\pi$,
  $P^{\pi}$ is linear and 1-contrative on
  $\Cal{L}^\infty(\Cal{S} \times \Cal{A})$. %todo: proof
  Also \[ T^{\pi} Q^{\pi} = Q^{\pi}, \;
    T Q = T^{\pi_Q} Q, \;
  T Q^* = Q^* = Q^{\pi^*} \]
  where $\pi^*$ is greedy w.r.t. $Q^*$. 
  If $f > f'$ for $f,f':\Cal{S}\times \Cal{A} \to \R$
  then $P^{\pi} f \geq P^{\pi} f'$. %nice-to-have proof
  
  The proof consists of four steps.

  \textbf{Step 1}
  We start by relating $Q^* - Q^{\pi_K}$, the quantity of interest,
  to $Q^* - \wt{Q}_K$, which is more related to the output of the algorithm.
  Using \cref{lem:tlemma} we can make the upper bound
  \begin{align}
    Q^* - Q^{\pi_K} &= T^{\pi^*} Q^* - T^{\pi_K} Q^{\pi_K} \notag
    \\ &= T^{\pi^*} Q^* + (T^{\pi^*} \wt{Q}_K - T^{\pi^*} \wt{Q}_K)
    + (T \wt{Q}_K - T \wt{Q}_K)- T^{\pi_K} Q^{\pi_K} \notag
    \\ &= (T^{\pi^*} \wt{Q}_K - T \wt{Q}_K)
    + (T^{\pi^*} Q^* - T^{\pi^*} \wt{Q}_K) 
    + (T \wt{Q}_K - T^{\pi_K} Q^{\pi_K}) \notag
    \\ &\leq (T^{\pi^*} Q^* - T^{\pi^*} \wt{Q}_K) 
    + (T \wt{Q}_K - T^{\pi_K} Q^{\pi_K}) \notag
    \\ &= (T^{\pi^*} Q^* - T^{\pi^*} \wt{Q}_K) 
    + (T^{\pi_K} \wt{Q}_K - T^{\pi_K} Q^{\pi_K}) \notag
    \\ &= \gamma P^{\pi^*} (Q^* - \wt{Q}_K)
    + \gamma P^{\pi_K} (\wt{Q}_K - Q^{\pi_K}) \notag
    \\ &= \gamma (P^{\pi^*} - P^{\pi_K})(Q^* - \wt{Q}_K)
    + \gamma P^{\pi_K} (Q^* - Q^{\pi_K}) \label{preU}
  \end{align}
  This implies
  \[ (I - \gamma P^{\pi_K})(Q^* - Q^{\pi_K})
  \leq \gamma (P^{\pi^*} - P^{\pi_K})(Q^* - \wt{Q}_K) \]
  Since $\gamma P^{\pi_K}$ is $\gamma$-contractive,
  $U = (I - \gamma P^{\pi_K})^{-1}$ exists as a bounded operator on
  $\Cal{L}^\infty(\Cal{S}\times \Cal{A})$ and equals
  \[ U = \sum_{i=0}^\infty \gamma^i (P^{\pi_K})^i \] 
  From this we also see that $f \geq f' \implies U f \geq U f'$ for any
  $f, f' : \Cal{S}\times \Cal{A} \to \R$.
  Therefore we can apply $U$ on both sides of \cref{preU} to obtain 
  \begin{equation} Q^* - Q^{\pi_K} \leq \gamma U^{-1}(P^{\pi^*}(Q^* - \wt{Q}_K)
  - P^{\pi_K} (Q^* - \wt{Q}_K)) \label{eq:qq1} \end{equation} 

  \textbf{Step 2}
  Using \cref{lem:tlemma} for any $i \in [K]$ we can get an upper bound
  \begin{align}
    Q^* - \wt{Q}_{i+1} &= Q^* + (T\wt{Q}_i - T \wt{Q}_i) - \wt{Q}_{i+1}
    + (T^{\pi^*}\wt{Q}_i - T^{\pi^*}\wt{Q}_i) \notag
    \\ &= (Q^* - T^{\pi^*} \wt{Q}_i) + (T \wt{Q}_i - \wt{Q}_{i+1})
    + (T^{\pi^*}\wt{Q}_i - T \wt{Q}_i) \notag
    \\ &= (T^{\pi^*} Q^* - T^{\pi^*} \wt{Q}_i) + \varrho_{i+1}
    + (T^{\pi^*}\wt{Q}_i - T \wt{Q}_i) \notag
    \\ &\leq T^{\pi^*} Q^* - T^{\pi^*} \wt{Q}_i + \varrho_{i+1} \notag
    \\ &= \gamma P^{\pi^*} (Q^* - \wt{Q}_i) + \varrho_{i+1} \label{Qst_high}
  \end{align}
  and a lower bound
  \begin{align}
    Q^* - \wt{Q}_{i+1} &= Q^* + (T\wt{Q}_i - T \wt{Q}_i) - \wt{Q}_{i+1}
    + (T^{\pi_i}Q^* - T^{\pi_i}Q^*) \notag
    \\ &= (T^{\pi_i}Q^* - T^{\pi_i} \wt{Q}_i) + \varrho_{i+1}
    + (T Q^* - T^{\pi_i} Q^*) \notag
    \\ &\geq T^{\pi_i}Q^* - T^{\pi_i} \wt{Q}_i + \varrho_{i+1} \notag
    \\ &= \gamma P^{\pi_i} (Q^* - \wt{Q}_i) + \varrho_{i+1} \label{Qst_low}
  \end{align}
  Applying \cref{Qst_high} and \cref{Qst_low} iteratively we get  
  \begin{align}
    Q^* - \wt{Q}_K \leq \gamma^K (P^{\pi^*})^K (Q^* - \wt{Q}_0)
    + \sum_{i=0}^{K-1} \gamma^{K-1-i} (P^{\pi^*})^{K-1-i} \varrho_{i+1}
    \label{eq:Qsk_high}
  \end{align}
  and
  \begin{align}
    Q^* - \wt{Q}_K \geq \gamma^K (P^{\pi_{K-1}} \dots P^{\pi_0})(Q^* - \wt{Q}_0)
    + \sum_{i=0}^{K-1}\gamma^{K-1-i}(P^{\pi_{K-1}}\dots P^{\pi_{i+1}})
    \varrho_{i+1} \label{eq:Qsk_low}
  \end{align}

  \textbf{Step 3}
Combining \cref{eq:Qsk_high} and \cref{eq:Qsk_low} with \cref{eq:qq1} we get
  \begin{equation}
    \begin{split}
      Q^* - Q^{\pi_K} \leq U^{-1} \bigg(
	\gamma^{K+1}((P^{\pi^*})^{K+1} - P^{\pi_K} \dots P^{\pi_0})(Q^* - \wt{Q}_0)
	\\ + \sum_{i=0}^{K-1} \gamma^{K-i}
      ((P^*)^{K-i} - P^{\pi_K} \dots P^{\pi_{i+1}}) \varrho_{i+1} \bigg)
    \end{split}
    \label{eq:step3_1}
  \end{equation}
  For shorthand define constants
  \begin{equation} \alpha_i = \frac{(1-\gamma) \gamma^{K-i-1}}{1 - \gamma^{K+1}}
    \; \mbox{ for } 0 \leq i \leq K-1 \mbox{ and }
    \alpha_K = \frac{(1-\gamma) \gamma^K}{1 - \gamma^{K+1}}
  \end{equation}
  (note that $\sum_{i=0}^K \alpha_i = 1$) and operators
  \begin{align}
    O_i = (1-\gamma)/2 U^{-1} [(P^{\pi^*})^{K-i}
    + (P^{\pi_K} \dots P^{\pi_{i+1}})]
    \\ O_K = (1-\gamma)/2 U^{-1} [(P^{\pi^*})^{K+1}
    + (P^{\pi_K} \dots P^{\pi_0})]
  \end{align}
  Then by \cref{eq:step3_1}
  \begin{align}
    \abs{Q^* - Q^{\pi_K}} \leq \frac{2 \gamma (1- \gamma^{K+1})}{(1-\gamma)^2}
    \left[ \sum_{i=0}^{K-1} \alpha_i O_i \abs{\varrho_{i+1}}
    + \alpha_K O_K \abs{Q^* - \wt{Q}_0} \right]
  \end{align}
  So by linearity of expectation
  \begin{align}
    \norm{Q^* - Q^{\pi_K}}_{1, \mu} &= \E_\mu \abs{Q^* - Q^{\pi_K}}
    \\ &\leq \frac{2 \gamma (1- \gamma^{K+1})}{(1-\gamma)^2}
    \left[ \sum_{i=0}^{K-1} \alpha_i \E_\mu (O_i \abs{\varrho_{i+1}})
    + \alpha_K \E_\mu (O_K \abs{Q^* - \wt{Q}_0}) \right]
    \label{eq:step3_final}
  \end{align}
  With the bound on rewards we (crudely) estimate
  \begin{equation}
    \E_\mu O_K \abs{Q^* - \wt{Q}_0} \leq 2 V_{\max} = 2 R_{\max} / (1-\gamma)
    \label{eq:step3_crude}
  \end{equation}
  The remaining difficulty lies in $\E_\mu(O_i\abs{\varrho_{i+1}})$.

  \textbf{Step 4}
  Using the sum expansion of $U^{-1}$ we get
  \begin{align}
    & \E_\mu (O_i \abs{\varrho_{i+1}})
    \\ &= \frac{1 - \gamma}{2} \E_\mu \left( U^{-1} [(P^{\pi_K})^{K-i} +
    P^{\pi_K} \dots P^{\pi_{i+1}}] \abs{\varrho_{i+1}} \right)
    \\ &= \frac{1 - \gamma}{2} \E_\mu \left( \sum_{j=0}^\infty
      [(P^{\pi_K})^j (P^{\pi_K})^{K-i}
      + (P^{\pi_K})^{j+1} P^{\pi_{K-1}} \dots P^{\pi_{i+1}}]
    \abs{\varrho_{i+1}} \right)
    \\ &= \frac{1 - \gamma}{2} \sum_{j=0}^\infty
      \E_\mu \left( [(P^{\pi_K})^j (P^{\pi_K})^{K-i}
      + (P^{\pi_K})^{j+1} P^{\pi_{K-1}} \dots P^{\pi_{i+1}}] 
      \abs{\varrho_{i+1}} \right)
  \end{align}
  Notice that there are $K-i+j$ $P$-operators on both terms
  in the sum. Therefore were can employ \cref{lem:MRN} twice.
  Moreover define
  $\varepsilon_{\max} = \max_{i \in [K]} \norm{\varrho_i}_{2,\nu}$.
  Then
  \begin{align}
    \E_\mu(O_i \abs{\varrho_{i+1}}) &\leq (1-\gamma)
    \sum_{j=0}^\infty \gamma^j \kappa(K-i+j;\mu,\nu) \norm{\varrho_{i+1}}_{2,\nu}
    \notag
    \\ &\leq \varepsilon_{\max} (1-\gamma)
    \sum_{j=0}^\infty \gamma^j \kappa(K-i+j;\mu,\nu) 
    \label{eq:step4_1}
  \end{align} 
  Using \cref{eq:step3_final}, \cref{eq:step3_crude} and \cref{eq:step4_1}
  \begin{equation}
    \begin{split}
    \norm{Q^* - Q^{\pi_K}}_{1,\mu} \leq
    \frac{2 \gamma (1- \gamma^{K+1})}{1-\gamma} 
    \left[ \sum_{i=0}^{K-1} \sum_{j=0}^\infty
    \alpha_i \gamma^j \kappa(K-i+j; \mu, \nu) \right] \varepsilon_{\max}
    \\ + \frac{4 \gamma (1-\gamma^{K+1})}{(1-\gamma)^3} \alpha_K R_{\max}
  \end{split}
  \label{eq:step4_2}
  \end{equation} 
  Focusing on the first term on RHS of \cref{eq:step4_2}, if we 
  then we can take the norm out of the sum as a constant. We are left with
  \begin{align}
    & \sum_{i=0}^{K-1} \sum_{j=0}^\infty \alpha_i \gamma^j \kappa(K-i+j;\mu,\nu)
    \notag
    \\ &= \sum_{i=0}^{K-1} \sum_{j=0}^\infty  
    \frac{(1-\gamma) \gamma^{K-i+j-1}}{1-\gamma^{K+1}} \kappa(K-i+j;\mu,\nu)
    \notag
    \\ &= \frac{1-\gamma}{1-\gamma^{K+1}} \sum_{j=0}^\infty \sum_{i=0}^{K-1} 
    \gamma^{K-i+j-1} \kappa(K-i+j;\mu,\nu) \notag
    \\ &\leq \frac{1-\gamma}{1-\gamma^{K+1}} \sum_{m=0}^\infty
    \gamma^{m-1} \cdot m \cdot \kappa(m; \mu, \nu) \notag
    \\ &\leq \frac{1}{1-\gamma^{K+1}(1-\gamma)} \phi_{\mu,\nu}
    \label{eq:step4_3}
  \end{align}
  Where the last inequality is due to \cref{asm:A2}.
  %todo reference assumption 2 (concentration coefficients)
  Combining \cref{eq:step4_2} and \cref{eq:step4_3} we arrive at
  \begin{equation}
    \norm{Q^* - Q^{\pi_K}}_{1,\mu} \leq
    \frac{2\gamma \cdot \phi_{\mu,\nu}}{(1-\gamma)^2} \cdot \varepsilon_{\max}
    + \frac{4\gamma^{K+1}}{(1-\gamma)^2} \cdot R_{\max}
  \end{equation}
\end{proof}


